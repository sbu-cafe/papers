%\documentclass[12pt, preprint]{aastex}
\documentclass[preprint2,times]{aastex6}
%\documentclass[preprint,times]{aastex6}

% these lines seem necessary for pdflatex to get the paper size right
\pdfpagewidth 8.5in
\pdfpageheight 11.0in


\usepackage{epsf,color,amsmath}

\usepackage{cancel}

\newcommand{\sfrac}[2]{\mathchoice%
  {\kern0em\raise.5ex\hbox{\the\scriptfont0 #1}\kern-.15em/
    \kern-.15em\lower.25ex\hbox{\the\scriptfont0 #2}}
  {\kern0em\raise.5ex\hbox{\the\scriptfont0 #1}\kern-.15em/
    \kern-.15em\lower.25ex\hbox{\the\scriptfont0 #2}}
  {\kern0em\raise.5ex\hbox{\the\scriptscriptfont0 #1}\kern-.2em/
    \kern-.15em\lower.25ex\hbox{\the\scriptscriptfont0 #2}} {#1\!/#2}}


\newcommand{\castro}{{\sf Castro}}

\newcommand{\nablab}{{\mathbf{\nabla}}}
\newcommand{\Ub}{\mathbf{U}}
\newcommand{\gb}{\mathbf{g}}
\newcommand{\omegadot}{\dot{\omega}}
\newcommand{\Sdot}{\dot{S}}
\newcommand{\ddx}[1]{{\frac{{\partial#1}}{\partial x}}}
\newcommand{\ddt}[1]{{\frac{{\partial#1}}{\partial t}}}
\newcommand{\odt}[1]{{\frac{{d#1}}{dt}}}
\newcommand{\divg}[1]{{\nablab \cdot \left (#1\right)}}

\newcommand{\Ic}{\mathcal{I}}
\newcommand{\smax}{{s_\mathrm{max}}}

\usepackage{bm}

\newcommand{\Uc}{{\bm{\mathcal{U}}}}
\newcommand{\Fb}{\mathbf{F}}
\newcommand{\Sc}{\mathbf{S}}

\newcommand{\xv}{{(x)}}
\newcommand{\yv}{{(y)}}
\newcommand{\zv}{{(z)}}

\newcommand{\ex}{{\bf e}_x}
\newcommand{\ey}{{\bf e}_y}
\newcommand{\ez}{{\bf e}_z}

\newcommand{\Ab}{{\bf A}}
\newcommand{\Sq}{{\bf S}_\qb}
\newcommand{\Sqhydro}{{\Sq^{\mathrm{hydro}}}}
\newcommand{\qb}{{\bf q}}

\newcommand{\Shydro}{{{\bf S}^{\mathrm{hydro}}}}
\newcommand{\Rb}{{\bf R}}
\newcommand{\Rq}{{\bf R}}
\newcommand{\Adv}[1]{{\left [\mathcal{A} \left(#1\right)\right]}}
\newcommand{\Advt}[1]{{\left [\mathcal{\tilde{A}} \left(#1\right)\right]}}
\newcommand{\Advs}[1]{{\mathcal{A} \left(#1\right)}}

\setlength{\marginparwidth}{0.75in}
\newcommand{\MarginPar}[1]{\marginpar{\vskip-\baselineskip\raggedright\tiny\sffamily\hrule\smallskip{\color{red}#1}\par\smallskip\hrule}}

\begin{document}
%======================================================================
% Title
%======================================================================
\title{CASTRO: A New Compressible Astrophysical Solver. IV. Performance Portability}

\shorttitle{}
\shortauthors{}

\author{Castro developers / M.~P.~Katz et al.}
\affil{}
\email{}


%======================================================================
% Abstract and Keywords
%======================================================================
\begin{abstract}
blah
\end{abstract}

\keywords{hydrodynamics---methods: numerical}

%======================================================================
% Introduction
%======================================================================
\section{Introduction}\label{Sec:Introduction}

\citep{castro}

Existing GPU-enabled astrophysical hydro codes include
\citet{gamer,cholla,fargo3d,pekkila:2017}

GPU-enabled integration has been explored by \citep{brock:2015}.

\section{Performance portable compute kernels}

Describe mfiter structure

Describe MPI, MPI+OpenMP w/ tiling, MPI + CUDA

\begin{figure}
\centering
\includegraphics[height=0.25\textheight]{gpu_1} \\
\includegraphics[height=0.25\textheight]{gpu_2} \\
\includegraphics[height=0.25\textheight]{gpu_3}
\caption{\label{fig:loops} caption}
\end{figure}

\section{Scaling}



\acknowledgements \castro\ is freely available at
\url{http://github.com/AMReX-Astro/Castro}.  The work at Stony Brook
was supported by DOE/Office of Nuclear Physics grant DE-FG02-87ER40317
and NSF award AST-1211563.  An award of computer time was provided by
the Innovative and Novel Computational Impact on Theory and Experiment
(INCITE) program.  This research used resources of the Oak Ridge
Leadership Computing Facility at the Oak Ridge National Laboratory,
which is supported by the Office of Science of the U.S. Department of
Energy under Contract No. DE-AC05-00OR22725.  We thank NVIDIA Corporation
for the donation of a Titan X Pascal used in this research.





%======================================================================
% References
%======================================================================

\bibliographystyle{apj}
\bibliography{ws}

\end{document}
