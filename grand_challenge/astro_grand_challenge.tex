\documentclass[11pt]{article}

\usepackage[margin=1in]{geometry}

\usepackage{mathpazo}

\usepackage{fancyhdr}
\pagestyle{fancy}
\renewcommand{\headrulewidth}{0pt}

\rfoot{\footnotesize \em Future Directions in Extreme Scale\\ Computing for Scientific Grand Challenges}

\begin{document}

\begin{center}
{\Large Challenges in Modeling Astrophysical Thermonuclear Explosions} \\[0.25em]
Michael Zingale (Stony Brook)
\end{center}

The end states of stellar evolution result in exotic compact objects:
white dwarfs, neutron stars, and black holes.  When these remnants are
in binary systems, interactions with the companion star can lead to a
variety of different thermonuclear explosions: classical novae, X-ray
bursts (XRBs), and thermonuclear (Type Ia) supernovae (SN Ia).  These
explosions are important sites of nucleosynthesis in the galaxy.  XRBs
allow us to probe the physics of dense nuclear matter, and SN Ia are
important distant indicators for measuring the expansion history of
the Universe.  Simulations on these events has been ongoing for
decades and allow us to understand the physics of the explosion,
interpret the observations, and test different progenitor models for
viability.

Modeling these events require multiscale and multiphysics simulations.
Spatial scales range from the size of the white dwarf or neutron star
down to the burning or dissipation scale.  Temporary scales range from
the millions of years (SN Ia) to hours (XRBs) of evolution that
brought the star to the brink of explosions to the seconds-long
explosion.  Models must incorporate hydrodynamics, nuclear reactions,
gravity, magnetic fields, and radiation.

algorithms: hydrodynamics, finite-volume, AMR

new developments: low speed / implicit, integration techniques, coupling

performance: GPUs \& CPUs

Example: XRBs


\end{document}
