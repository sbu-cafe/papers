\documentclass[linenumbers,trackchanges]{aastex631}

%\usepackage{xcolor,amsmath}

\usepackage[T1]{fontenc}

\newcommand{\sfrac}[2]{\mathchoice%
  {\kern0em\raise.5ex\hbox{\the\scriptfont0 #1}\kern-.15em/
    \kern-.15em\lower.25ex\hbox{\the\scriptfont0 #2}}
  {\kern0em\raise.5ex\hbox{\the\scriptfont0 #1}\kern-.15em/
    \kern-.15em\lower.25ex\hbox{\the\scriptfont0 #2}}
  {\kern0em\raise.5ex\hbox{\the\scriptscriptfont0 #1}\kern-.2em/
    \kern-.15em\lower.25ex\hbox{\the\scriptscriptfont0 #2}} {#1\!/#2}}

\newcommand{\myhalf}{\sfrac{1}{2}}
\newcommand{\nph}{{n+\myhalf}}
\newcommand{\nmh}{{n-\myhalf}}

\newcommand{\inp}{\mathrm{in}}
\newcommand{\outp}{\mathrm{out}}

% boldsymbol means bold italic
\newcommand{\eb}{{\bf{e}}}
\newcommand{\Ub}{{\bf{U}}}
\newcommand{\xb}{{\bf{x}}}
\newcommand{\kb}{{\bf{k}}}
\newcommand{\Vb}{{\bf{V}}_n}
\newcommand{\Vbhat}{{\bf{\widehat{V}}}_n}
\newcommand{\Omegab}{{\bf{\Omega}}}
\newcommand{\gb}{{\bf{g}}}
\newcommand{\rb}{{\bf{r}}}

\newcommand{\pb}{p_\mathrm{base}}
\newcommand{\epsdot}{\dot{\epsilon}}
\newcommand{\qburn}{q_\mathrm{burn}}
\newcommand{\rt}{\tilde{r}_0}


\newcommand{\nablab}{\mathbf{\nabla}}
\newcommand{\dt}{\Delta\ t}

\newcommand{\omegadot}{\dot{\omega}}

\newcommand{\Hext}{H_{\rm ext}}
\newcommand{\Hnuc}{H_{\rm nuc}}
\newcommand{\kth}{k_{\rm th}}

\newcommand{\Gammaonebar}{\overline{\Gamma}_1}
\newcommand{\Sbar}{\overline{S}}

\newcommand{\etarho}{\eta_\rho}
\newcommand{\etarhoh}{\eta_{\rho~h}}

\newcommand{\Ubt}{\widetilde{\Ub}}
\newcommand{\wt}{\widetilde{w}}

\newcommand{\He}{$^4$He}
\newcommand{\C}{$^{12}$C}
\newcommand{\Fe}{$^{56}$Fe}

\newcommand{\isot}[2]{$^{#2}\mathrm{#1}$}
\newcommand{\isotm}[2]{{}^{#2}\mathrm{#1}}

\newcommand{\maestro}{{\sf MAESTRO}}
\newcommand{\castro}{{\sf Castro}}
\newcommand{\amrex}{{\sf AMReX}}
\newcommand{\pynucastro}{{\sf pynucastro}}

\newcommand{\avg}[1]{\overline{#1}}
\newcommand{\avgtwod}[1]{\langle~#1 \rangle}
\newcommand{\rms}[2]{\left(\delta#1\right)_{r_{#2}}}
\newcommand{\mymax}[1]{\left(#1\right)_{\rm max}}

\newcommand{\Tb}{\ensuremath{T_\mathrm{base}}}
\newcommand{\gcc}{\mathrm{g~cm^{-3} }}
\newcommand{\cms}{\mathrm{cm~s^{-1} }}

\newcommand{\half}{\frac{1}{2}}


\newcommand{\ddx}[1]{{\frac{{\partial#1}}{\partial x}}}
\newcommand{\ddxs}[1]{{\frac{{\partial}}{\partial x}}#1}
\newcommand{\ddt}[1]{{\frac{{\partial#1}}{\partial t}}}
\newcommand{\odt}[1]{{\frac{{d#1}}{dt}}}
\newcommand{\divg}[1]{{\nablab \cdot \left (#1\right)}}
\newcommand{\dedr}{\left . {\partial{}e}/{\partial\rho}\right |_{T, X_k}}
\newcommand{\dedrd}{\left . \frac{\partial{}e}{\partial\rho}\right |_{T, X_k}}
\newcommand{\dedX}{\left . {\partial{}e}/{\partial{}X_k} \right |_{\rho, T}}
\newcommand{\dedXd}{\left . \frac{\partial{}e}{\partial{}X_k} \right |_{\rho, T, X_{j,j\ne k}}}
\newcommand{\dedT}{\left . {\partial{}e}/{\partial{}T} \right |_{\rho,X_k}}
\newcommand{\dedTd}{\left . \frac{\partial{}e}{\partial{}T} \right |_{\rho,X_k}}

\newcommand{\Ic}{{\boldsymbol{\mathcal{I}}}}
\newcommand{\Ics}{{\mathcal{I}}}
\newcommand{\kmax}{{k_\mathrm{max}}}
\usepackage{bm}

\newcommand{\Uc}{{\,\bm{\mathcal{U}}}}
\newcommand{\Fb}{\mathbf{F}}
\newcommand{\Sc}{\mathbf{S}}

\newcommand{\xv}{{(x)}}
\newcommand{\yv}{{(y)}}
\newcommand{\zv}{{(z)}}

\newcommand{\ex}{{\bf e}_x}
\newcommand{\ey}{{\bf e}_y}
\newcommand{\ez}{{\bf e}_z}

\newcommand{\Ab}{{\bf A}}
\newcommand{\Sq}{{\bf S}_\qb}
\newcommand{\Sqhydro}{{\Sq^{\mathrm{hydro}}}}
\newcommand{\qb}{{\bf q}}

\newcommand{\Gb}{{\bf G}}
\newcommand{\Rb}{{\bf R}}
\newcommand{\Rq}{{\bf R}}
\newcommand{\Adv}[1]{{\left [\boldsymbol{\mathcal{A}} \left(#1\right)\right]}}
\newcommand{\Advt}[1]{{\left [\boldsymbol{\mathcal{\tilde{A}}} \left(#1\right)\right]}}
\newcommand{\Advss}[1]{{\left [{\mathcal{{A}}} \left(#1\right)\right]}}
\newcommand{\Advsst}[1]{{\left [{\mathcal{\tilde{A}}} \left(#1\right)\right]}}
\newcommand{\Advs}[1]{\boldsymbol{\mathcal{A}} \left(#1\right)}



\setlength{\marginparwidth}{0.5in}

\newcommand{\MarginPar}[1]{
    \marginpar{\vskip-\baselineskip%
               \raggedright%
               \tiny\sffamily%
               {\color{red}\hrule%
               \smallskip%
               #1\par%
               \smallskip%
               \hrule}}%
}

\newcommand{\AssignTo}[1]{
    \marginpar{\vskip-\baselineskip%
               \raggedright%
               \tiny\sffamily%
               {\color{blue}\hrule%
               \smallskip%
               #1\par%
               \smallskip%
               \hrule}}%
}

\begin{document}
%======================================================================
% Title
%======================================================================
\title{Influence of Time Integration Methodology on Double Detonation Type
Ia Supernova Nucleosynthesis}

\shorttitle{Integration Methods and SNe Ia Nucleosynthesis}

\author[0000-0001-8401-030X]{Michael Zingale}
\affiliation{Dept.\ of Physics and Astronomy, Stony Brook University,
             Stony Brook, NY 11794-3800, USA}

\author[0000-0002-2839-107X]{Zhi Chen}
\affiliation{Department of Physics and Astronomy,
Stony Brook University,
Stony Brook, NY 11794-3800, USA}


\author[0000-0003-0439-4556]{Max Katz}
\affiliation{Dept.\ of Physics and Astronomy, Stony Brook University,
             Stony Brook, NY 11794-3800, USA}


\author[0000-0001-5961-1680]{Alexander Smith Clark}
\affiliation{Department of Physics and Astronomy,
Stony Brook University,
Stony Brook, NY 11794-3800, USA}

\author[0000-0003-3603-6868]{Eric T. Johnson}
\affiliation{Department of Physics and Astronomy,
Stony Brook University,
Stony Brook, NY 11794-3800, USA}

\correspondingauthor{Michael Zingale}
\email{michael.zingale@stonybrook.edu}


%======================================================================
% Abstract and Keywords
%======================================================================
\begin{abstract}
We explore the coupling of hydrodynamics and reactions in simulations of
the double detonation model for Type Ia supernovae.
\end{abstract}

\keywords{convection---hydrodynamics---methods: numerical}

%======================================================================
% Introduction
%======================================================================
\section{Introduction}\label{Sec:Introduction}

The double detonation model


These events are modeled using the Euler equations with reactive and
gravitational sources:
\begin{eqnarray*}
\ddt{\rho} + \nabla \cdot (\rho \Ub) &=& 0 \\
\ddt{(\rho \Ub)} + \nabla \cdot (\rho \Ub \Ub) + \nabla p &=& \rho \gb \\
\ddt{(\rho E)} + \nabla \cdot (\rho \Ub E + \Ub p) &=& \rho \Ub \cdot \gb + \rho \dot{S} \\
\ddt{(\rho X_k)} + \nabla \cdot (\rho \Ub X_k) &=& \rho \omegadot_k
\end{eqnarray*}

Integration of reaction networks takes various forms in astrophysical
simulation codes.  The most common method is via operator splitting:
the advection and reactive terms are treated independently, with each
process working on the result of the other.  Often, Strang-splitting
\citep{strang:1968} is used, which alternates advection and reaction
to yield second-order accuracy in time.

When using operator-splitting, the density remains constant, since there
is no reaction source in the mass continuity equation.  The mass
fractions, $X_k$, evolve according to:
\begin{equation}
\frac{dX_k}{dt} = \omegadot_k
\end{equation}
where $\omegadot_k$ is the creation rate of species $k$ due to nuclear
reaction.  In the simplest approximation of operator splitting, the
evolution of the mass fractions is solved alone, without integrating
the temperature or energy.  This is the form that was used in FLASH
\cite{flash}.  A more accurate operator splitting also includes the
energy evolution (perhaps in terms of temperature).  This will require
calling the equation of state each time we evaluate the righthand side
of the ODE system, increasing the cost.  However, as we showed in
\citet{strang_rnaas}, this is needed to get second-order convergence.

Detonations can be difficult to model numerically.  Time step cuts \MarginPar{ref}.
For the double detonation scenario, the ignition of the second detonation
can be very sensitive to how the simulation is run \MarginPar{refs}.
Artifically limiting the rates \MarginPar{ref}

In \citet{castro_simple_sdc}, we introduced a method based on the
ideas of spectral deferred corrections for coupling reactions and
hydrodyamics (that we called ``simplified-SDC'').  Here, the overall
time-integration is done iteratively, with the hydrodynamics seeing an
explicit reactive source, and the reaction update evolving an ODE
system that includes a piecewise-constant-in-time advective source,
$\Adv{\Uc}^{n+1/2}$,
\begin{equation}
\ddt{\Uc} = \Rb(\Uc) + \Adv{\Uc}^{n+1/2}
\end{equation}

In addition to the time-integration strategy, integrating the reaction
ODE system requires specifying tolerances.  Unfortunately, the tolerances
used are not normally reported in papers, and in many cases, may not
be easily controlled by a code user at runtime.



\citet{castro_simple_sdc} looked at an extreme version of the
double detonation scenario---a very large perturbation was applied
to drive a detonation directly into the underlying
carbon-oxygen white dwarf.  We showed in that paper that the Strang
split integration method had difficulty with the integration unless
we used tighter tolerances

Summary of the SDC subch results form the original paper

%======================================================================
% Results
%======================================================================
\section{Simulations and Results}\label{Sec:results}


We use a 2d axisymmetric domain with a size of $5.12\times 10^9~\mathrm{cm}$ by $1.024\times 10^{10}~\mathrm{cm}$.  This is
much larger than the initial size of the star, giving it plenty of
room to expand.  The coarse grid is $640\times 1280$ zones and we use
2 levels of refinement, giving a maximum resolution of
$20~\mathrm{km}$.  The refinement strategy is picked to refine on the
star and any regions where the temperature is greater than
$10^8~\mathrm{K}$.  \castro\ uses subcycling, so finer grids are
advanced at a smaller timestep than the coarse grids.  Self-gravity is
done using a full Poisson solve using geometric multigrid, with
Dirichlet boundary conditions on the domain boundary computed via a
multipole expansion with a maximum order of 6.  For the low density
regions outside of the star, we use a sponge term on the momentum
equation to prevent the very low density material that is not in
hydrostatic equilibrium from raining down on the star.

Our default CFL number is 0.2 for these simulations, and the CFL constraint
is the only timestep constraint used.  For larger CFL numbers, the ODE integration
sometimes encounters problems \MarginPar{more}.

All simulations use the same network, the {\tt subch\_simple} as
described in \cite{zhi2023}.  This includes 22 nuclei and 94 rates
(from ReacLib \citealt{reaclib}) and is produced with \pynucastro
\citep{pynucastro2}.  Importantly, we include
$\isotm{N}{14}(\alpha,\gamma)\isotm{F}{18}(\alpha,p)\isotm{Ne}{21}$
which creates protons to allow for
$\isotm{C}{12}(p,\gamma)\isotm{N}{13}(\alpha,p)\isotm{O}{16}$.  This
sequence can be faster than
$\isotm{C}{12}(\alpha,\gamma)\isotm{O}{16}$, as pointed out by
\citet{shenbildsten} and is important for getting the detonation speed
correctly.  To reduce the size of the network, some
$(\alpha,p)(p,\gamma)$ and $(\alpha,\gamma)$ rates are combined into
an effective $(\alpha,\gamma)$ rate.  Screening is provided following
the procedure in \citet{wallace:1982}, combining the screening
functions of \citet{graboske:1973,alastuey:1978,itoh:1979}.

The reaction system is integrated using a version of the VODE ODE integrator
\citep{vode} ported to C++.  Our modifications to this integrator are described
in \citet{castro_simple_sdc}.  We use an analytic approximation to the Jacobian
and found that we get best results when we disable the caching of the Jacobian
in the integrator.

If during the advance of a timestep an error is generated (negative
density, ODE integration fails, mass fractions don't sum to 1, CFL constraint violated at the new time), the
step is thrown-out and retied with a smaller timestep.  This is done
on a level-by-level basis in the overall AMR subcycling hierarchy.

We will run a suite of simulations all using the same initial model.
The initial model is constructed following the methodology in
\citet{subchandra}.  We use an isothermal core of C/O of $1.1~M_\odot$
with $T = 10^7~\mathrm{K}$ and a thin transition region where the
temperature ramps up to $1.75\times 10^8~\mathrm{K}$ and the
composition changes to 99\% \isot{He}{4} and 1\% \isot{N}{14}.    This He envelope is then integrated isentropically.  The
density of the transition from the underlying CO white dwarf to the He
envelope was selected to yield an envelope mass of $0.05~M_\odot$.

We place a small temperature perturbation in the He layer using the
same form as in \citet{castro_simple_sdc}:
\begin{equation}
  T = T_0 \left [ 1 + X(\isotm{He}{4}) f (1 + \tanh(2 - r_1)) \right ]
\end{equation}
where
\begin{equation}
  r_1 = \left [ x^2 + (y - r_0)^2 \right ]^{1/2} / \lambda
\end{equation}
and
\begin{equation}
  r_0 = r_\mathrm{pert} + r_\mathrm{base}
\end{equation}
Here, $r_\mathrm{base}$ is the radius at which the helium layer begins
and $r_\mathrm{pert}$ is the distance above the base to put the
perturbation.  We choose $r_\mathrm{pert} = 100~\mathrm{km}$.  The
temperature is perturbed above the initial model value, denoted as
$T_0$ here.  The amplitude of the perturbation is $f = 3$ and the
scale of the perturbation is $\lambda = 12.5~\mathrm{km}$, and
$r_\mathrm{pert} = 100~\mathrm{km}$.  This is a very small
perturbation, but enough to seed the initial He detonation in the
envelope.



The simulations are all run on the OLCF Frontier machine, using 4
nodes / 32 AMD GPUs.  The data is moved to the GPUs at the start of
the simulation and all computation is done there.  Our GPU offloading
strategy takes advantage of the \amrex\ C++ lambda-capturing
functionality to be performance portable.


We run 3 different types of integration methods: traditional operator
splitting, using Strang splitting, both with integration of the energy
equation and without any temperature evolution during the burn.  \MarginPar{comment about the EOS}.  These methods are compared to the simplified-SDC integtation.

\begin{figure*}[t]
\centering
\plotone{subch_Temp_sequence}
\caption{\label{fig:temp_sequence} Time-sequence of the SDC run showing the temperature.}
\end{figure*}

\begin{figure*}[t]
\centering
\plotone{subch_lap_rho_sequence}
\caption{\label{fig:lap_rho_sequence} Time-sequence of the SDC run showing the compression.}
\end{figure*}


Figure~\ref{fig:lap_rho_sequence} shows a Schlieren-style plot, which
highlights density gradients.  It is constructed by plotting
$\log_{10}(|\rho^{-1}\nabla^2\rho|)$.  This clearly shows the
compression wave launched by the He detonation propagating in toward
the center of the white dwarf.  The middle row shows this compression
wave converging slightly off-center which then leads to the ignition
of the C detonation.  This also highlights how the speed of the He
detonation is important in determining how off-center the C detonation
ignition will be.  Our choice of network was motivated by ensuring
that we capture the energy release from He burning accurately, in
particular the inclusion of the
$\isotm{C}{12}(p,\gamma)\isotm{N}{13}(\alpha,p)\isotm{O}{16}$
sequence.


We also vary the CFL number and the integration tolerances.

Show Ni56 mass, intermediate mass

\begin{figure}[t]
\centering
\plotone{subch_ni56}
\caption{\label{fig:ni56} \isot{Ni}{56} mass vs.\ time for the different simulations.}
\end{figure}

\begin{figure}[t]
\centering
\plotone{subch_cpu}
\caption{\label{fig:cpu} MPI-hours of GPU time for the different simulations.}
\end{figure}


Show time-sequence + compression plot

Figure~\ref{fig:cpu} shows the computational expense of the
simulations, in terms of node hours.  All simulations were run on 4
nodes (32 AMD GPUs total).  We note that we have not spent any time on
load-balancing these simulations, so we should consider this to be
only a guide.  At the start of the simulation, there are 56 boxes on
the finest grid (representing 0.77\% of the domain) and at 1 s of
evolution there is are 60 boxes covering 5.64\% of the domain.  The
figure shows that the Strang simulation with a CFL of 0.2 is the
cheapest, but as we saw above, this does not get the nucleosynthesis
correct.  All of the other Strang simulations are more costly than the
baseline SDC simulation.  This shows that not only does the
simplified-SDC simulation accurately evolve the nucleosynthesis, it
does so in a very efficient fashion.

\section{Summary}

Our results suggest that exploring the time-integration method and
integrator tolerances is an important part of demonstrating
convergence of simulations involving explosive reacting flows in
astrophysics.

The ignition of the carbon detonation remains numerically challenging
for the ODE integrator, especially with large
timesteps.  \MarginPar{add some info on why it fails} In the near
future we will show how to include various nuclear statistical
equilibrium approximations into the simplified-SDC formalism while
retaining second-order accuracy.  We will also explore different
integrators and auto-differentiation for the Jacobian (in particular,
the compositional contribution from the screening)

We encourage authors to include a discussion about the reaction
integration algorithm and tolerances in papers to permit comparisons
in the future.

\begin{acknowledgements}
\castro\ is open-source and freely available at
\url{http://github.com/AMReX-Astro/Castro}.  The problem setup used
here is available in the git repo as {\tt flame\_wave}. The metadata
describing the build environment and the global diagnostic output
files are available on Zenodo at \citet{xrb_data}.  We thank Alice
Harpole for contributions to the AMReX Astrophysics suite.

The work at Stony Brook was supported by DOE/Office of Nuclear
Physics grant DE-FG02-87ER40317.  This research used resources of the
National Energy Research Scientific Computing Center, a DOE Office of
Science User Facility supported by the Office of Science of the
U.~S.\ Department of Energy under Contract No.\ DE-AC02-05CH11231.
This research was supported by the Exascale Computing Project
(17-SC-20-SC), a collaborative effort of the U.S. Department of Energy
Office of Science and the National Nuclear Security Administration.
This research used resources of the Oak Ridge Leadership Computing
Facility at the Oak Ridge National Laboratory, which is supported by
the Office of Science of the U.S. Department of Energy under Contract
No. DE-AC05-00OR22725, awarded through the DOE INCITE program.  We
thank NVIDIA Corporation for the donation of a Titan X and Titan V GPU
through their academic grant program.  This research has made use of
NASA's Astrophysics Data System Bibliographic Services.
\end{acknowledgements}

\facilities{NERSC, OLCF}

\software{\amrex~\citep{amrex_joss},
          \castro~\citep{castro,castro_joss},
          GCC (\url{https://gcc.gnu.org/}),
          helmeos \citep{timmes_swesty:2000},
          linux (\url{https://www.kernel.org/}),
          matplotlib (\citealt{Hunter:2007}, \url{http://matplotlib.org/}),
          NumPy \citep{numpy,numpy2},
          python (\url{https://www.python.org/}),
          valgrind \citep{valgrind},
          VODE \citep{vode},
          yt \citep{yt}}



%======================================================================
% References
%======================================================================

\bibliographystyle{aasjournal}
\bibliography{ws}


\end{document}
