\documentclass[modern]{aastex631}

%\usepackage{xcolor,amsmath}

\usepackage[T1]{fontenc}

\newcommand{\sfrac}[2]{\mathchoice%
  {\kern0em\raise.5ex\hbox{\the\scriptfont0 #1}\kern-.15em/
    \kern-.15em\lower.25ex\hbox{\the\scriptfont0 #2}}
  {\kern0em\raise.5ex\hbox{\the\scriptfont0 #1}\kern-.15em/
    \kern-.15em\lower.25ex\hbox{\the\scriptfont0 #2}}
  {\kern0em\raise.5ex\hbox{\the\scriptscriptfont0 #1}\kern-.2em/
    \kern-.15em\lower.25ex\hbox{\the\scriptscriptfont0 #2}} {#1\!/#2}}

\newcommand{\myhalf}{\sfrac{1}{2}}
\newcommand{\nph}{{n+\myhalf}}
\newcommand{\nmh}{{n-\myhalf}}

\newcommand{\inp}{\mathrm{in}}
\newcommand{\outp}{\mathrm{out}}

% boldsymbol means bold italic
\newcommand{\eb}{{\bf{e}}}
\newcommand{\Ub}{{\bf{U}}}
\newcommand{\xb}{{\bf{x}}}
\newcommand{\kb}{{\bf{k}}}
\newcommand{\Vb}{{\bf{V}}_n}
\newcommand{\Vbhat}{{\bf{\widehat{V}}}_n}
\newcommand{\Omegab}{{\bf{\Omega}}}
\newcommand{\gb}{{\bf{g}}}
\newcommand{\rb}{{\bf{r}}}

\newcommand{\pb}{p_\mathrm{base}}
\newcommand{\epsdot}{\dot{\epsilon}}
\newcommand{\qburn}{q_\mathrm{burn}}
\newcommand{\rt}{\tilde{r}_0}


\newcommand{\nablab}{\mathbf{\nabla}}
\newcommand{\dt}{\Delta\ t}

\newcommand{\omegadot}{\dot{\omega}}

\newcommand{\Hext}{H_{\rm ext}}
\newcommand{\Hnuc}{H_{\rm nuc}}
\newcommand{\kth}{k_{\rm th}}

\newcommand{\Gammaonebar}{\overline{\Gamma}_1}
\newcommand{\Sbar}{\overline{S}}

\newcommand{\etarho}{\eta_\rho}
\newcommand{\etarhoh}{\eta_{\rho~h}}

\newcommand{\Ubt}{\widetilde{\Ub}}
\newcommand{\wt}{\widetilde{w}}

\newcommand{\He}{$^4$He}
\newcommand{\C}{$^{12}$C}
\newcommand{\Fe}{$^{56}$Fe}

\newcommand{\isot}[2]{$^{#2}\mathrm{#1}$}
\newcommand{\isotm}[2]{{}^{#2}\mathrm{#1}}

\newcommand{\maestro}{{\sf MAESTRO}}
\newcommand{\castro}{{\sf Castro}}
\newcommand{\amrex}{{\sf AMReX}}
\newcommand{\pynucastro}{{\sf pynucastro}}

\newcommand{\avg}[1]{\overline{#1}}
\newcommand{\avgtwod}[1]{\langle~#1 \rangle}
\newcommand{\rms}[2]{\left(\delta#1\right)_{r_{#2}}}
\newcommand{\mymax}[1]{\left(#1\right)_{\rm max}}

\newcommand{\Tb}{\ensuremath{T_\mathrm{base}}}
\newcommand{\gcc}{\mathrm{g~cm^{-3} }}
\newcommand{\cms}{\mathrm{cm~s^{-1} }}

\newcommand{\half}{\frac{1}{2}}


\newcommand{\ddx}[1]{{\frac{{\partial#1}}{\partial x}}}
\newcommand{\ddxs}[1]{{\frac{{\partial}}{\partial x}}#1}
\newcommand{\ddt}[1]{{\frac{{\partial#1}}{\partial t}}}
\newcommand{\odt}[1]{{\frac{{d#1}}{dt}}}
\newcommand{\divg}[1]{{\nablab \cdot \left (#1\right)}}
\newcommand{\dedr}{\left . {\partial{}e}/{\partial\rho}\right |_{T, X_k}}
\newcommand{\dedrd}{\left . \frac{\partial{}e}{\partial\rho}\right |_{T, X_k}}
\newcommand{\dedX}{\left . {\partial{}e}/{\partial{}X_k} \right |_{\rho, T}}
\newcommand{\dedXd}{\left . \frac{\partial{}e}{\partial{}X_k} \right |_{\rho, T, X_{j,j\ne k}}}
\newcommand{\dedT}{\left . {\partial{}e}/{\partial{}T} \right |_{\rho,X_k}}
\newcommand{\dedTd}{\left . \frac{\partial{}e}{\partial{}T} \right |_{\rho,X_k}}

\newcommand{\Ic}{{\boldsymbol{\mathcal{I}}}}
\newcommand{\Ics}{{\mathcal{I}}}
\newcommand{\kmax}{{k_\mathrm{max}}}
\usepackage{bm}

\newcommand{\Uc}{{\,\bm{\mathcal{U}}}}
\newcommand{\Fb}{\mathbf{F}}
\newcommand{\Sc}{\mathbf{S}}

\newcommand{\xv}{{(x)}}
\newcommand{\yv}{{(y)}}
\newcommand{\zv}{{(z)}}

\newcommand{\ex}{{\bf e}_x}
\newcommand{\ey}{{\bf e}_y}
\newcommand{\ez}{{\bf e}_z}

\newcommand{\Ab}{{\bf A}}
\newcommand{\Sq}{{\bf S}_\qb}
\newcommand{\Sqhydro}{{\Sq^{\mathrm{hydro}}}}
\newcommand{\qb}{{\bf q}}

\newcommand{\Gb}{{\bf G}}
\newcommand{\Rb}{{\bf R}}
\newcommand{\Rq}{{\bf R}}
\newcommand{\Adv}[1]{{\left [\boldsymbol{\mathcal{A}} \left(#1\right)\right]}}
\newcommand{\Advt}[1]{{\left [\boldsymbol{\mathcal{\tilde{A}}} \left(#1\right)\right]}}
\newcommand{\Advss}[1]{{\left [{\mathcal{{A}}} \left(#1\right)\right]}}
\newcommand{\Advsst}[1]{{\left [{\mathcal{\tilde{A}}} \left(#1\right)\right]}}
\newcommand{\Advs}[1]{\boldsymbol{\mathcal{A}} \left(#1\right)}



\setlength{\marginparwidth}{0.5in}

\newcommand{\MarginPar}[1]{
    \marginpar{\vskip-\baselineskip%
               \raggedright%
               \tiny\sffamily%
               {\color{red}\hrule%
               \smallskip%
               #1\par%
               \smallskip%
               \hrule}}%
}

\newcommand{\AssignTo}[1]{
    \marginpar{\vskip-\baselineskip%
               \raggedright%
               \tiny\sffamily%
               {\color{blue}\hrule%
               \smallskip%
               #1\par%
               \smallskip%
               \hrule}}%
}

\begin{document}
%======================================================================
% Title
%======================================================================
\title{Influence of Time Integration Methodology on Double Detonation Type
Ia Supernova Nucleosynthesis}

\shorttitle{Integration Methods and SNe Ia Nucleosynthesis}

\author[0000-0001-8401-030X]{Michael Zingale}
\affiliation{Dept.\ of Physics and Astronomy, Stony Brook University,
             Stony Brook, NY 11794-3800, USA}

\author[0000-0002-2839-107X]{Zhi Chen}
\affiliation{Department of Physics and Astronomy,
Stony Brook University,
Stony Brook, NY 11794-3800, USA}


\author{Melissa Rasmussen}
\affiliation{Stony Brook? Utah?}

\author{Abigail Polin}
\affiliation{Caltech}

\author[0000-0003-0439-4556]{Max Katz}
\affiliation{Dept.\ of Physics and Astronomy, Stony Brook University,
             Stony Brook, NY 11794-3800, USA}


\author[0000-0001-5961-1680]{Alexander Smith Clark}
\affiliation{Department of Physics and Astronomy,
Stony Brook University,
Stony Brook, NY 11794-3800, USA}

\author[0000-0003-3603-6868]{Eric T. Johnson}
\affiliation{Department of Physics and Astronomy,
Stony Brook University,
Stony Brook, NY 11794-3800, USA}

\correspondingauthor{Michael Zingale}
\email{michael.zingale@stonybrook.edu}


%======================================================================
% Abstract and Keywords
%======================================================================
\begin{abstract}
We study the coupling of hydrodynamics and reactions in simulations of
the double detonation model for Type Ia supernovae.  When assessing the
convergence of simulations, the focus is usually on spatial resolution, 
however, the method of coupling the physics together as well as the
tolerances used in integrating a reaction network also play an important role.
In this paper, we explore how the choices made in both coupling and integrating the reaction
portion of a simulation influences the accuracy, efficiency, and the nucleosynthesis
of simulations of the double detonation model of Type Ia supernovae.
We find that the stronger coupling between hydrodynamics and reactions,
afforded by the simplified spectral deferred corrections method introduced
previousl,y provides a more accurate and efficient result than standard
operator splitting.  We also find no need to limit reactions rates or reduce the
simulation timestep to the reaction timescale.  The entire simulation
methodology used here is freely available as part of the \castro\ simulation
code.
\end{abstract}

\keywords{convection---hydrodynamics---methods: numerical}

%======================================================================
% Introduction
%======================================================================
\section{Introduction}\label{Sec:Introduction}

There are many different progenitor models for Type Ia supernova (SN Ia) that
are currently being explored to explain the observed diversity of SN Ia explosions \cite{taubenberger:2017}.  The double detonation model
\citep{fink:2007} involves a sub-Chandrasekhar mass white dwarf with a
thin accreted He layer.  A detonation in the He layer propagates
around the star while sending a compression wave converging toward the
center.  This compression wave ignites a carbon detonation that then
propagates outward from the center of the white dwarf.  Many groups
have modeled this scenario, varying the initial conditions,
resolution \citep{rivas:2022}, and reaction networks to understand the robustness of this
model as well as its connection to observations.

\citep{townsley:2019} \citep{gronow:2020}, \citep{gronow:2021}, \citep{glasner:2018}, \citep{roy:2022}, \citep{shenbildsten:2014}, \cite{polin:2019}, \citep{kushnirkatz:2020}


These events are modeled using the Euler equations with reactive and
gravitational sources:
\begin{eqnarray}
\ddt{\rho} + \nabla \cdot (\rho \Ub) &=& 0 \\
\ddt{(\rho \Ub)} + \nabla \cdot (\rho \Ub \Ub) + \nabla p &=& \rho \gb \\
\ddt{(\rho E)} + \nabla \cdot (\rho \Ub E + \Ub p) &=& \rho \Ub \cdot \gb + \rho \dot{S} \\
\ddt{(\rho X_k)} + \nabla \cdot (\rho \Ub X_k) &=& \rho \omegadot_k
\end{eqnarray}
where $\rho$ is the mass density, $\Ub$ is the velocity, $E$ is the specific
total energy, $p$ is the pressure, and $X_k$ are the mass fractions
of the nuclei that react.  The sources are the gravitational acceleration,
$\gb$, which is obtained by solving the the Poisson equation
\begin{equation}
\nabla^2 \Phi = 4 \pi G \rho
\end{equation}
(where $G$ is Newton's constant) for the gravitational potential, $\Phi$, and defining $\gb = -\nabla \Phi$.
The reaction network provides the species creation rate, $\omegadot$ and the energy generation rate, $\dot{S}$.
Finally, the system is closed via the equation of state:
\begin{equation}
p = p(\rho, e, X_k)
\end{equation}
where $e$ is the specific internal energy, obtained as $e = E - |\Ub|^2/2$.

Simulation codes use different methods to advance this system of equations.  For hydrodynamics,
explicit-in-time integration is usually used, with the timestep, $\Delta t$, restricted by the Courant limit.  In two-dimensions, this is:
\begin{equation}
\Delta t = C \min_{i,j} \left \{ \frac{\Delta x}{|u_{i,j}| + c_{i,j}}, \frac{\Delta y}{|v_{i,j}| + c_{i,j}}  \right \}
\end{equation}
where $\Delta x$ and $\Delta y$ are the grid width in the $x$- and $y$-directions, $u$ and $v$
are the $x$- and $y$-components of the velocity and $c$ is the sound
speed, and the minimum is taken over all zones on the grid.  Here, the
Courant number, $C$ is required to be $C < 1$ for stability.  We note that
this is the form of the stability criterion for the corner transport upwind
formulation \citep{ppmunsplit} of hydrodynamics we use.  Dimensionally-split
or method-of-lines formulations will have a slighly more restrictive criteria.
Reactions, on the other hand, tend to be stiff, and are usually evolved
using implicit-in-time integration, with many small steps taken to
make up the hydrodynamics timestep $\Delta t$.

A key concern in modeling astrophysical reacting flows is the coupling
of hydrodynamics and reactions.  If a large amount of energy is dumped
into a zone, then a large flow will result to carry away this energy,
requiring tight communication between the reaction and hydrodynamics
solvers.  This coupling takes various forms in astrophysical
simulation codes.  The most common method is via operator splitting:
the advection and reactive terms are treated independently, with each
process working on the result of the other.  Often, Strang-splitting
\citep{strang:1968} is used, which alternates advection and reaction
to yield second-order accuracy in time.  The primary appeal of operator
splitting is that it is easy to implement---the hydrodynamics and reaction
solvers are largely independent of one another.

When using operator-splitting, the density remains constant, since there
is no reaction source in the mass continuity equation.  The mass
fractions, $X_k$, and energy evolve according to:
\begin{eqnarray}
\frac{dX_k}{dt} &= \omegadot_k(\rho, T, X_k) \label{eq:stranga} \\
\frac{de}{dt} &= \dot{S}(\rho, T, X_k) \\
T &= T(\rho, e, X_k) \label{eq:strangc}
\end{eqnarray}
In the simplest approximation of operator splitting, the
evolution of the mass fractions is solved alone, without integrating
the temperature or energy.  This is the form that was used, e.g., in FLASH
\citep{flash}.  A more accurate operator splitting also includes the
energy evolution (perhaps in terms of temperature), solving the system Eqs.~\ref{eq:stranga}--\ref{eq:strangc}.  This will require
calling the equation of state each time we evaluate the righthand side
of the ODE system, potentially increasing the computational expense.  However, as we showed in
\citet{strang_rnaas}, this is needed to get second-order convergence.

Detonations can be difficult to model numerically.  One common
approach is to use a simultion timestep based on the reaction
timescale, as discussed in \citet{prometheus}, instead of the
hydrodynamics timescale.  For the double detonation scenario, the
ignition of the second detonation can be very sensitive to how the
simulation is run \MarginPar{refs}.  Artifically limiting the
rates \MarginPar{ref}

In \citet{castro_simple_sdc}, we introduced a method based on the
ideas of spectral deferred corrections for coupling reactions and
hydrodyamics (that we called ``simplified-SDC'').  Here, the overall
time-integration is done iteratively, with the hydrodynamics seeing an
explicit reactive source, and the reaction update evolving an ODE
system that includes a piecewise-constant-in-time advective source,
$\Adv{\Uc}^{n+1/2}$,
\begin{equation}
\label{eq:simplesdc}
\ddt{\Uc} = \Rb(\Uc) + \Adv{\Uc}^{n+1/2}
\end{equation}
Here, $\Uc$ is the conserved state, $\Uc = (\rho, \rho \Ub, \rho E,
\rho X_k)^\intercal$, $\Rb(\Uc)$ are the reactive sources, and
$\Adv{\Uc}^{n+1/2}$ is an approximation to the advective update over
the timestep, as computed by the corner transport upwind formulation
of the piecewise parabolic method
\citep{ppmunsplit,millercolella:2002,ppm}. Crucially, this advective
term includes an explicit reaction source term and the update,
Eq.~\ref{eq:simplesdc} means that the reactions will ``see'' what
advection is doing as the reaction network is integrated.

In addition to the time-integration strategy, integrating the reaction
ODE system requires specifying tolerances, usually both an absolute, $\epsilon_\mathrm{abs}$, and relative tolerance, $\epsilon_\mathrm{rel}$, that are combined together into a weight of the form:
\begin{equation}
w_i = \epsilon_{\mathrm{abs},i} + \epsilon_{\mathrm{rel},i} |y_i|
\end{equation}
where the $y_i$ is one of the variables being integrated by the ODE
integrator and $i$ is the index, indicating that different tolerances
can be used for each variable.  Traditionally one set of tolerances is
used for the mass fractions and another for the energy (if it is
integrated).  Unfortunately, the tolerances used by simulations are
not normally reported in papers, and in many cases, may not be easily
controlled by a code user at runtime (and therefore, users may not be
aware of what they are using).  Some examples can be found in publicly
available networks.  For example, the freely available version of the
{\tt aprox13} network uses $\epsilon_\mathrm{rel} = 10^{-5}$ and a
floor on the mass fractions corresponding to $\epsilon_\mathrm{abs} =
10^{-6}$.  Since the simplified-SDC update works in terms of the
conserved variables, $\Uc$, we apply a density weighting to the absolute tolerances when comparing to the values used in Strang integration.

We considered an extreme version of the double detonation scenario in
\citet{castro_simple_sdc}---a very large perturbation was applied to
drive a detonation directly into the underlying carbon-oxygen white
dwarf.  We showed that the Strang split integration method had
difficulty with the integration unless we used tighter tolerances,
while the simplified-SDC method was well-behaved.  Here we look at a
more realistic setup for a double detonation, with a small
perturbation that drives a He detonation around the star and a
compression wave that results in the ignition of a C detonation near
the center of the star.  We focus on the nucleosynthesis and the
efficiency of the method.

%======================================================================
% Results
%======================================================================
\section{Simulations and Results}\label{Sec:results}

\subsection{Simulation setup}

We use the freely-available \castro\ simulation code for all the
simulations shown here.  Our domain uses a 2D axisymmetric geomety
with a size of $5.12\times 10^9~\mathrm{cm}$ by $1.024\times
10^{10}~\mathrm{cm}$.  This is much larger than the initial size of
the star, giving it plenty of room to expand (we note that the figures
shown in the results section are all zoomed in on the star).  The
coarse grid is $640\times 1280$ zones and we use 2 levels of
refinement, each a jump of $2\times$, giving a maximum resolution of
$20~\mathrm{km}$.  \castro\ uses the \amrex\ library
\citep{amrex_joss} to manage the grid hierarchy.  The refinement
strategy is picked to refine on the star and any regions where the
temperature is greater than $10^8~\mathrm{K}$.  Finally, \castro\ uses
subcycling, so finer grids are advanced at a smaller timestep than the
coarse grids.

As noted above, we use the unsplit PPM algorithm for advection, with
the Riemann solver from \citet{colellaglaz:1985} and the general
stellar equation of state of \citet{timmes_swesty:2000}.  In
all simulations, the CFL constraint is the
only timestep constraint used.  Self-gravity is done using a full
Poisson solve using geometric multigrid, with Dirichlet boundary
conditions on the domain boundary computed via a multipole expansion
with a maximum order of 6.  For the low density regions outside of the
star, we use a sponge term on the momentum equation to prevent the
very low density material that is not in hydrostatic equilibrium from
raining down on the star.


All simulations use the same reaction network: the {\tt subch\_simple}
as described in \cite{zhi2023}.  This includes 22 nuclei and 94 rates
from ReacLib \citep{reaclib} and is produced with
\pynucastro\ \citep{pynucastro2}.  \MarginPar{show I show the
  network?} Importantly, we include
$\isotm{N}{14}(\alpha,\gamma)\isotm{F}{18}(\alpha,p)\isotm{Ne}{21}$
which creates protons to allow for
$\isotm{C}{12}(p,\gamma)\isotm{N}{13}(\alpha,p)\isotm{O}{16}$.  This
sequence can be faster than
$\isotm{C}{12}(\alpha,\gamma)\isotm{O}{16}$, as pointed out by
\citet{shenbildsten} and is important for getting the detonation speed
correctly.  To reduce the size of the network, we have
\pynucastro\ combine some $(\alpha,p)(p,\gamma)$ and $(\alpha,\gamma)$
rates into an effective $(\alpha,\gamma)$ rate.  Screening is provided
following the procedure in \citet{wallace:1982}, combining the
screening functions of \citet{graboske:1973,alastuey:1978,itoh:1979}.

The reaction system is integrated using a version of the VODE ODE
integrator \citep{vode} ported to C++.  Our modifications to this
integrator are described in \citet{castro_simple_sdc}.  We use an
analytic approximation to the Jacobian (the main approximation is that
the species derivatives of the screening function are not included)
and found that we get best results when we disable the caching of the
Jacobian in the integrator.

If during the advance of a timestep an
error is generated (negative density, ODE integration fails, mass
fractions don't sum to 1, CFL constraint violated at the new time),
the step is thrown-out and retied with a smaller timestep.  This is
done on a level-by-level basis in the overall AMR subcycling
hierarchy.

We will run a suite of simulations all using the same initial model,
constructed following the methodology in \citet{subchandra}.  We use
an isothermal core of C/O of $1.1~M_\odot$ with $T = 10^7~\mathrm{K}$
and a thin transition region where the temperature ramps up to
$1.75\times 10^8~\mathrm{K}$ and the composition changes to 99\%
\isot{He}{4} and 1\% \isot{N}{14}.  This He envelope is then
integrated isentropically.  The density of the transition from the
underlying CO white dwarf to the He envelope was selected to yield an
envelope mass of $0.05~M_\odot$.  The code for generating the model is
freely available at
\url{https://github.com/amrex-astro/initial_models}.  \MarginPar{should
  we do a zenodo release of this?}

We place a small temperature perturbation in the He layer using the
same form as in \citet{castro_simple_sdc}:
\begin{equation}
  T = T_0 \left \{ 1 + X(\isotm{He}{4}) f \left [1 + \tanh(2 - \sigma) \right ] \right \}
\end{equation}
where
\begin{equation}
  \sigma = \left [ x^2 + (y - R_0)^2 \right ]^{1/2} / \lambda
\end{equation}
and
\begin{equation}
  R_0 = R_\mathrm{pert} + R_\mathrm{base}
\end{equation}
where, $R_\mathrm{base}$ is the radius at which the helium layer
begins and $R_\mathrm{pert}$ is the distance above the base to put the
perturbation.  We choose $R_\mathrm{pert} = 100~\mathrm{km}$.  The
temperature is perturbed above the initial model value, denoted as
$T_0$ here.  The amplitude of the perturbation is $f = 3$ and the
scale of the perturbation is $\lambda = 12.5~\mathrm{km}$.  This is a
very small perturbation, with a peak temperature of about $1.1\times
10^9~\mathrm{K}$, but enough to seed the initial He detonation in the
envelope.

Our base simulation uses the simplified-SDC integration, a CFL number
of $0.2$, and sets $\epsilon_\mathrm{rel} = \epsilon_\mathrm{abs} =
10^{-5}$ for both the species and energy.  Overall, the simulation
requires 21,784 coarse grid timesteps to evolve to $1.0~\mathrm{s}$ of
simulation time.

The simulations are all run on the OLCF Frontier machine, using 4
nodes / 32 AMD GPUs.  The data is moved to the GPUs at the start of
the simulation and all computation is done there.  Our GPU offloading
strategy \citep{castro_gpu} takes advantage of the \amrex\ C++
lambda-capturing functionality to be performance portable.
All of the simulation code is on github at
\url{https://github.com/amrex-astro/} and the inputs files, global
diagnostics used to make the line plots shown below, and metadata
describing the git hashes, compiler environment, and runtime
parameters for all of the simulaitons is available at \MarginPar{add
  Zenodo ref}.

\begin{figure*}[t]
\centering
\plotone{subch_Temp_sequence}
\caption{\label{fig:temp_sequence} Time-sequence of the SDC run showing the temperature.}
\end{figure*}

Figure~\ref{fig:temp_sequence} shows the evolution of the temperature field for the base simulation.  The initial perturbation is placed at the north pole at $t=0$ and leads to a detonation in the He layer that begins to wrap around the star.  By $0.1~\mathrm{s}$ the detonation is almost halfway around the star and by $0.2~\mathrm{s}$ is almost at the south pole.  Figure~\ref{fig:lapl_rho_sequence} shows shows a Schlieren-style plot, which
highlights density gradients.  It is constructed by plotting
$\log_{10}(|\rho^{-1}\nabla^2\rho|)$.  This clearly shows a
compression wave launched by the early evolution of the He detonation propagating inward toward
the center of the white dwarf.  

\begin{figure*}[t]
\centering
\plotone{subch_lap_rho_sequence}
\caption{\label{fig:lap_rho_sequence} Time-sequence of the SDC run showing the compression.}
\end{figure*}

Between $t = 0.3$ and $t = 0.7~\mathrm{s}$ the burning is confined to
the outer layer, but the compression wave continues to advance inward,
converging slightly off-center at around $t = 0.7~\mathrm{s}$.  This
highlights how the speed of the He detonation is important in
determining how off-center the C detonation ignition will be.  Our
choice of network was motivated by ensuring that we capture the energy
release from He burning accurately, in particular the inclusion of the
$\isotm{C}{12}(p,\gamma)\isotm{N}{13}(\alpha,p)\isotm{O}{16}$
sequence.

In the bottom row of Figure~\ref{fig:temp_sequence}, we see
the second detonation ignite near the center of the star and
begin to propagate outward.  By $1.0~\mathrm{s}$ it has evolved
past the initial radius of the star and is into the extended
ash layer left behind by the previous He detonation.

\subsection{Time-integration and Nucleosynthesis}

We now look at how Strang integration to this base simulation.

We run 3 different types of integration methods: traditional operator
splitting, using Strang splitting, both with integration of the energy
equation and without any temperature evolution during the
burn.  \MarginPar{comment about the EOS}.  These methods are compared
to the simplified-SDC integtation.  We also vary the CFL number and
the integration tolerances.

\begin{figure}[t]
\centering
\plotone{subch_ni56}
\caption{\label{fig:ni56} \isot{Ni}{56} mass vs.\ time for the different simulations.}
\end{figure}

Figure~\ref{fig:ni56} shows the total amount of \isot{Ni}{56} as a function of time for
a suite of simulations.  

To understand the difference in the \isot{Ni}{56} mass between the SDC and Strang simulations at 
CFL = 0.2, 

\begin{figure}[t]
\centering
\plotone{sdc_strang_compare_abar_summary_plot}
\caption{\label{fig:abar_sdc_strang} Comparison of the mean molecular weight between SDC and Strang (both run with CFL = 0.2) at $t = 0.2~\mathrm{s}$.}
\end{figure}



\begin{figure}[t]
\centering
\plotone{subch_cpu}
\caption{\label{fig:cpu} node hours of GPU time for the different simulations.}
\end{figure}




Figure~\ref{fig:cpu} shows the computational expense of the
simulations, in terms of node hours.  All simulations were run on 4
nodes (32 AMD GPUs total).  We note that we have not spent any time on
load-balancing these simulations, so we should consider this to be
only a guide.  At the start of the simulation, there are 56 boxes on
the finest grid (representing 0.77\% of the domain) and at 1 s of
evolution there is are 60 boxes covering 5.64\% of the domain.  The
figure shows that the Strang simulation with a CFL of 0.2 is the
cheapest, but as we saw above, this does not get the nucleosynthesis
correct.  All of the other Strang simulations are more costly than the
baseline SDC simulation.  This shows that not only does the
simplified-SDC simulation accurately evolve the nucleosynthesis, it
does so in a very efficient fashion.

\subsection{Spatial resolution}

Figure~\ref{fig:res_panel} shows a time sequence of the simplified-SDC
simulation with CFL = 0.2 and our standard tolerances
($\epsilon_\mathrm{abs} = \epsilon_\mathrm{rel} = 10^{-5}$ at 3
diferent spatial resolutions: 40 km, 20 km, and 10 km.  For the 40 km
run, we simply use one fewer levels of refinement compared to our base
simulation.  For the 10 km run, we add an additional level of
refinement that captures only the high temperature regions (including
the initial He layer) but with the very center always refined by
adding an additional refinement tagging criteria on densities $\rho >
7.5\times 10^7~\gcc$.  The first column of the figure shows the
initial grid structure---darker colors are higher levels of
refinement.  Each of the boxes shown is a multiple of 32 zones in each
dimension, with a maximum of 256 zones in any dimension.  This
refinement for the highest resolution run ensures that we capture the
convergence of the pressure wave at the center.  For the 10 km resolution
run, we needed to switch from an analytic approximation to the Jacobian
to a numerical difference approximation once the C detonation ignited, because
the evolution became too challenging for the ODE integrator otherwise.

\begin{figure}[t]
\centering
\plotone{subch_Temp_res_compare}
\caption{\label{fig:res_panel} Temperature evolution for SDC simulations with CFL=0.2 at 40, 20, and 10 km.  For the earliest time
  in each resolution, the AMR grid structure is shown.}
\end{figure}


As we see from the time-evolution, the three different resolutions all
behave largely the same.  The timescale for the He detonaiton to
propagate around the star and the timescale at which the second C
detonation ignite match well.  Figure~\ref{fig:res} shows the
\isot{Ni}{56} mass produced from the three different simulations.
Again, the agreement is quite strong.

\begin{figure}[t]
\centering
\plotone{subch_ni56_res}
\caption{\label{fig:res} \isot{Ni}{56} mass vs.\ time for 3 different spatial resolutions, using the simplified-SDC time integration method.}
\end{figure}


We note that these resolution results differ from the conclusions of
\citet{rivas:2022}.  There are a number of reasons why this might be
the case.  First the initial models differ \MarginPar{details}.
Second, we use a much larger reaction network here that better
captures the energetics.  This means that the He detonation moves
faster and the subsequent convergence of the pressure wave will be
more on-center, meaning that the ignition of the C detonation occurs
at a higher temperature.  Finally, we are using the simplified-SDC time-integration, which that work used Strang splitting \MarginPar{do they give details about e evolution?}

\section{Summary}

Our results suggest that exploring the time-integration method and
integrator tolerances is an important part of demonstrating
convergence of simulations involving explosive reacting flows in
astrophysics.

The ignition of the carbon detonation remains numerically challenging
for the ODE integrator, especially with large
timesteps.  \MarginPar{add some info on why it fails} In the near
future we will show how to include various nuclear statistical
equilibrium approximations into the simplified-SDC formalism while
retaining second-order accuracy.  We will also explore different
integrators and auto-differentiation for the Jacobian (in particular,
the compositional contribution from the screening)

We encourage authors to include a discussion about the reaction
integration algorithm and tolerances in papers to permit comparisons
in the future.

\begin{acknowledgements}
\castro\ is open-source and freely available at
\url{http://github.com/AMReX-Astro/Castro}.  The problem setup used
here is available in the git repo as {\tt subchandra}.  

The work at Stony Brook was supported by DOE/Office of Nuclear
Physics grant DE-FG02-87ER40317.  This research used resources of the
National Energy Research Scientific Computing Center, a DOE Office of
Science User Facility supported by the Office of Science of the
U.~S.\ Department of Energy under Contract No.\ DE-AC02-05CH11231.
This research was supported by the Exascale Computing Project
(17-SC-20-SC), a collaborative effort of the U.S. Department of Energy
Office of Science and the National Nuclear Security Administration.
This research used resources of the Oak Ridge Leadership Computing
Facility at the Oak Ridge National Laboratory, which is supported by
the Office of Science of the U.S. Department of Energy under Contract
No. DE-AC05-00OR22725, awarded through the DOE INCITE program.  We
thank NVIDIA Corporation for the donation of a Titan X and Titan V GPU
through their academic grant program.  This research has made use of
NASA's Astrophysics Data System Bibliographic Services.
\end{acknowledgements}

\facilities{NERSC, OLCF}

\software{\amrex~\citep{amrex_joss},
          \castro~\citep{castro,castro_joss},
          GCC (\url{https://gcc.gnu.org/}),
          helmeos \citep{timmes_swesty:2000},
          linux (\url{https://www.kernel.org/}),
          matplotlib (\citealt{Hunter:2007}, \url{http://matplotlib.org/}),
          NetworkX \citep{networkx},
          NumPy \citep{numpy,numpy2},
          pynucastro \citep{pynucastro,pynucastro2},
          python (\url{https://www.python.org/}),
          SymPy \citep{sympy},
          valgrind \citep{valgrind},
          VODE \citep{vode},
          yt \citep{yt}}



%======================================================================
% References
%======================================================================

\bibliographystyle{aasjournal}
\bibliography{ws}


\end{document}
