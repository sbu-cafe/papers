\documentclass[twocolumn,times,tighten]{aastex631}

\usepackage{epsf,color,amsmath}

\usepackage{cancel}

\newcommand{\sfrac}[2]{\mathchoice%
  {\kern0em\raise.5ex\hbox{\the\scriptfont0 #1}\kern-.15em/
    \kern-.15em\lower.25ex\hbox{\the\scriptfont0 #2}}
  {\kern0em\raise.5ex\hbox{\the\scriptfont0 #1}\kern-.15em/
    \kern-.15em\lower.25ex\hbox{\the\scriptfont0 #2}}
  {\kern0em\raise.5ex\hbox{\the\scriptscriptfont0 #1}\kern-.2em/
    \kern-.15em\lower.25ex\hbox{\the\scriptscriptfont0 #2}} {#1\!/#2}}

\newcommand{\myhalf}{\sfrac{1}{2}}
\newcommand{\nph}{{n+\myhalf}}
\newcommand{\nmh}{{n-\myhalf}}

\newcommand{\inp}{\mathrm{in}}
\newcommand{\outp}{\mathrm{out}}

% boldsymbol means bold italic
\newcommand{\eb}{{\bf{e}}}
\newcommand{\Ub}{{\bf{U}}}
\newcommand{\xb}{{\bf{x}}}
\newcommand{\kb}{{\bf{k}}}
\newcommand{\Vb}{{\bf{V}}_n}
\newcommand{\Vbhat}{{\bf{\widehat{V}}}_n}
\newcommand{\Omegab}{{\bf{\Omega}}}
\newcommand{\gb}{{\bf{g}}}
\newcommand{\rb}{{\bf{r}}}

\newcommand{\pb}{p_\mathrm{base}}
\newcommand{\epsdot}{\dot{\epsilon}}
\newcommand{\qburn}{q_\mathrm{burn}}
\newcommand{\rt}{\tilde{r}_0}


\newcommand{\nablab}{\mathbf{\nabla}}
\newcommand{\dt}{\Delta\ t}

\newcommand{\omegadot}{\dot{\omega}}

\newcommand{\Hext}{H_{\rm ext}}
\newcommand{\Hnuc}{H_{\rm nuc}}
\newcommand{\kth}{k_{\rm th}}

\newcommand{\Gammaonebar}{\overline{\Gamma}_1}
\newcommand{\Sbar}{\overline{S}}

\newcommand{\etarho}{\eta_\rho}
\newcommand{\etarhoh}{\eta_{\rho~h}}

\newcommand{\Ubt}{\widetilde{\Ub}}
\newcommand{\wt}{\widetilde{w}}

\newcommand{\He}{$^4$He}
\newcommand{\C}{$^{12}$C}
\newcommand{\Fe}{$^{56}$Fe}

\newcommand{\isot}[2]{$^{#2}\mathrm{#1}$}
\newcommand{\isotm}[2]{{}^{#2}\mathrm{#1}}

\newcommand{\maestro}{{\sf MAESTRO}}
\newcommand{\castro}{{\sf Castro}}
\newcommand{\amrex}{{\sf AMReX}}
\newcommand{\pynucastro}{{\sf pynucastro}}

\newcommand{\avg}[1]{\overline{#1}}
\newcommand{\avgtwod}[1]{\langle~#1 \rangle}
\newcommand{\rms}[2]{\left(\delta#1\right)_{r_{#2}}}
\newcommand{\mymax}[1]{\left(#1\right)_{\rm max}}

\newcommand{\Tb}{\ensuremath{T_\mathrm{base}}}
\newcommand{\gcc}{\mathrm{g~cm^{-3} }}
\newcommand{\cms}{\mathrm{cm~s^{-1} }}

\newcommand{\half}{\frac{1}{2}}

\setlength{\marginparwidth}{0.5in}

\newcommand{\MarginPar}[1]{
    \marginpar{\vskip-\baselineskip%
               \raggedright%
               \tiny\sffamily%
               {\color{red}\hrule%
               \smallskip%
               #1\par%
               \smallskip%
               \hrule}}%
}

\newcommand{\AssignTo}[1]{
    \marginpar{\vskip-\baselineskip%
               \raggedright%
               \tiny\sffamily%
               {\color{blue}\hrule%
               \smallskip%
               #1\par%
               \smallskip%
               \hrule}}%
}

\begin{document}
%======================================================================
% Title
%======================================================================
\title{Comparing Early Evolution of Flames in X-ray Bursts in Two and Three Dimensions}

\shorttitle{Early XRB Flame Evolution}
%% \shortauthors{Harpole et al.}

\author[0000-0001-8401-030X]{Michael Zingale}
\affiliation{Dept.\ of Physics and Astronomy, Stony Brook University,
             Stony Brook, NY 11794-3800}

%% \author[0000-0002-1530-781X]{Alice Harpole}
%% \affiliation{Dept.\ of Physics and Astronomy, Stony Brook University,
%%              Stony Brook, NY 11794-3800}

\author[0000-0001-6191-4285]{Kiran Eiden}
\affiliation{Dept.\ of Physics and Astronomy, Stony Brook University,
             Stony Brook, NY 11794-3800}


%% \author[0000-0003-2300-5165]{Donald Willcox}
%% \affiliation{Lawrence Berkeley National Laboratory, Berkeley, CA}

\author[0000-0003-0439-4556]{Max P.\ Katz}
\affiliation{NVIDIA Corp}


\correspondingauthor{Michael Zingale}
\email{michael.zingale@stonybrook.edu}


%======================================================================
% Abstract and Keywords
%======================================================================
\begin{abstract}
We explore the early evolution of a hot spot and a flame ignition and
spreading on the surface of a neutron star in three-dimensions.  We
look at the nucleosynthesis and morphology of the burning front and
compare to two-dimensional axisymmetric simulations.  Finally, we discuss
the progress toward full-star resolved flame simulations.\end{abstract}

\keywords{convection---hydrodynamics---methods: numerical---stars: neutron---X-rays: bursts}

%======================================================================
% Introduction
%======================================================================
\section{Introduction}\label{Sec:Introduction}

X-ray bursts (XRBs) result from thermonuclear burning of an accreted
H/He or He layer on a neutron star \citep{galloway:2017}.
Observations of brightness oscillations in the rise of the lightcurve
suggest that the burning begins localized and then spreads across the
neutron star \citep{bhattacharyya:2007}.  This spreading is inherently a multi-dimensional
phenomena, and hydrodynamic simulations that resolve the reactive
zone, use realistic reaction networks, and capture the scales over
which rotation is important are needed to understand the flame
propagation and nucleosynthesis (especially rp-process in mixed H/He bursts \citealt{rpprocess}).  We show a first attempt at modeling
the early evolution of a spreading hotspot in three dimensions.  This
builds on our earlier two dimensional work
\citep{eiden:2020,harpole:2021} that developed our simulation
framework and explored how the acceleration of the burning front
depends on the initial model structure.  As with those simulations, we
use the freely-available \castro\ simulation code
\citep{castro,castro_joss} simulation code, with all of the code
needed to run these simulations in our public github repos.

Our simulations complement the multidimensional models of flames of
\citet{cavecchi:2013,art-2015-cavecchi-etal,art-2016-cavecchi-etal,Cavecchi2019}
by focusing on resolving the reaction zone of the flame and modeling
the vertical structure hydrodynamically instead of hydrostatically.
We are also modeling flames starting from a small hot spot.  This
however means that we are confined to smaller domains, with similar
computational resources, so one of the goals of this study is to
understand how far we can push resolved XRB flame simulations.  The
work by \citet{goodwin:2021} also explores multidimensional evolution,
looking at the thermal transport and how affects the location where
the hotspot ignites.  Together, all three approaches help build a
comprehensive multi-dimensioanl picture of X-ray bursts.

The goal of this paper is to understand the challenges of a fully-resolved
three-dimensional simulation and explore what is possible.  This will inform
our follow-on simulations.

%======================================================================
% Results
%======================================================================
\section{Simulations and Results}\label{Sec:results}

In \citet{harpole:2021}, we explored several different rotation rates
and initial model thermal structures.  We explored several different
temperature structures in our initial model.  We choose a model with a
neutron star crust temperature of $2\times 10^8~\mathrm{K}$.  We adopt
this model for the 3D simulation presented here, since it gives a clean
well-defined flame.  The model is constructed following the procedure
described in \citet{eiden:2020,harpole:2021}.

Moving to 3D is very expensive, so to further reduce the computational
expense, we use an aniostropic resolution, with 32 cm laterally and 16
cm resolution vertically.  This is finer vertically than done in
\citet{harpole:2021}.  We pick a rotation rate of 1000 Hz.  Because
the hotspot is placed in the center of the domain, the size of the
domain also becomes a constraint, and as a result, the evolution time
we can reach will become limited by the time it takes the burning to
approach the edge of the domain.  As a result, we focus here on the
early evolution.

Our 3D simulation used a base grid of $768^2 \times 192$ zones with 2
levels of refinement, the first jumping by a factor of 4 and the next
by a factor of 2.  The domain has a size $(1.96608\times
10^5~\mathrm{cm})^2 \times 2.4576\times 10^4~\mathrm{cm}$, giving a
fine grid resolution of $32~\mathrm{cm}$ laterally and
$16~\mathrm{cm}$ vertically.  The vertical resolution is finer to help
maintain hydrostatic equilibrium.  The simulation methodology and
parameters are mostly identical to those used in \citet{harpole:2021}.
We use an unsplit piecewise parabolic method \cite{ppmunsplit} for the
advection and operator (Strang) splitting for the reactions, evolving
an internal energy evolution equation during the burn, as described in
\citet{strang_rnaas}.  The main difference with the previous
simulations is that we switched from a hydrostatic to reflecting
boundary at the bottom of the domain and used a simple well-balanced
scheme (similar to \citealt{kappeli:2016} but adapted to deal with
characteristic tracing in PPM as described in \citealt{ppm-hse}).
Tests demonstrated that this boundary does a better job than our
previous approach at supporting the atmosphere as the flame propagates
for long times.

We used static mesh refinement, to achieve better
load balancing, fully refining the atmosphere below a height of
$3600~\mathrm{cm}$.  These simulations used the 7-isotope He-burning
network used in \citet{eiden:2020}.  The simulations were run on the
OLCF Summit supercomputer, on 342 to 1366 nodes, with 6 NVIDIA V100
GPUs per node.  The entire computation was offloaded to GPUs following
the strategy described in \citet{castro_gpu}.  Overall, about 250,000
node-hours were used for the calculation.  The full state of the
calculation was output only every 0.005 s, as each output file is 2.4
TB in size (for single precision data).  We output a few fields more
frequently for visualization.  We ran a 2D simulation with the same
resolution and refinement strategy to compare with here.

\begin{figure}[t]
\centering
%\epsscale{0.75}
\plotone{abar_top_stack}
%\epsscale{1.0}
\caption{\label{fig:vr_abar} Volume rendering of the three-dimensional XRB
simulation showing the mean molecular weight, $\bar{A}$.  The flame is viewed
from above.}
\end{figure}

Figure~\ref{fig:vr_abar} shows the a three-dimensional simulation via
top-down volume rendering of the mean-molecular weight,
\begin{equation}
\frac{1}{\bar{A}} = \sum_k \frac{X_k}{A_k}
\end{equation}
where $X_k$ is the mass fraction of species $k$ and $A_k$ is the
atomic weight (in mass units).  The structure is shown at 10, 20, and
40~ms of simulation time.  While initially quite symmetric, the axial
symmetry is broken and the flame structure takes on a more tenuous
form.  The panels show the flame spreading considerably through the
domain as time evolves, and the simulation is halted once the flame
nears the boundaries.  

Figure~\ref{fig:2d_abar} shows the corresponding two-dimensional
axisymmetric simulation.  The initial model and simulation parameters
are identical to the three-dimensional case.  This simulation uses
zones that are 32 cm wide and 16 cm tall, analogous to the lateral and
vertical resolution of the three-dimensional simulation.  We see that
the overall structure and evolution looks identical to that reported
in \cite{harpole:2021}.


\begin{figure*}[t]
\centering
\plotone{time_series}
\caption{\label{fig:2d_abar} Time-sequence of the two-dimensional axisymmetric
simualtion using the same setup as the three-dimensional simulation shown
in Figure~\ref{fig:vr_abar}.  The domain is zoomed in, showing only the left
two-thirds of the surface.}
\end{figure*}


%% \begin{figure}[t]
%% \centering
%% \plotone{flame_wave_1000Hz_25cm_smallplt88081_Hnuc_annotated_top} \\
%% \plotone{flame_wave_1000Hz_25cm_smallplt174562_Hnuc_annotated_top.png}
%% \caption{\label{fig:vr_hnuc} stuff}
%% \end{figure}


Because the 3D flame does not stay perfectly circular, it is difficult
to define the flame speed using the same method employed in
\citet{eiden:2020}.  Instead, we will look at the mass of the ash
material to assess how quickly the burning is taking place.
Figure~\ref{fig:mass_plot} shows the mass of the different species as
a function of time, for the 2D and 3D calculation.  The axisymmetric
2D calculation has slightly less volume than the 3D domain, since
rotating about the symmetry axis produces a cylindrical domain
inscribed in the 3D domain.  To compensate for this, we scale the
masses by the initial \isot{He}{4} mass.  This plot shows that the
masses of the heavy species, especially \isot{C}{12}, grow quickly
with time.  The differences between the 2D and 3D simulations appear
small---the burning is slightly faster in 2D, but the trends track
very well.  This slight difference is likely because the assumption of
axisymmetry there does not allow for the complex structure we see in
the evolution of composition in the 3D flame.  This strong agreement
suggests that using 2D axisymmetric calculations is a good model for
the early flame evolution.  And since they are so much less
computationally expensive, we this will allow us to explore much more
complex reaction networks and understand the nucleosynthesis in more
detail.

\begin{figure}[t]
\centering
\plotone{mass_plot}
\caption{\label{fig:mass_plot} Mass of species scaled to initial He
  mass as a function of time.  The 3D simulation is shown as the solid
  lines and the 2D simulation is shown as the dashed lines.}
\end{figure}

\section{Summary}

We have extended our simulation framework to three-dimensions, and
explored the differences between a two-dimensional axisymmetric
simulation and a fully three-dimensional hydrodynamical simulation.
Overall, the early evolution of the flame spreading from a hotspot
behaves similarly in two- and three-dimensions.  As this
three-dimensional simulation pushed the limits of available
computational resources, this result suggests that we can continue to
use two-dimensional axisymmetric simulations to explore the initial
flame structure and spend our computational resources on increasing
the size of our reaction networks to explore the sensitivity to
network size and understand the overall nucleosynthesis.

There is still a role for three-dimensional simulations---the
convective burning in the accreted layer should create turbulence that
the flame will encounter as it propagates.  Although we have shown
that for the flame in a Type Ia supernovae, the turbulence from the
era of convection is not strong enough to disrupt the
flame~\citep{wdturb}, the flame in an XRB is considerably slower and
thicker, so it remains to be seen what effect it might have.  This can
be assessed in a different geometry, like a long channel, which would
allow us to trade domain size for resolution to increase the numerical
Reynolds number.  We also will consider higher-order simulation
methodologies which could allow us to capture these effects at lower
resolution.  We've already demonstrated a fully fourth-order accruate
(in space and time) algorithm for coupling hydro, diffusion, and
reactions in Castro \citep{castro-sdc} that could be used here.  The
main outstanding issue with applying that work to the present problem
is the multilevel time integration.


\begin{acknowledgements}
\castro\ is open-source and freely available at
\url{http://github.com/AMReX-Astro/Castro}.  The problem setup used
here is available in the git repo as {\tt flame\_wave}. The metadata
describing the build environment and the global diagnostic output
files are available on Zenodo at \citet{xrb_data}.  We thank Alice
Harpole and Max Katz for their contributions to the AMReX Astrophysics
suite.  

The work at Stony Brook was supported by DOE/Office of Nuclear
Physics grant DE-FG02-87ER40317.  This research used resources of the
National Energy Research Scientific Computing Center, a DOE Office of
Science User Facility supported by the Office of Science of the
U.~S.\ Department of Energy under Contract No.\ DE-AC02-05CH11231.
This research was supported by the Exascale Computing Project
(17-SC-20-SC), a collaborative effort of the U.S. Department of Energy
Office of Science and the National Nuclear Security Administration.
This research used resources of the Oak Ridge Leadership Computing
Facility at the Oak Ridge National Laboratory, which is supported by
the Office of Science of the U.S. Department of Energy under Contract
No. DE-AC05-00OR22725, awarded through the DOE INCITE program.  We
thank NVIDIA Corporation for the donation of a Titan X and Titan V GPU
through their academic grant program.  This research has made use of
NASA's Astrophysics Data System Bibliographic Services.
\end{acknowledgements}

\facilities{NERSC, OLCF}

\software{AMReX \citep{amrex_joss},
          Castro \citep{castro,castro_joss},
          GCC (\url{https://gcc.gnu.org/}),
          linux (\url{https://www.kernel.org/}),
          matplotlib (\citealt{Hunter:2007}, \url{http://matplotlib.org/}),
          NumPy \citep{numpy,numpy2},
          python (\url{https://www.python.org/}),
          valgrind \citep{valgrind},
          VODE \citep{vode},
          yt \citep{yt}}



%======================================================================
% References
%======================================================================

\bibliographystyle{aasjournal}
\bibliography{ws}


\end{document}
