\documentclass[preprint,times,tighten]{aastex631}

\usepackage{epsf,color,amsmath}

\usepackage{cancel}

\newcommand{\sfrac}[2]{\mathchoice%
  {\kern0em\raise.5ex\hbox{\the\scriptfont0 #1}\kern-.15em/
    \kern-.15em\lower.25ex\hbox{\the\scriptfont0 #2}}
  {\kern0em\raise.5ex\hbox{\the\scriptfont0 #1}\kern-.15em/
    \kern-.15em\lower.25ex\hbox{\the\scriptfont0 #2}}
  {\kern0em\raise.5ex\hbox{\the\scriptscriptfont0 #1}\kern-.2em/
    \kern-.15em\lower.25ex\hbox{\the\scriptscriptfont0 #2}} {#1\!/#2}}

\newcommand{\myhalf}{\sfrac{1}{2}}
\newcommand{\nph}{{n+\myhalf}}
\newcommand{\nmh}{{n-\myhalf}}

\newcommand{\inp}{\mathrm{in}}
\newcommand{\outp}{\mathrm{out}}

% boldsymbol means bold italic
\newcommand{\eb}{{\bf{e}}}
\newcommand{\Ub}{{\bf{U}}}
\newcommand{\xb}{{\bf{x}}}
\newcommand{\kb}{{\bf{k}}}
\newcommand{\Vb}{{\bf{V}}_n}
\newcommand{\Vbhat}{{\bf{\widehat{V}}}_n}
\newcommand{\Omegab}{{\bf{\Omega}}}
\newcommand{\gb}{{\bf{g}}}
\newcommand{\rb}{{\bf{r}}}

\newcommand{\pb}{p_\mathrm{base}}
\newcommand{\epsdot}{\dot{\epsilon}}
\newcommand{\qburn}{q_\mathrm{burn}}
\newcommand{\rt}{\tilde{r}_0}


\newcommand{\nablab}{\mathbf{\nabla}}
\newcommand{\dt}{\Delta\ t}

\newcommand{\omegadot}{\dot{\omega}}

\newcommand{\Hext}{H_{\rm ext}}
\newcommand{\Hnuc}{H_{\rm nuc}}
\newcommand{\kth}{k_{\rm th}}

\newcommand{\Gammaonebar}{\overline{\Gamma}_1}
\newcommand{\Sbar}{\overline{S}}

\newcommand{\etarho}{\eta_\rho}
\newcommand{\etarhoh}{\eta_{\rho~h}}

\newcommand{\Ubt}{\widetilde{\Ub}}
\newcommand{\wt}{\widetilde{w}}

\newcommand{\He}{$^4$He}
\newcommand{\C}{$^{12}$C}
\newcommand{\Fe}{$^{56}$Fe}

\newcommand{\isot}[2]{$^{#2}\mathrm{#1}$}
\newcommand{\isotm}[2]{{}^{#2}\mathrm{#1}}

\newcommand{\maestro}{{\sf MAESTRO}}
\newcommand{\castro}{{\sf Castro}}
\newcommand{\amrex}{{\sf AMReX}}
\newcommand{\pynucastro}{{\sf pynucastro}}

\newcommand{\avg}[1]{\overline{#1}}
\newcommand{\avgtwod}[1]{\langle~#1 \rangle}
\newcommand{\rms}[2]{\left(\delta#1\right)_{r_{#2}}}
\newcommand{\mymax}[1]{\left(#1\right)_{\rm max}}

\newcommand{\Tb}{\ensuremath{T_\mathrm{base}}}
\newcommand{\gcc}{\mathrm{g~cm^{-3} }}
\newcommand{\cms}{\mathrm{cm~s^{-1} }}

\newcommand{\half}{\frac{1}{2}}

\setlength{\marginparwidth}{0.5in}

\newcommand{\MarginPar}[1]{
    \marginpar{\vskip-\baselineskip%
               \raggedright%
               \tiny\sffamily%
               {\color{red}\hrule%
               \smallskip%
               #1\par%
               \smallskip%
               \hrule}}%
}

\newcommand{\AssignTo}[1]{
    \marginpar{\vskip-\baselineskip%
               \raggedright%
               \tiny\sffamily%
               {\color{blue}\hrule%
               \smallskip%
               #1\par%
               \smallskip%
               \hrule}}%
}

\begin{document}
%======================================================================
% Title
%======================================================================
\title{Dynamics of Laterally Propagating Flames in X-ray Bursts. III. Three-Dimensional Flames}

\shorttitle{Lateral Flame Dynamics III}
%% \shortauthors{Harpole et al.}

%% \author[0000-0002-1530-781X]{Alice Harpole}
%% \affiliation{Dept.\ of Physics and Astronomy, Stony Brook University,
%%              Stony Brook, NY 11794-3800}

%% \author[0000-0001-6191-4285]{Kiran Eiden}
%% \affiliation{Dept.\ of Physics and Astronomy, Stony Brook University,
%%              Stony Brook, NY 11794-3800}

%% \author[0000-0001-8401-030X]{Michael Zingale}
%% \affiliation{Dept.\ of Physics and Astronomy, Stony Brook University,
%%              Stony Brook, NY 11794-3800}

%% \author[0000-0003-2300-5165]{Donald Willcox}
%% \affiliation{Lawrence Berkeley National Laboratory, Berkeley, CA}

%% \author{Yuri Cavecchi}
%% \affiliation{Mathematical Sciences and STAG Research Centre, University of Southampton, SO17 1BJ}

%% \author[0000-0003-0439-4556]{Max P.\ Katz}
%% \affiliation{NVIDIA Corp}

%% \author[0000-0001-8092-1974]{Weiqun Zhang}
%% \affiliation{Lawrence Berkeley National Laboratory, Berkeley, CA}


%% \correspondingauthor{Alice Harpole}
%% \email{alice.harpolee@stonybrook.edu}


%======================================================================
% Abstract and Keywords
%======================================================================
\begin{abstract}
We continue to investigate laterally propagating flames in XRBs,
comparing the dynamics between two- and three-dimensions.
\end{abstract}

\keywords{convection---hydrodynamics---methods: numerical---stars: neutron---X-rays: bursts}

%======================================================================
% Introduction
%======================================================================
\section{Introduction}\label{Sec:Introduction}

X-ray bursts result from thermonuclear burning of an accreted H/He or
He layer on a neutron star \citep{galloway:2017}.  Observations of
brightness oscillations in the rise of the lightcurve suggest that the
burning begins localized and then spreads across the neutron star.
One-dimensional simulations of X-ray bursts reproduce the lightcurve
and recurrence time \citep{woosley-xrb} and can use large reaction
networks to explore rp-process nucleosynthesis.

To understand how the burning front propagates across the surface, we
need to do multidimensional simulations.  The work of
\citet{cavecchi:2013,art-2015-cavecchi-etal,art-2016-cavecchi-etal}

The three-dimensional study by \citet{Cavecchi2019} showed...

We have been performing simulations of flame spreading on a neutron
star resolving the flame structure using realistic reaction networks.  Our two-dimensional studies \citep{eiden:2020,harpole:2021} demonstrated our open-source framework for modeling X-ray bursts using 

We use the \castro\ \citep{castro,castro_joss} simulation code, following
the same numerical approach described in \citet{eiden:2020,harpole:2021}.

%======================================================================
% Results
%======================================================================
\section{Simulations and Results}\label{Sec:results}

In \citet{harpole:2021}, we explored several different rotation rates
and initial model thermal structures.  We found that a model with a
neutron star crust temperature of $3\times 10^8~\mathrm{K}$ resulted
in a flame that accelerated quickly as it spread across the neutron
star surface.  We adopt this model for the 3D simulation presented
here, since it requires the least amount of simulation time to yield a
developed flame.  Moving to 3D is very expensive, so to further reduce
the computational expense, we use a slightly lower resolution of 25 cm
instead of 20 cm from \citet{harpole:2021}.  We pick a rotation rate
of 1000 Hz.

Our 3D simulation used a base grid of $1024^2 \times 128$ zones with 2
levels of refinement, the first jumping by a factor of 4 and the next
by a factor of 2.  The domain has a size $(2.048\times
10^5~\mathrm{cm})^2 \times 2.56\times 10^4~\mathrm{cm}$, giving a fine
grid resolution of $25~\mathrm{cm}$.  The initial model and other
simulation parameters are identical to those used in
\citet{harpole:2021}.  We used static mesh refinement, to achieve
better load balancing, fully refining the atmosphere below a height of
$3600~\mathrm{cm}$.  These simulations used the 7-isotope He-burning
network used in \citet{eiden:2020}.  The simulations were run on the
OLCF Summit supercomputer, on 342 to 1366 nodes, with 6 NVIDIA V100
GPUs per node.  The entire computation was offloaded to GPUs following
the strategy described in \citet{castro_gpu}.

Things to show:

Slice plot compared to 2-d simulation

Phase plot comparisons

Volume renderings


\section{Summary}

We have extended our simulation framework to three-dimensions, and
explored the differences between a two-dimensional axisymmetric
simulation and a fully three-dimensional simulation.

Future work will explore mixed H/He bursts, using this same framework.

\begin{acknowledgements}
\castro\ is open-source and freely available at
\url{http://github.com/AMReX-Astro/Castro}.  The problem setup used
here is available in the git repo as {\tt flame\_wave}.  The work at
Stony Brook was supported by DOE/Office of Nuclear Physics grant
DE-FG02-87ER40317.  This research used resources of the National
Energy Research Scientific Computing Center, a DOE Office of Science
User Facility supported by the Office of Science of the
U.~S.\ Department of Energy under Contract No.\ DE-AC02-05CH11231.
This research used resources of the Oak Ridge Leadership Computing
Facility at the Oak Ridge National Laboratory, which is supported by
the Office of Science of the U.S. Department of Energy under Contract
No. DE-AC05-00OR22725, awarded through the DOE INCITE program.  We
thank NVIDIA Corporation for the donation of a Titan X and Titan V GPU
through their academic grant program.  This research has made use of
NASA's Astrophysics Data System Bibliographic Services.
\end{acknowledgements}

\facilities{NERSC, OLCF}

\software{AMReX \citep{amrex_joss},
          Castro \citep{castro,castro_joss},
          GCC (\url{https://gcc.gnu.org/}),
          linux (\url{https://www.kernel.org/}),
          matplotlib (\citealt{Hunter:2007}, \url{http://matplotlib.org/}),
          NumPy \citep{numpy,numpy2},
          python (\url{https://www.python.org/}),
          valgrind \citep{valgrind},
          VODE \citep{vode},
          yt \citep{yt}}



%======================================================================
% References
%======================================================================

\bibliographystyle{aasjournal}
\bibliography{ws}


\end{document}
