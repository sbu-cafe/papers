\documentclass[times,modern]{aastex63}

% these lines seem necessary for pdflatex to get the paper size right
\pdfpagewidth 8.5in
\pdfpageheight 11.0in

\usepackage[T1]{fontenc}
\usepackage{epsf,color,amsmath}

\usepackage{cancel}

\newcommand{\sfrac}[2]{\mathchoice%
  {\kern0em\raise.5ex\hbox{\the\scriptfont0 #1}\kern-.15em/
    \kern-.15em\lower.25ex\hbox{\the\scriptfont0 #2}}
  {\kern0em\raise.5ex\hbox{\the\scriptfont0 #1}\kern-.15em/
    \kern-.15em\lower.25ex\hbox{\the\scriptfont0 #2}}
  {\kern0em\raise.5ex\hbox{\the\scriptscriptfont0 #1}\kern-.2em/
    \kern-.15em\lower.25ex\hbox{\the\scriptscriptfont0 #2}} {#1\!/#2}}


\newcommand{\castro}{{\sf Castro}}
\newcommand{\maestro}{{\sf Maestro}}
\newcommand{\flash}{{\sf Flash}}
\newcommand{\amrex}{{\sf AMReX}}

\newcommand{\isot}[2]{$^{#2}\mathrm{#1}$}
\newcommand{\isotm}[2]{{}^{#2}\mathrm{#1}}

\newcommand{\gcc}{\mathrm{g~cm^{-3} }}
\newcommand{\cms}{\mathrm{cm~s^{-1} }}

\newcommand{\nablab}{{\mathbf{\nabla}}}
\newcommand{\Ub}{\mathbf{U}}
\newcommand{\gb}{\mathbf{g}}
\newcommand{\omegadot}{\dot{\omega}}
\newcommand{\Sdot}{\dot{S}}
\newcommand{\ddx}[1]{{\frac{{\partial#1}}{\partial x}}}
\newcommand{\ddxs}[1]{{\frac{{\partial}}{\partial x}}#1}
\newcommand{\ddt}[1]{{\frac{{\partial#1}}{\partial t}}}
\newcommand{\odt}[1]{{\frac{{d#1}}{dt}}}
\newcommand{\divg}[1]{{\nablab \cdot \left (#1\right)}}
\newcommand{\dedr}{\left . {\partial{}e}/{\partial\rho}\right |_{T, X_k}}
\newcommand{\dedrd}{\left . \frac{\partial{}e}{\partial\rho}\right |_{T, X_k}}
\newcommand{\dedX}{\left . {\partial{}e}/{\partial{}X_k} \right |_{\rho, T}}
\newcommand{\dedXd}{\left . \frac{\partial{}e}{\partial{}X_k} \right |_{\rho, T, X_{j,j\ne k}}}
\newcommand{\dedT}{\left . {\partial{}e}/{\partial{}T} \right |_{\rho,X_k}}
\newcommand{\dedTd}{\left . \frac{\partial{}e}{\partial{}T} \right |_{\rho,X_k}}

\newcommand{\Ic}{{\boldsymbol{\mathcal{I}}}}
\newcommand{\Ics}{{\mathcal{I}}}
\newcommand{\smax}{{s_\mathrm{max}}}
\newcommand{\kth}{k_\mathrm{th}}
\usepackage{bm}

\newcommand{\Uc}{{\,\bm{\mathcal{U}}}}
\newcommand{\Fb}{\mathbf{F}}
\newcommand{\Sc}{\mathbf{S}}

\newcommand{\xv}{{(x)}}
\newcommand{\yv}{{(y)}}
\newcommand{\zv}{{(z)}}

\newcommand{\ex}{{\bf e}_x}
\newcommand{\ey}{{\bf e}_y}
\newcommand{\ez}{{\bf e}_z}

\newcommand{\Ab}{{\bf A}}
\newcommand{\Sq}{{\bf S}_\qb}
\newcommand{\Sqhydro}{{\Sq^{\mathrm{hydro}}}}
\newcommand{\qb}{{\bf q}}

\newcommand{\Shydro}{{{\bf H}}}
\newcommand{\Hb}{{\bf H}}
\newcommand{\Rb}{{\bf R}}
\newcommand{\Rq}{{\bf R}}
\newcommand{\Adv}[1]{{\left [\boldsymbol{\mathcal{A}} \left(#1\right)\right]}}
\newcommand{\Advt}[1]{{\left [\boldsymbol{\mathcal{\tilde{A}}} \left(#1\right)\right]}}
\newcommand{\Advss}[1]{{\left [{\mathcal{{A}}} \left(#1\right)\right]}}
\newcommand{\Advsst}[1]{{\left [{\mathcal{\tilde{A}}} \left(#1\right)\right]}}
\newcommand{\Advs}[1]{\boldsymbol{\mathcal{A}} \left(#1\right)}

\newcommand{\avg}[1]{{\left \langle #1 \right \rangle}}

\newcommand{\nse}[1]{{\mathtt{NSE}( #1 )}}

\newcommand{\rhonse}{{\rho_\mathrm{nse}}}
\newcommand{\tnse}{{T_\mathrm{nse}}}
\newcommand{\Anse}{{A_\mathrm{nse}}}
\newcommand{\Bnse}{{B_\mathrm{nse}}}
\newcommand{\Cnse}{{C_\mathrm{nse}}}

\newcommand{\out}{{\rm out}}
\newcommand{\inp}{{\rm in}}


\setlength{\marginparwidth}{0.75in}
\newcommand{\MarginPar}[1]{\marginpar{\vskip-\baselineskip\raggedright\tiny\sffamily\hrule\smallskip{\color{red}#1}\par\smallskip\hrule}}

\begin{document}
%======================================================================
% Title
%======================================================================
%\title{A Simplified Spectral Deferred Correction Method for Coupling Hydrodynamics with Reaction Networks and Nuclear Statistical Equilibrium}
\title{An Improved Method for Coupling Hydrodynamics with Reaction Networks and Nuclear Statistical Equilibrium}

%\shorttitle{A Simplified SDC Method}
\shorttitle{Hydro/Reaction Coupling with NSE}


\shortauthors{}

\author[0000-0001-8401-030X]{M.~Zingale}
\affiliation{Dept.\ of Physics and Astronomy, Stony Brook University,
  Stony Brook, NY 11794-3800}

\author[0000-0003-0439-4556]{M.~P.~Katz}
\affiliation{Nvidia Corp}

\author[0000-0003-1791-0265]{A.~J.~Nonaka}
\affiliation{Center for Computational Sciences and Engineering, Lawrence Berkeley National Laboratory, Berkeley, CA  94720}

\author[0000-0002-1530-781X]{Alice Harpole}
\affiliation{Dept.\ of Physics and Astronomy, Stony Brook University,
             Stony Brook, NY 11794-3800}

\correspondingauthor{Michael Zingale}
\email{michael.zingale@stonybrook.edu}


%======================================================================
% Abstract and Keywords
%======================================================================
\begin{abstract}
Reacting astrophysical flows can be challenging to model because of
the difficulty in accurately coupling hydrodynamics and reactions.
This can be particularly acute during explosive burning or at high
temperatures where nuclear statistical equilibrium is established.  We
develop a simplified approach based on the ideas of spectral deferred
corrections (SDC) coupling of explicit hydrodynamics and stiff
reaction sources as an alternative to operator splitting or the more
comprehensive SDC approach we demonstrated previously.  We apply the
new method to some example problems and show how to modify it to work
with a hybrid network consisting of a reaction ODE system and a table
for nuclear statistical equilibrium.  This is all done in the
framework of the \castro\ hydrodynamics code, and all algorithm
implementations are freely available.
\end{abstract}

\keywords{hydrodynamics---methods: numerical}

%======================================================================
% Introduction
%======================================================================
\section{Introduction}\label{Sec:Introduction}

Modeling astrophysical reacting flows can be challenging because of
the disparity better the nuclear and hydrodynamics timescales.
Reaction networks tend to be stiff, requiring implicit integration
techniques to stably integrate the system.  In contrast, compressible
hydrodynamics flows are limited by the (often much longer)
sound-crossing time over a computational cell and can be solved using
explicit time integration. Traditional methods of coupling
hydrodynamics and reactions used in astrophysics use operator
splitting---each physical process acts on the output of the previous
process in alternating fashion.  This makes it easy to use different
time-integration methods for the different physics, and to build a
simulation code in a modular way.  However, competition between the
different physical processes can cause the coupling to breakdown.
These splitting errors can lead to loss of accuracy and further time
step limitations.

A particularly difficult phase of evolution to model is the nuclear
statistical equilibrium that sets in for
temperatures in excess of $\mbox{few} \times 10^9~\mathrm{K}$.  Physically, the
forward and reverse rates of reaction should balance leading to an equilibrium.
With operator splitting, an NSE region will have a large positive flow
through the network in a zone in one step followed by a large negative
flow over the next timestep, as the code struggles to produce an
equilibrium.  These large changes in abundances (and large alternately
positive and negative energy generation rates) can be a challenge for
a code.  The easiest way to improve the coupling is to cut the
timestep, but this makes simulations prohibitively expensive.
Sometimes the burning is simply halted on a zone-by-zone basis when
NSE conditions are reached (e.g., as in \citealt{hedet}).
Alternately, at high temperatures, a reaction network can be replaced
with a table of NSE abundances and the zone's composition set through
table look-ups (e.g.\ \citealt{ma:2013}) \MarginPar{need more here}

The \castro\ hydrodynamics code \citep{castro,castro_joss} is used for
all of our numerical experiments.  \castro\ is a compressible
(magneto-, radiation-) hydrodynamics code built on the AMReX adaptive
mesh refinement (AMR) framework \citep{amrex_joss}.  \castro\ has been
designed to be performance portable and runs on massively parallel CPU
and GPU architectures \citep{castro_gpu}.  For hydrodynamics, the
corner transport upwind (CTU) \citep{ppmunsplit} method with the
piecewise parabolic method (PPM) \citep{ppm,millercolella:2002} is
used.  \castro\ includes self-gravity, rotation, arbitrary equations
of state and reaction networks, and has been used for modeling X-ray
bursts and different models of thermonuclear, core-collapse, and
pair-instability supernovae.  Recently in \citet{castro_sdc}, we
developed second- and fourth-order accurate methods in space and time
for coupling hydrodynamics and nuclear reaction networks based on
spectral deferred corrections (SDC) methods, and demonstrated the
method in a variety of test problems.

The time-integration approach presented here is considerably simpler than
the SDC method of \citet{castro_sdc}, but allows us to reuse the main CTU
hydrodynamics construction and a largely similar ODE integration
scheme, making this method easier to add to existing simulation codes.
Furthermore, it also extends to adaptive mesh refinement with
subcycling in a straightforward manner, avoiding the complications
described in \citep{mccorquodalecolella} needed to fill ghost cells
when using method-of-lines integration.  However, it is restricted to
second-order accuracy overall.  We term this algorithm the
``simplified SDC method''.  In this paper we describe the overall method, show its extension to NSE,
and demonstrate it on several test problems using \castro.  All of the
code to reproduce the results in this paper are in the \castro\ github
repository\footnote{\url{https://github.com/amrex-astro/Castro/}}.

\section{Numerical Methodology}


We solve the Euler equations for compressible, reacting flow.  For ease
of exposition we describe the one-dimensional case;
multidimensional extensions are a straightforward modification to
include the CTU hydrodynamics scheme.  Our conserved variables are
\begin{equation}
  \Uc = \left ( \begin{array}{c}
           \rho \\
           \rho X_k \\
           \rho \alpha_l \\
           \rho u \\
           \rho E \\
           \rho e \end{array}\right )
\end{equation}
where $\rho$ is the mass density, $u$ is velocity, $E$ is specific
total energy, $p$ is the pressure, and we carry nuclear species mass
fractions, $X_k$, and auxiliary composition variables, $\alpha_l$.  The specific total
energy relates to the specific internal energy, $e$, as $E = e + u^2/2$,
and we also separately evolve $e$, as part of a dual
energy formulation (see \citealt{bryan:1995,wdmergerI}).
 The mass fractions are constrained to sum to 1, $\sum_k X_k = 1$, but
 no such constraint exists on the auxiliary variables. 
Defining the hydrodynamical fluxes:
\begin{equation}
  \Fb(\Uc) = \left ( \begin{array}{c}
         \rho u \\
         \rho X_k u \\
         \rho \alpha_l u \\
         \rho u^2 p \\
         (\rho E + p) u \\
         \rho e u \end{array}\right )
\end{equation}
We can write the system in conservative form for all state variables aside from $(\rho e)$ as:
\begin{equation}
  \ddt{\Uc} + \ddx{\Fb(\Uc)} = \Sc(\Uc)
\end{equation}
where for the special case of $(\rho e)$, we have an additional ``pdV'' term:
\begin{equation}
\ddt{(\rho e)} + \ddx{F(\rho e)} + p \ddx{u} = S(\rho e)
\end{equation}

We'll split the source term, $\Sc(\Uc)$ into pure hydrodynamic and reactive parts:
\begin{equation}
  \Hb(\Uc) = \left ( \begin{array}{c}
    0 \\
    0 \\
    0 \\
    \rho g \\
    \rho u g \\
    0 \end{array} \right )
  \qquad
  \Rb(\Uc) = \left ( \begin{array}{c}
     0 \\
     \rho \omegadot(X_k) \\
     \rho \omegadot(\alpha_l) \\
     0 \\
     \rho \dot{S} \\
     \rho \dot{S} 
  \end{array} \right )
\end{equation}
with 
\begin{equation}
  \Sc(\Uc) = \Hb(\Uc) + \Rb(\Uc).
\end{equation}
Here, gravity, represented by the gravitational acceleration $g$
appears as a pure hydrodynamical source term.  From reactions,
$\omegadot(X_k)$ is the creation rate for species $k$,
$\omegadot(\alpha_l)$ is the creation rate for auxiliary composition
variable $l$, and $\dot{S}$ is the energy generation rate per unit
mass.  We note that the internal energy ``pdV'' work is not treated as
a source term but is instead constructed with the hydrodynamical
fluxes are computed in the CTU method.

Aside from the $p\partial u/\partial x$ (``pdV'') term in the internal
energy equation, this system is in conservative form with source
terms.

We'll define the advective terms with hydrodynamic sources, $\Advs{\Uc}$ as
\begin{equation}
\Advs{\Uc} = -\ddx{\Fb(\Uc)} + \Hb(\Uc)
\end{equation}
for the general case, with the $(\rho e)$ component again having the extra
``pdV'' term:
\begin{equation}
\mathcal{A}(\rho e) = -\ddx{F(\rho e)} -p \ddx{u} + H(\rho e)
\end{equation}


To close the system, we need the equation of state.  For a system
where the composition is completely specified by the mass fractions,
$X_k$, the equation of state would take the form:
\begin{equation}
p = p(\rho, X_k, e)
\end{equation}
However, as we'll see shortly, when we use the NSE table, we are using
the results of a much larger network then can be represented by the
mass fractions, and in this case, we rely on the auxiliary composition
variables, $\alpha_l$, to describe the state of the composition, and
our EOS has the form:
\begin{equation}
p = p(\rho, \alpha_l, e)
\end{equation}


Sometimes it is preferable to work with the primitive variables,
\begin{equation}
\qb = \left ( \begin{array}{c}
  \rho \\
  X_k \\
  \alpha_l \\
  u \\
  p \\
  (\rho e) \\
\end{array} \right )
\end{equation}
Here, the system appears
as:
\begin{equation}
\qb_t + \Ab^\xv(\qb) \qb_x  = \Sc(\qb)
\end{equation}
with the matrix $\Ab^\xv$ giving the coefficients of the spatial derivatives
of the primitive variables:
\begin{equation}
\Ab^\xv(\qb) = \left ( \begin{array}{cccccc}
    u & 0 & 0 & \rho & 0 & 0 \\
    0 & u & 0 & 0    & 0 & 0 \\
    0 & 0 & u & 0    & 0 & 0 \\
    0 & 0 & 0 & u    & 1/\rho & 0 \\
    0 & 0 & 0 & \Gamma_1 p & u & 0 \\
    0 & 0 & 0 & \rho h & 0 & u
  \end{array} \right )
\end{equation}
where $h$ is the specific enthalpy and $\Gamma_1$ is an adiabatic index,
$\Gamma_1 = d\log p/d\log\rho|_s$ at constant entropy.
The CTU+PPM algorithm uses the characteristic wave structure of $\Ab$
to collect the information that makes it to an interface over a timestep
in order to compute the fluxes through the interface.
Note, the primitive state has two thermodynamic quantities, $p$
and $(\rho e)$, to more efficiently handle the general equation of
state in the Riemann solver, as described in \citet{castro}, but
alternate formulations are possible \citep{colellaglaz:1985}.
The source term vector, $\Sc(\qb)$, can again be decomposed into hydrodynamic
sources (now in terms of the primitive variables) and reaction terms,
\begin{equation}
  \Sc(\qb) = \Hb(\qb) + \Rb(\qb)
\end{equation}
with
\begin{equation}
\label{eq:prim_sources}
\Hb(\qb) = \left ( \begin{array}{c}
     0 \\
     0 \\
     0 \\
     g \\
     0 \\
     0 \\
   \end{array} \right )
\qquad
\Rb(\qb) = \left ( \begin{array}{c}
     0 \\
     \omegadot(X_k) \\
     \omegadot(\alpha_l) \\
     0 \\
     \Gamma_1 p \sigma \Sdot \\
     \rho \Sdot
   \end{array} \right )
\end{equation}
where
\begin{equation}
\sigma \equiv \frac{\partial p/\partial T |_\rho}{\rho c_p \partial p/\partial \rho |_T}
\end{equation}
and $c_p$ is the specific heat at constant pressure, $c_p = \partial
h/\partial T |_p$.  A derivation of this source for the pressure
equation can be found in \cite{ABNZ:III}.  We note that this source is
algebraically identical to that shown in Eq.~25 of \cite{castro}.

The CTU+PPM method for hydrodynamics is second-order accurate in space
and time.  We want to couple the reactive sources to the hydrodynamics
to be second-order in time as well.  As discussed above, nuclear
reaction sources are stiff, and need to be integrated using implicit
methods for stabilty.  Operator splitting (e.g., Strang) is
traditionally employed here, and is used as a benchmark for comparison
in this paper.  We'll discuss this traditional approach next before moving
on to our new time-coupling method.



\subsection{Strang Splitting}

In Strang splitting, we first integrate the system with reactions terms only (no advection)
over $\Delta t/2$, then integrate the advection terms only (no reactions) over $\Delta t$,
and finally integrate the reaction terms only over $\Delta t/2$.

In the absence of advective terms, our reaction system appears as just
$d\Uc/dt = \Rb(\Uc)$, or:
\begin{align}
\odt{\rho} & = 0 \\
\odt{(\rho X_k)} &= \rho \omegadot(X_k) \\
\odt{(\rho \alpha_l)} &= \rho \omegadot(\alpha_l) \\
\odt{(\rho u)} &= 0 \\
\odt{(\rho E)} &= \rho \Sdot
\end{align}
We can write the energy equation as:
\begin{equation}
\odt{(\rho E)} = \odt{(\rho e)} + \odt{K} = \rho \Sdot
\end{equation}
where $K$ is the kinetic energy, $K = |u|^2/2$.  Since the density and velocity
are unchanged by reactions, our energy equation becomes:
\begin{equation}
\label{eq:strang:e}
\odt{(\rho e)} = \rho \odt{e} = \rho \Sdot
\end{equation}
The reaction rates are typically expressed as $\omegadot_k(\rho, T, X_k)$
when we evolve this system, which requires us to get temperature from
the equation of state each
time we need to evaluate the reactive terms.  

We also typically integrate mass fraction itself, instead of partial
densities:
\begin{equation}
\label{eq:strang:X}
\odt{X_k} = \omegadot_k
\end{equation}
We integrate Equations (\ref{eq:strang:e}) and (\ref{eq:strang:X}) using
an implicit ODE solver designed for stiff systems of
equations, VODE~\citep{vode}.

We explored this and other approaches (including not evolving an
energy / temperature equation during the reaction step) in
\citet{strang_rnaas} and showed that the above formulation gets second
order convergence.  However, for very strong reactions, when using
Strang splitting the state can drift significantly off of the smooth
solution to the coupled reactive hydrodynamics equations, as shown
graphically in \cite{astronum:2018} using an earlier version of the
present algorithm.

A variation on Strang splitting called (re-)balanced splitting was
developed in \citet{speth:2013}.


\subsection{Timestep Limiters and Retry Mechanism}

Since this method is based off of the CTU hydrodynamics scheme, it
benefits from the larger timestep that method can take (when done with
full corner coupling, the advective CFL condition is unity)
as compared to a method-of-lines approach (see
\citealt{ppmunsplit}).  In addition to the standard CFL timestep
limiter for explicit hydrodynamics, \castro\ can also enforce
timestep limiters based on the energy generation or abundance changes
over a timestep: \MarginPar{do we enforce this in the NSE region?}
\begin{align}
\label{eq:dt:nuce}
\Delta t &\le f_e\, \min_{i} \left\{\frac{e_{i}}{\dot{S}_{i}}\right\} \\
%
\label{eq:dt:nucX}
\Delta t &\le f_X\, \min_{i} \left\{\min_{k,X_k > \epsilon_X}\frac{{X_k}_{i}}{{\omegadot(X_k)_i}}\right\}
\end{align}
where $i$ is the zone index and $f_e$ and $f_X$ are runtime parameters
used to control the allowed change, and only species for which $X_k > \epsilon_X$ are considered.

\castro\ has the ability to reject a timestep if it detects a failure
and retry with smaller timesteps (subcycling to make up the original
required timestep).  Among the conditions that can trigger this are
density falling below zero during advection, the ODE integration
failing to converge in the implicit solve, or violation of one of the
timestep limiters during the step.  This means that equations
(\ref{eq:dt:nuce}) and (\ref{eq:dt:nucX}) are not reactive, but instead
guaranteed to be met throughout the simulation, because the step is
rejected if they are violated.  This contrasts to similar approaches
used in other codes where conditions \ref{eq:dt:nuce} and
\ref{eq:dt:nucX} are used to restrict the next timestep size, but the
update over the current step is kept, which already violated the
constraints. \MarginPar{would be nice to have a reference here} The retry mechanism in \castro\ works with both the
Strang and simplified-SDC integration scheme.


\subsection{Spectral Deferred Corrections}

The basic idea of spectral deferred corrections \MarginPar{ref} is to express the
update as an integral and divide the time-update into a number of
discrete time nodes.  The integral is then approximated using a
quadrature rule over these time nodes and low order approximations are
used to update the state from one time node to the next.  The method
uses iteration to successively improve the solution and the ultimate
accuracy is determined by the quadrature method used to evaluate the
integral, which is done using an iteratively-lagged solution.  This is the
classic approach that we implemented in \cite{castro_sdc}.

Our simplified-SDC approach is based on the method described in \citet{SDC-old} and
a similar implementation in the \maestro\ simulation code.
We again start by considering the update
in integral form:
\begin{equation}
\Uc^{n+1} = \Uc^n + \int_t^{t+\Delta t} \left [ \Advs{\Uc} + \Rb(\Uc) \right ] dt,
\end{equation}
In this approach, we 
approximate the advective term
piecewise constant in time, using the value at the midpoint in time,
to achieve second-order accuracy, giving us:
\begin{equation}
\label{eq:integral:simplesdc}
\Uc^{n+1} = \Uc^n + \int_t^{t+\Delta t} \left \{ \Adv{\Uc}^{n+1/2} + \Rb(\Uc) \right \} dt
\end{equation}
The \castro\ version of the simplified-SDC algorithm
differs from the previous versions due to the need to do some operations
on the conserved variable state and some on the primitive variable
state. 

The basic time update algorithm proceeds as:

\begin{itemize}

\item {\em Initialization}

\label{sec:initialization}

  \begin{itemize}
  \item We need an approximation of how much the reactions alone
    changed the primitive variable state over the timestep, which we
    will denote $\Ic_q$ (this is basically an approximation of
    $\Rb(q)$).

    Since we do not have any information about the current
    timestep in the first iteration, we use the value from the last
    iteration of the previous timestep:
    \begin{equation}
      \Ic^{n+1/2,(0)}_{\qb} = \Ic^{n-1/2,(\smax)}_{\qb}
    \end{equation}
  %% \item Solve the Poisson problem for the initial gravitational potential:
  %%   \begin{equation}
  %%     \nabla^2 \Phi^n = 4\pi G \rho^n
  %%   \end{equation}

  %% \item Set the first guess at the new time potential as
  %%   $\Phi^{n+1,(0)} = \Phi^n$.

  \end{itemize}

\item {\em Iterate}

  Iterate from $k = 1, \ldots, \smax$.  For second-order accuracy,
  $\smax = 2$ is sufficient.  In addition to denoting the time-level
  with a superscript (like $n$ or $n+1$), we'll use a second subscript
  in parentheses to keep track of the iteration.  A single iteration
  starts with $\Uc^n$ and results in the new time-level state for that
  iteration, $\Uc^{n+1,(k)}$.

  \begin{itemize}
  \item {\em Create the advective update term, $\Adv{\Uc}^{n+1/2,(k)}$}

    \begin{itemize}
    \item convert $\Uc \rightarrow \qb$.  This is an algebraic transformation,
      but will require the EOS.

    \item predict $\qb$ to the interfaces at $t^{n+1/2}$ using the CTU PPM
      method.  The source terms, $\Sq$, used in the prediction are:
      \begin{equation}
        \Sc(\qb) = \Hb(\qb) + \Ic_\qb^{n+1/2,(k-1)}
      \end{equation}
      Here we use the iteratively lagged integrals of the primitive variable
      terms accounting only for reactions, $\Ic_\qb^{n+1/2,(k-1)}$, as the
      reactive source.  This is in contrast to Strang-splitting, where no
      explicit reactive source terms are seen by the hydrodynamics update.
%%       Any hydrodynamic source terms are time-centered
%%       using the previous iteration:
%% \MarginPar{Formally both $t^n$ or $t^{n+1/2}$ sources give you 2nd-order since they are CTU predictor source terms.  I've never found this type of iteration for this term to amount to any benefit}
%%       \begin{equation}
%%         \Sqhydro^{n+1/2} = \frac{1}{2} \left ( \Sqhydro^n + \Sqhydro^{n+1,(k-1)} \right )
%%       \end{equation}

      In the unsplit CTU method \citep{ppmunsplit}, the interface
      states used for the final Riemann problem through the zone
      interface consist of a normal predictor and a transverse flux
      correction.  We can add the source terms either to the normal
      predictor (for example, doing characteristic tracing as
      described in \citealt{ppm}) or after all of the transverse flux
      corrections are made.  Both are second-order accurate.  For the
      hydrodynamics sources, we do those in the normal predictor.
      However, for the reactive sources, we found that it is most
      reliable to add them at the end of the interface state
      construction, after the transverse flux corrections.  This is
      because we want to ensure that sum over species of $\Ic_\qb$ is
      zero, and characteristic tracing or the various flux corrections
      may not preserve this.  We enforce that the species interface
      states remain in $[0, 1]$ after adding the reactive source.


    \item solve the Riemann problem at each interface to get a unique
      conserved state on each interface, $\Uc^{n+1/2,(k)}_{i+1/2}$

    \item construct the advective update terms, $\Advt{\Uc}^{n+1/2,(k)}_{i}$,
      by first ignoring the hydrodynamics sources,
      \begin{equation}
        \Advt{\Uc}^{n+1/2,(k)}_{i} =
          - \frac{\Fb^\xv(\Uc^{n+1/2,(k)}_{i+1/2}) - \Fb^\xv(\Uc^{n+1/2,(k)}_{i-1/2})}{\Delta x}
      \end{equation}
      for the general case, and
      \begin{align}
        \Advsst{\rho e}^{n+1/2,(k)}_{i} =
          &- \frac{F\left((\rho e)^{n+1/2,(k)}_{i+1/2}\right) - F\left((\rho e)^{n+1/2,(k)}_{i-1/2}\right)}{\Delta x} \\
          &- \left (\frac{p_{i+1/2}^{n+1/2,(k)} + p_{i-1/2}^{n+1/2,(k)}}{2} \right )\frac{u_{i+1/2}^{n+1/2,(k)} - u_{i-1/2}^{n+1/2,(k)}}{\Delta x}
      \end{align}
      for $(\rho e)$.

    Now the conservative hydrodynamics source terms are computed by first updating to the
    new state with advection and the old-time source term applied for the full $\Delta t$ as: 
    \begin{equation}
      \Uc^{\star\star} = \Uc^n + \Delta t \Advt{\Uc}^{n+1/2,(k)} + \Delta t \Hb(\Uc^n)
    \end{equation}
    %% first updating the density to the new state with advection only as:
    %% \begin{equation}
    %%   \rho^{\star\star} = \rho^n + \Delta t \Advt{\rho}^{n+1/2,(k)}
    %% \end{equation}
    %% then constructing the momentum source term:
    %% \begin{equation}
    %%   {\bf S}_{\rho\Ub}^{n+1/2} = \frac{1}{2} (\rho^n \gb^n + \rho^{\star\star} \gb^{n+1,(k-1)})
    %% \end{equation}
    %% Then updating the momentum with only advection as:
    %% \begin{equation}
    %%   (\rho \Ub)^{\star\star} = (\rho \Ub)^n + \Delta t \Advt{\rho\Ub}^{n+1/2,(k)} + \Delta t {\bf S}_{\rho\Ub}^{n+1/2}
    %% \end{equation}
    %% and finally constructing the energy source:
    %% \begin{equation}
    %%   S_{\rho E}^{n+1/2} = \frac{1}{2} \left [ (\rho\Ub)^n \cdot \gb^n + (\rho\Ub)^{\star\star} \cdot \gb^{n+1,(k-1)}
    %%     \right ]
    %% \end{equation}
    We then evaluate the source terms with $\Uc^{\star\star}$ and
    correct the advective term so that we have a time-centered
    source.
    The final advective update term is then\footnote{For a source like gravity, this update can be done first for $\rho$ and then define the new momentum source using $\rho^{\star\star}$ and likewise for energy}:
    \begin{equation}
      \Adv{\Uc}^{n+1/2,(k)}_{i} = \Advt{\Uc}^{n+1/2,(k)}_{i} +
      \frac{\Delta t}{2} \left (\Shydro(\Uc^n) + \Shydro(\Uc^{\star\star}) \right )
    \end{equation}

    %% with
    %% \begin{equation}
    %%   \Shydro^{n+1/2} = \left ( \begin{array}{c}
    %%                  0 \\ 0 \\ 0 \\
    %%                 {\bf S}_{\rho\Ub}^{n+1/2} \cdot \ex \\
    %%                 S_{\rho E}^{n+1/2} \end{array} \right )
    %% \end{equation}

    We note that this procedure works because the source terms here do
    not depend on the outcome of the reactions.

    \end{itemize}

  \item {\em Update the System Using a Method of Lines Integration}

    We update the state by doing the integral in equation
    (\ref{eq:integral:simplesdc}).  Since we are approximating the
    advective term as piecewise constant in time, we can simply use an
    ODE integrator to integrate this, just as we do with the reaction
    system in Strang splitting.  The difference here being that we are
    integrating the conserved variables and the state sees the effect
    of advection as we integrate the reactions.  So rather than use
    $d\Uc/dt = \Rb(\Uc)$, the ODE form we use is
    \begin{equation}
      \odt{\Uc} = \Adv{\Uc}^{n+1/2,(k)} + \Rb(\Uc)
    \end{equation}

    Looking at the form of Eq.~\ref{eq:react_source}, we see that only
    the species, auxiliary quantities, and energies have source terms.
    Since we are reacting, we will use the $X_k$ as the primary
    composition variable and compute $\alpha_l$ as needed from them,
    so we don't directly integrate the auxiliary quantities.
    Likewise, the total and internal energies both provide the same
    information, and integrating both overconstrains the system, so we
    just integrate the internal energy.  We define the subset of variables
    that are directly integrated as:
    \begin{equation}
      \Uc^\prime = \left ( \begin{array}{c} \rho X_k \\ \rho e \end{array} \right )
    \end{equation}
    and we integrate
    \begin{equation}
      \odt{\Uc^\prime} = \Adv{\Uc^\prime}^{n+1/2,(k)} + \Rb(\Uc^\prime)
    \end{equation}
    This integration begins with $\Uc^{\prime,n}$ and results in $\Uc^{\prime,n+1,(k)}$.

    We will need the density at times during the integration, which we construct as:
    \begin{equation}
      \rho(t) = \rho^n + \Advss{\rho}^{n+1/2,(k)} \, t
    \end{equation}
    As we are integrating this system we need to get the temperature,
    $T$, for the rate evaluations.  We obtain this by directly from
    internal energy, composition, and density using the equation of
    state.

    Our integrator also needs the Jacobian of the system, in terms of
    the $\Uc^\prime$.  This is different than the form of the
    Jacobian usually used in reaction networks.  We describe the form
    of the Jacobian in appendix \ref{sec:app:jacobian}.


    At the end of the integration, we can do the conservative update of momentum
    and energy.  Momentum is straightforward, since there are no reactive sources:
    \begin{equation}
      (\rho u)^{n+1} = (\rho u)^n + \Delta t \Advss{\rho u}^{n+1/2,(k)}
    \end{equation}
    For total energy, we first need to isolate the reactive source for energy:
    \begin{equation}
      (\rho \Sdot)^{n+1/2} = \frac{(\rho e)^{n+1} - (\rho e)^n}{\Delta t} - \Advss{\rho e}^{n+1/2,(k)}
    \end{equation}
    Then the update is
    \begin{equation}
      (\rho E)^{n+1} = (\rho E)^n + \Delta t \Advss{\rho E}^{n+1/2,(k)} + \Delta t (\rho \Sdot)^{n+1/2}
    \end{equation}

  \item {\em Compute the Reactive Source Terms.}

    We now seek the $\Ic$'s that capture the effect of just the
    reaction sources on the state variables for the next iteration.
    For the conserved quantities, these would simply be:
    \begin{equation}
      \Ic^{(k)}_{\Uc} = \frac{\Uc^{n+1,(k)} - \Uc^n}{\Delta t} - \Adv{\Uc}^{n+1/2,(k)}
    \end{equation}
    However, for our primitive variables, which are used in the
    prediction, we need to construct the required source terms more
    carefully.  We want:
    \begin{equation}
      \label{eq:Iq}
      \Ic^{(k)}_{\qb} = \frac{\qb^{n+1,(k)} - \qb^n}{\Delta t} - \Adv{\qb}^{n+1/2,(k)}
    \end{equation}
    but we need the advective update for $\qb$, which we have not
    constructed.  We note that $\Ic^{(k)}_\qb$ is an approximation to the integral of 
    Eq.~\ref{eq:prim_sources} over the timestep.  Additionally, we cannot simply use the equation of
    state on $\Ic^{(k)}_{\Uc}$ since this is a time-derivative and
    does not represent a well-defined state in itself.  Instead, we
    derive $\Ic^{(k)}_{\qb}$ via a multi-step process.  We first find
    the conservative state as if it were updated only with advection:
    \begin{equation}
      \Uc^\star = \Uc^n + \Delta t \Adv{\Uc}^{n+1/2,(k)}
    \end{equation}
    and then construct the corresponding primitive variable state via an algebraic transform,
    $\Uc^\star \rightarrow \qb^\star$.
    This allows us to define the advective update for $\qb$ as:
    \begin{equation}
      \Adv{\qb}^{n+1/2,(k)} = \frac{\qb^\star - \qb^n}{\Delta t}
    \end{equation}
    Defining the primitive state corresponding to the fully-updated
    conserved state via an algebraic transform, $\Uc^{n+1,(k)}
    \rightarrow \qb^{n+1,(k)}$, we can construct $\Ic^{(k)}_{\qb}$ as given
    in Equation (\ref{eq:Iq}).
    Putting all of this together, we see:
    \begin{equation}
      \Ic^{(k)}_{\qb} = \frac{\qb^{n+1,(k)} - \qb^\star}{\Delta t}
    \end{equation}



  %% \item {\em Solve for the New Gravitational Potential.}

  %%   We solve
  %%   \begin{equation}
  %%     \nabla^2 \Phi^{n+1,(k)} = 4\pi G \rho^{n+1,(k)}
  %%   \end{equation}

  \end{itemize}

\end{itemize}



\section{Reaction Networks and Nuclear Statistical Equilibrium}

For this study we will use the 19 nuclei network containing
\isot{H}{1}, \isot{He}{3}, \isot{He}{4}, \isot{C}{12}, \isot{N}{14},
\isot{O}{16}, \isot{Ne}{20}, \isot{Mg}{24}, \isot{Si}{28},
\isot{S}{32}, \isot{Ar}{36}, \isot{Ca}{40}, \isot{Ti}{44},
\isot{Cr}{48}, \isot{Fe}{52}, \isot{Fe}{54}, \isot{Ni}{56}, protons
(from photodisintegration), and neutrons.  This is based on the
``aprox19'' network from \cite{aprox19} and originally described in
\cite{Kepler}.  We use a modified version of the VODE~\citep{vode}
integrator---ported to C++ with checks added to the timestep rejection
logic that ensure that the species mass fractions stay between 0 and \MarginPar{we need to be clear that there is different logic for SDC}
1\footnote{This modified version of VODE is available in our
Microphysics repo:
\url{https://github.com/starkiller-astro/Microphysics}.}.\MarginPar{discuss
  tolerances} We combine this with a table that gives the nuclear
statistical equilibrium (NSE) abundances in regions where the system
is in NSE.  The NSE table was generated using a 127 nuclei reaction
network and is the same as described in \cite{ma:2013}.  In our
simulations, we carry all 19 isotopes in the main network in each zone
and advect them in the hydrodynamics portion of the algorithm.  The
composition of the larger 127 nuclei network is mapped into the 19
isotopes we carry according to Table \ref{table:nuclei}.

The NSE table requires: \MarginPar{need a statement about how the table was constructed and what NSE means}
\begin{equation}
  \label{eq:aux:ye}
  Y_e = \sum_k \frac{Z_k X_k}{A_k}
\end{equation}
(where $A_k$ and $Z_k$ are the atomic weight and atomic number of nucleus $k$) and provides
\begin{align}
\label{eq:aux:abar}
\bar{A} &= \left [ \sum_k \frac{X_k}{A_k} \right ]^{-1} \\
\label{eq:aux:bea}
\left (\frac{B}{A} \right ) &= \sum_k \frac{B_k X_k}{A_k}
\end{align}
where $B_k$ is the binding energy of nucleus $k$.  In our simulations,
we store these 3 quantities as auxiliary data that is carried along
with the rest of the fluid state in each zone.  The table also returns
values of the mass fractions mapped onto the 19-isotopes we carry, $X_k$, and the time-derivative of $Y_e$, $dY_e/dt$.
We use the notation:
\begin{equation}
\nse{\rho,T,Y_e} \rightarrow \bar{A}, X_k, (B/A), dY_e/dt
\end{equation}
to represent the NSE table call and its inputs and outputs.

We can derive evolution equations for each of these composition quantities as:
\begin{align}
\frac{DY_e}{Dt} &= \sum_k \frac{Z_k}{A_k} \frac{DX_k}{Dt} = \sum_k \frac{Z_k}{A_k} \omegadot_k \\
\frac{D\bar{A}}{Dt} &= -\bar{A}^2 \sum_k \frac{1}{A_k} \frac{DX_k}{Dt} = -\bar{A}^2 \sum_k \frac{1}{A_k} \omegadot_k \\
\frac{D}{Dt} \left (\frac{B}{A} \right ) &= \sum_k \frac{B_k}{A_k} \frac{DX_k}{Dt} = \sum_k \frac{B_k}{A_k} \omegadot_k
\end{align}
For Strang split coupling of hydro and reactions, $DX_k/Dt = 0$,
therefore each of these auxillary equations obeys an advection
equation in the hydro part of the advancement.  In the SDC algorithm,
there will be a reactive source (an $\Ic_q$) for each of these that is
computed in the same manner as above.  We note that our NSE table
provides the evolution of $Y_e$ due to reactions ($DY_e/Dt$) directly.

The compositional quantities
it carries, $\bar{A}$ and $Y_e$ are not representable from the 19
isotopes we carry in the main network. For this reason, when we are
using the NSE network, we always provide these two composition quantities as
inputs to the EOS rather than using the $X_k$ directly.
the EOS directly from the auxiliary state in each zone instead of
using the $X_k$ directly.
Our equation of state needs the mean charge per nucleus, $\bar{Z}$, in addition
to the auxiliary quantities, which is computed as
\begin{equation}
\bar{Z} = \bar{A} \sum_k \frac{Z_k X_k}{A_k} = \bar{A} Y_e
\end{equation}
(see, e.g., \citealt{flash}).

\subsection{Initialization}

We initialize the mass fractions, $X_k$, and then compute the
electron fraction, $Y_e$, using (\ref{eq:aux:ye}).
For the two other composition
quantities we carry, $\bar{A}$, and $(B/A)$, we need values that are
consistent with the value of $Y_e$ and the nuclei stored in the NSE
table.  Therefore, if the thermodynamic conditions put us in NSE
(using the conditions defined below), then we obtain $\bar{A}$ and
$(B/A)$ from the NSE table.  Otherwise, we compute these directly from
$X_k$ using (\ref{eq:aux:abar}) and (\ref{eq:aux:bea}).

\subsection{NSE condition}

We treat a zone as being in NSE if the density and temperature exceed
some threshold and the composition is mainly $\alpha$ and Fe-group
nuclei, where the Fe-group nuclei are \isot{Cr}{48}, \isot{Fe}{52},
\isot{Fe}{54}, and \isot{Ni}{56}.  The full condition is:
\begin{align}
\rho &> \rhonse \\
T &> \tnse \\
X(\isotm{C}{12}) &< \Anse \\
X(\isotm{He}{4}) + \sum_{k \in \mathrm{Fe-group}} X_k &> \Bnse \\
X(\isotm{Si}{26}) &< \Cnse 
\end{align}
Typical values are $\rhonse = 2\times 10^6~\gcc$, $T_\mathrm{nse} =
3\times 10^9~\mathrm{K}$, $\Anse = 0.12$, $\Bnse=
0.88$, and $\Cnse = 0.01$.  The values of $\Anse$ and $\Bnse$ are based on \citet{ma:2013}.
The value of $\Cnse$ ensures that Si-burning has ended before invoking the table.


\subsection{Strang-split algorithm for NSE}

For Strang splitting, we alternate the reaction and hydrodynamics,
with each operation working on the output of the previous operation.
If we define an advection operator over a timestep $\Delta t$ acting on $\Uc$ as
$\mathbb{A}_{\Delta t}(\Uc)$ and a reaction operator over a timestep of
$\Delta t/2$ acting on $\Uc$ as $\mathbb{R}_{\Delta t/2}(\Uc)$, then the advance appears
as:
\begin{align}
  \Uc^\star &= \mathbb{R}_{\Delta t/2}(\Uc^n) \\
  \Uc^{n+1,\star} &= \mathbb{A}_{\Delta t}(\Uc^\star) \\
  \Uc^{n+1} &= \mathbb{R}_{\Delta t/2}(\Uc^{n+1,\star})
\end{align}
Tthe hydrodynamic and reactive substeps over
the overall time-advancemenet scheme using the aprox19 + NSE network
are as follows:
\begin{itemize}

\item {\em Hydrodynamics update}
  
  At the beginning of each hydrodynamic update we have an input state,
  $\Uc_{\rm in}$ that we wish to integrate over $\Delta t$ to obtain $\Uc_{\rm out}$
  The hydrodynamics update proceeds as normal, but with an advection
  equation for each of the auxiliary composition variables:
  \begin{align}
    \ddt{(\rho Y_e)} + \ddx{(\rho Y_e u)} &= 0 \\
    \ddt{(\rho \bar{A})} + \ddx{(\rho \bar{A} u)} &= 0 \\
    \ddt{[\rho (B/A)]} + \ddx{[\rho (B/A) u]} &= 0 
  \end{align}


\item {\em Reactive update}

  At the beginning of each reactive update we have an input state,
  $\Uc_{\rm in}$ that we wish to integrate over $\Delta t/2$ to obtain $\Uc_{\rm out}$.
  
  \begin{itemize}

    \item For a zone that is in NSE:

      The goal is to update the composition and thermodynamics due to the 
      change in the nuclei abundances over the (half-)timestep.  \citet{ma:2013}
      uses a first-order in time difference to get the new composition state
      and evaluates the energy release from the change in binding energy in the NSE
      state.  They only apply a fraction (0.3) of the energy release to the internal
      energy in a zone, to avoid a potential instability that can arise if too much
      energy is added to a zone in a single timestep.  We prefer to deal with this
      issue through the retry mechanism already built into \castro.

      The approach we use begins with calling the NSE table with the input state:
      \begin{equation}
        \nse{\rho_\inp, T_\inp, (Y_e)_\inp} \rightarrow \bar{A}^\star, (X_k)^\star, (B/A)^\star, dY_e/dt
        \end{equation}
      We indicate with a $\star$ that most of those values are
      provisional, and we will seek a better value in the corrector step.
      We only update $Y_e$ from this call:
      \begin{equation}
        (Y_e)_\out = (Y_e)_\inp + \Delta t \frac{dY_e}{dt}
      \end{equation}
      and then we use the table again, but with the updated $Y_e$
      (note that $\rho_{\rm out}=\rho_{\rm in}$ in the Strang reaction formulation):
      \begin{equation}
        \nse{\rho_\out, T_\inp, (Y_e)_\out} \rightarrow \bar{A}_\out, (X_k)_\out, (B/A)_\out, dY_e/dt
      \end{equation}
      In this formulation, we have not updated the temperature. \MarginPar{discuss this more, and perhaps switch to a mode where we try to predict T}
      This corrects the composition so that it is consistent with the updated
      $Y_e$.  We use these values $\bar{A}_\out$, $(X_k)_\out$, and $(B/A)_\out$ to then complete the update, computing the energy release, $\Sdot$ as:
      \begin{equation}
        \label{eq:nse_energy}
        \Sdot = \left [ \left ( \frac{B}{A} \right )_\out -
          \left ( \frac{B}{A} \right )_\inp \right ] N_A \frac{1}{\Delta t}
      \end{equation}
      \MarginPar{we could imagine looping over the EOS to get an updated T and then the table to get the updated $\bar{A}$, to see if that converges?}
      
    \item For zones not in NSE:

    \begin{itemize}
    \item Integrate the full reaction network (Eqs.~\ref{eq:strang:e} and \ref{eq:strang:X}) as usual for Strang splitting
    \item Update the aux quantities at the end of the burn using Eqs.~\ref{eq:aux:ye}, \ref{eq:aux:abar}, and \ref{eq:aux:bea} with the new mass fractions, $X_k$.
    \end{itemize}
  \end{itemize}
\end{itemize}

\subsection{SDC-NSE Coupling}

With SDC evolution, when we are in NSE, we need to do the advective
and reactive updates together.  We want to compute:
\begin{equation}
\Uc^{n+1,(k)} = \Uc^n + \Delta t \Adv{\Uc}^{n+1/2,(k)} + \Delta t \left [\Rb (\Uc) \right ]^{n+1/2,(k)}
\end{equation}
For density and momentum, we can do this update already, since there
are no reactive sources.  That gives us $\rho^{n+1,(k)}$ and $(\rho
\Ub)^{n+1,(k)}$.  For the other quantities, the NSE table gives only the
instantaneous values. aside from $Y_e$.  We therefore do an iterative
update.

Calling the NSE table on the starting state gives us:
\begin{equation}
\nse{\rho^n, T^n, (Y_e)^n} \rightarrow \bar{A}^n, (X_k)^n, (B/A)^n, (dY_e/dt)^n
\end{equation}
We now compute a first approximation to the reactive source:
\begin{align}
R^{(0)}(\rho e) &= 0 \\
R^{(0)}(\rho Y_e) &= \rho^n (dY_e/dt)^n \\
R^{(0)}(\rho \bar{A}) &= 0
\end{align}

Now we iterate $\xi = 0, \ldots N-1$, doing the following:

\begin{itemize}
\item Update the energy as:
   \begin{equation}
     (\rho e)^{n+1,(\xi)} = (\rho e)^n + \Delta t \Advss{\rho e}^{n+1/2,(k)} + \Delta t R^{(\xi)}(\rho e)
   \end{equation}

\item Update the auxiliary data as:
   \begin{equation}
     (\rho \alpha_l)^{n+1,(\xi)} = (\rho \alpha_l)^n + \Delta t \Advss{\rho \alpha_l}^{n+1/2,(k)} + \Delta t R^{(\xi)}(\rho \alpha_l)
   \end{equation}

\item Compute the updated temperature, $T^{n+1,(k)}$ as
  \begin{equation}
    T^{n+1,(\xi)} = T(\rho^{n+1,(k)},  \alpha^{n+1,(\xi)},  e^{n+1,(\xi)})
  \end{equation}
  where $\alpha^{n+1,(\xi)} = (\rho  \alpha)^{n+1,(\xi)} / \rho^{n+1,(k)}$ and $e^{n+1,(\xi)} = (\rho e)^{n+1,(\xi)} / \rho^{n+1,(k)}$.

\item Call the NSE table to get the new NSE state
  \begin{equation}
    \nse{\rho^{n+1,(k)}, T^{n+1,(\xi)}, (Y_e)^{n+1,(\xi)}} \rightarrow \bar{A}^{n+1,(\xi)}, (X_k)^{n+1,(\xi)}, (B/A)^{n+1,(\xi)}, (dY_e/dt)^{n+1,(\xi)}
  \end{equation}

\item Recompute the reactive source terms for the next iteration.

  The energy generation rate is found as:
  \begin{equation}
    (\rho \Sdot)^{n+1/2,(\xi)} = \frac{\rho^{n+1,(k)} + \rho^n}{2}
    \left [ \left ( \frac{B}{A} \right )^{n+1,(\xi)} -
      \left ( \frac{B}{A} \right )^n \right ] N_A \frac{1}{\Delta t}
  \end{equation}
  and our new auxiliary sources are:
\begin{align}
R^{(\xi+1)}(\rho e) &= (\rho \Sdot)^{n+1/2,(\xi)} \\
R^{(\xi+1)}(\rho Y_e) &= \frac{\rho^{n+1,(k)} + \rho^n}{2} \frac{1}{2} \left [ \left (\frac{dY_e}{dt} \right )^n + \left (\frac{dY_e}{dt} \right )^{n+1,(\xi)} \right ] \\
R^{(\xi+1)}(\rho \bar{A}) &= \frac{\rho^{n+1,(k)} + \rho^n}{2} \frac{1}{\Delta t} \left [ \bar{A}^{n+1,(\xi)} - \frac{(\rho \bar{A})^n}{\rho^n} \right ]
\end{align}

\end{itemize}

With the iteration complete, we can do the final update of the total energy using last iterations prediction of the source (for iteration $N$):
\begin{equation}
(\rho E)^{n+1,(k)} = (\rho E)^n + \Delta t \Advss{\rho E}^{n+1/2} + \Delta t (\rho \Sdot)^{n+1/2,(N)}
\end{equation}
and we take the values of the auxiliary state and mass fractions from the last call to the NSE table:
\begin{align}
(\rho Y_e)^{n+1,(k)} &= \rho^{n+1,(k)} (Y_e)^{n+1,(N-1)} \\
(\rho \bar{A})^{n+1,(k)} &= \rho^{n+1,(k)} (\bar{A})^{n+1,(N-1)} \\
(\rho (B/A))^{n+1,(k)} &= \rho^{n+1,(k)} (B/A)^{n+1,(N-1)} \\
(\rho Y_e)^{n+1,(k)} &= \rho^{n+1,(k)} (X_k)^{n+1,(N-1)}
\end{align}

We note that this update is not fully second order, however, the
treatment of $Y_e$ is, and we expect that to have the dominant effect
in the thermodynamic evolution through NSE.

\subsection{NSE Bailout}

There is one additional part of the time integration algorithm.
Because $T$ evolves during the reactions (for both Strang and SDC), it
is possible for a zone to start out not in NSE but evolve into NSE
during the reaction update.  When this happens, the ODE integrator may
fail, because an excessive number of timesteps is required.  Rather
than trigger our retry mechanism and throw away the entire timestep
and start over with a smaller $\Delta t$, we first check if the zone
evolved into NSE, and if so, we redo the zone update using the NSE
prescription.  To ensure that this is not triggered at the very start
of integration, we require a minimum number of steps (10) to have
elapsed before checking for NSE.  After we leave the integrator, the
state is returned to the burner driver where we finish the integration
via NSE if the conditions satisfy the NSE criteria.  We optionally
apply a relaxation factor, typically, $0.9$, to the NSE conditions to
allow a state that is close to NSE after an integrator failure enter
NSE.

\section{Simulations}

\subsection{Reacting convergence test problem}

\cite{castro_sdc} introduced a test problem for measuring convergence
of a reacting hydrodynamic algorithm.  We run that same test here with the
simplified SDC algorithm.

Also explore: PLM vs PPM, 1 vs. 2 vs. 3 iterations

% run on 04-06-21
% Castro       git describe: 21.04-dirty
% AMReX        git describe: 21.03-58-g87c81d0ca
% Microphysics git describe: 21.04-1-g33a4d0fe

\begin{deluxetable}{lllllll}
\tablecaption{\label{table:react_convert_strang} Convergence ($L_1$ norm) for the reacting convergence problem using Strang splitting.}
\tablehead{\colhead{field} & \colhead{$\epsilon_{64 \rightarrow 128}$} & 
           \colhead{rate} & \colhead{$\epsilon_{128\rightarrow 256}$} & 
           \colhead{rate} & \colhead{$\epsilon_{256\rightarrow 512}$}}
\startdata
 $\rho$                      & $2.794 \times 10^{18}$  & 2.044  & $6.777 \times 10^{17}$  & 2.554  & $1.154 \times 10^{17}$  \\
 $\rho u$                    & $6.796 \times 10^{26}$  & 2.448  & $1.245 \times 10^{26}$  & 2.889  & $1.681 \times 10^{25}$  \\
 $\rho v$                    & $6.796 \times 10^{26}$  & 2.448  & $1.245 \times 10^{26}$  & 2.889  & $1.681 \times 10^{25}$  \\
 $\rho E$                    & $2.451 \times 10^{35}$  & 2.351  & $4.803 \times 10^{34}$  & 2.742  & $7.179 \times 10^{33}$  \\
 $\rho e$                    & $2.261 \times 10^{35}$  & 2.320  & $4.526 \times 10^{34}$  & 2.821  & $6.403 \times 10^{33}$  \\
 $T$                         & $2.237 \times 10^{21}$  & 1.691  & $6.927 \times 10^{20}$  & 2.482  & $1.240 \times 10^{20}$  \\
 $\rho X(\isotm{He}{4})$     & $2.878 \times 10^{18}$  & 2.020  & $7.096 \times 10^{17}$  & 2.529  & $1.229 \times 10^{17}$  \\
 $\rho X(\isotm{C}{12})$     & $1.698 \times 10^{17}$  & 1.950  & $4.393 \times 10^{16}$  & 2.232  & $9.353 \times 10^{15}$  \\
 $\rho X(\isotm{O}{16})$     & $1.687 \times 10^{14}$  & 1.660  & $5.338 \times 10^{13}$  & 1.957  & $1.375 \times 10^{13}$  \\
 $\rho X(\isotm{Fe}{56})$    & $2.794 \times 10^{8}$   & 2.044  & $6.777 \times 10^{7}$   & 2.554  & $1.154 \times 10^{7}$   \\
\enddata
\end{deluxetable}


% run on 04-06-21
% Castro       git describe: 21.04-dirty
% AMReX        git describe: 21.03-58-g87c81d0ca
% Microphysics git describe: 21.04-1-g33a4d0fe

\begin{deluxetable}{lllllll}
\tablecaption{\label{table:react_convert_sdc} Convergence ($L_1$ norm) for the reacting convergence problem using SDC integration (2 iterations).}
\tablehead{\colhead{field} & \colhead{$\epsilon_{64 \rightarrow 128}$} & 
           \colhead{rate} & \colhead{$\epsilon_{128\rightarrow 256}$} & 
           \colhead{rate} & \colhead{$\epsilon_{256\rightarrow 512}$}}
\startdata
 $\rho$                      & $2.795 \times 10^{18}$  & 2.044  & $6.777 \times 10^{17}$  & 2.554  & $1.154 \times 10^{17}$  \\
 $\rho u$                    & $6.798 \times 10^{26}$  & 2.449  & $1.245 \times 10^{26}$  & 2.889  & $1.680 \times 10^{25}$  \\
 $\rho v$                    & $6.798 \times 10^{26}$  & 2.449  & $1.245 \times 10^{26}$  & 2.889  & $1.680 \times 10^{25}$  \\
 $\rho E$                    & $2.452 \times 10^{35}$  & 2.352  & $4.804 \times 10^{34}$  & 2.742  & $7.179 \times 10^{33}$  \\
 $\rho e$                    & $2.262 \times 10^{35}$  & 2.321  & $4.527 \times 10^{34}$  & 2.822  & $6.404 \times 10^{33}$  \\
 $T$                         & $2.236 \times 10^{21}$  & 1.691  & $6.927 \times 10^{20}$  & 2.481  & $1.240 \times 10^{20}$  \\
 $\rho X(\isotm{He}{4})$     & $2.879 \times 10^{18}$  & 2.020  & $7.096 \times 10^{17}$  & 2.529  & $1.229 \times 10^{17}$  \\
 $\rho X(\isotm{C}{12})$     & $1.697 \times 10^{17}$  & 1.950  & $4.393 \times 10^{16}$  & 2.231  & $9.357 \times 10^{15}$  \\
 $\rho X(\isotm{O}{16})$     & $1.686 \times 10^{14}$  & 1.659  & $5.338 \times 10^{13}$  & 1.957  & $1.375 \times 10^{13}$  \\
 $\rho X(\isotm{Fe}{56})$    & $2.795 \times 10^{8}$   & 2.044  & $6.777 \times 10^{7}$   & 2.554  & $1.154 \times 10^{7}$   \\
\enddata
\end{deluxetable}


\subsection{NSE convergence test problem}

To test the coupling of the NSE table to the hydrodynamics, we run a
similar problem as above, except with the thermodynamic conditions
appropriate for the matter to be in NSE.  The initial conditions are:
\begin{align}
\rho &= \rho_0 \\
T &= T_0 \left [ 1 + (\delta T) e^{-(r/\lambda)^2} \cos^6(\pi r/L) \right ] \\
Y_e &= (Y_e)_0 \left [ 1 + (\delta Y_e) e^{-(r/\lambda)^2} \cos^6(\pi r/L) \right ] 
\end{align}
where $r$ is the distance from the center of the domain and the
remaining parameters are listed in Table~\ref{table:nse}.  This has
$Y_e$ varying between $[0.47, 0.5]$ initially\MarginPar{check}.  The mass fractions
and remaining composition variables are then initialized from the NSE table, as described in section~\ref{sec:initialization}.

The domain has a size $[0, L]^2$, and the timestep is fixed as:
\begin{equation}
\Delta t = 10^{-3} \left ( \frac{64}{N} \right )~\mathrm{s}
\end{equation}
where $N$ is the number of zones in each direction.

\begin{deluxetable}{lcc}
\tablecaption{\label{table:nse} NSE convergence test problem.}
\tablehead{\colhead{parameter} & \colhead{value}}
\startdata
$\rho_0$ & $5\times 10^8~\gcc$ \\
$T_0$    & $4\times 10^9~\mathrm{K}$ \\
$(\delta T)$ & $0.2$ \\
$(Y_e)_0$   & $0.5$ \\
$(\delta Y_e)$ & $-0.05$ \\
$\lambda$   & $2\times 10^7~\mathrm{cm}$ \\
$L$         & $10^8~\mathrm{cm}$ \\
\enddata
\end{deluxetable}

Figure~\ref{fig:nse_ye} shows the $Y_e$ profile for the Strang evolution case.

\begin{figure}[t]
\centering
\plotone{ye_plot}
\caption{\label{fig:nse_ye} Profile of $Y_e$ through the middle of the domain (slicing along $x$)
for the $128^2$ Strang NSE test simulation.}
\end{figure}


% run on 04-06-21
% Castro       git describe: 21.04-dirty
% AMReX        git describe: 21.03-58-g87c81d0ca
% Microphysics git describe: 21.04-1-g33a4d0fe

\begin{deluxetable}{lllllll}
\tablecaption{\label{table:nse_strang_methodII} Convergence ($L_1$ norm) for the NSE convergence
problem using Strang splitting.}
\tablehead{\colhead{field} & \colhead{$\epsilon_{64 \rightarrow 128}$} & 
           \colhead{rate} & \colhead{$\epsilon_{128\rightarrow 256}$} & 
           \colhead{rate} & \colhead{$\epsilon_{256\rightarrow 512}$}}
\startdata
 $\rho$                      & $4.522 \times 10^{19}$  & 1.659  & $1.432 \times 10^{19}$  & 1.542  & $4.917 \times 10^{18}$  \\
 $\rho u$                    & $1.883 \times 10^{28}$  & 1.681  & $5.873 \times 10^{27}$  & 1.524  & $2.042 \times 10^{27}$  \\
 $\rho v$                    & $1.883 \times 10^{28}$  & 1.681  & $5.873 \times 10^{27}$  & 1.524  & $2.042 \times 10^{27}$  \\
 $\rho E$                    & $5.933 \times 10^{37}$  & 1.613  & $1.940 \times 10^{37}$  & 1.411  & $7.293 \times 10^{36}$  \\
 $\rho e$                    & $5.931 \times 10^{37}$  & 1.613  & $1.939 \times 10^{37}$  & 1.411  & $7.291 \times 10^{36}$  \\
 $T$                         & $9.885 \times 10^{20}$  & 1.207  & $4.281 \times 10^{20}$  & 1.058  & $2.057 \times 10^{20}$  \\
 $\rho X(\isotm{H}{1})$      & $4.522 \times 10^{-11}$ & 1.659  & $1.432 \times 10^{-11}$ & 1.542  & $4.917 \times 10^{-12}$ \\
 $\rho X(\isotm{He}{4})$     & $5.086 \times 10^{17}$  & 1.430  & $1.887 \times 10^{17}$  & 1.187  & $8.288 \times 10^{16}$  \\
 $\rho X(\isotm{C}{12})$     & $1.585 \times 10^{12}$  & 1.829  & $4.462 \times 10^{11}$  & 1.433  & $1.652 \times 10^{11}$  \\
 $\rho X(\isotm{O}{16})$     & $8.060 \times 10^{12}$  & 1.706  & $2.470 \times 10^{12}$  & 1.347  & $9.710 \times 10^{11}$  \\
 $\rho X(\isotm{Cr}{48})$    & $1.149 \times 10^{19}$  & 1.400  & $4.353 \times 10^{18}$  & 1.258  & $1.820 \times 10^{18}$  \\
 $\rho X(\isotm{Fe}{52})$    & $9.018 \times 10^{19}$  & 1.173  & $4.001 \times 10^{19}$  & 1.127  & $1.832 \times 10^{19}$  \\
 $\rho X(\isotm{Fe}{54})$    & $5.036 \times 10^{20}$  & 1.731  & $1.517 \times 10^{20}$  & 1.727  & $4.582 \times 10^{19}$  \\
 $\rho X(\isotm{Ni}{56})$    & $5.686 \times 10^{20}$  & 1.622  & $1.847 \times 10^{20}$  & 1.563  & $6.252 \times 10^{19}$  \\
 $\rho Y_e$                  & $2.379 \times 10^{19}$  & 1.678  & $7.431 \times 10^{18}$  & 1.578  & $2.489 \times 10^{18}$  \\
 $\rho \bar{A}$              & $5.141 \times 10^{21}$  & 1.230  & $2.191 \times 10^{21}$  & 1.087  & $1.032 \times 10^{21}$  \\
 $\rho (B/A)$                & $4.149 \times 10^{20}$  & 1.693  & $1.283 \times 10^{20}$  & 1.562  & $4.345 \times 10^{19}$  \\
\enddata
\end{deluxetable}

% run on 04-06-21
% Castro       git describe: 21.04-dirty
% AMReX        git describe: 21.03-58-g87c81d0ca
% Microphysics git describe: 21.04-1-g33a4d0fe

\begin{deluxetable}{lllllll}
\tablecaption{\label{table:nse_sdc_methodII} Convergence ($L_1$ norm) for the NSE convergence
problem using the simplified-SDC algorithm.}
\tablehead{\colhead{field} & \colhead{$\epsilon_{64 \rightarrow 128}$} & 
           \colhead{rate} & \colhead{$\epsilon_{128\rightarrow 256}$} & 
           \colhead{rate} & \colhead{$\epsilon_{256\rightarrow 512}$}}
\startdata
 $\rho$                      & $4.553 \times 10^{19}$  & 1.636  & $1.465 \times 10^{19}$  & 1.583  & $4.891 \times 10^{18}$  \\
 $\rho u$                    & $1.949 \times 10^{28}$  & 1.650  & $6.210 \times 10^{27}$  & 1.513  & $2.176 \times 10^{27}$  \\
 $\rho v$                    & $1.949 \times 10^{28}$  & 1.650  & $6.210 \times 10^{27}$  & 1.513  & $2.176 \times 10^{27}$  \\
 $\rho E$                    & $5.691 \times 10^{37}$  & 1.622  & $1.848 \times 10^{37}$  & 1.561  & $6.266 \times 10^{36}$  \\
 $\rho e$                    & $5.689 \times 10^{37}$  & 1.622  & $1.848 \times 10^{37}$  & 1.561  & $6.265 \times 10^{36}$  \\
 $T$                         & $5.778 \times 10^{20}$  & 1.740  & $1.729 \times 10^{20}$  & 1.702  & $5.316 \times 10^{19}$  \\
 $\rho X(\isotm{H}{1})$      & $4.553 \times 10^{-11}$ & 1.636  & $1.465 \times 10^{-11}$ & 1.583  & $4.891 \times 10^{-12}$ \\
 $\rho X(\isotm{He}{4})$     & $2.541 \times 10^{17}$  & 1.807  & $7.262 \times 10^{16}$  & 1.634  & $2.339 \times 10^{16}$  \\
 $\rho X(\isotm{C}{12})$     & $1.140 \times 10^{12}$  & 1.838  & $3.189 \times 10^{11}$  & 1.670  & $1.002 \times 10^{11}$  \\
 $\rho X(\isotm{O}{16})$     & $5.413 \times 10^{12}$  & 1.819  & $1.535 \times 10^{12}$  & 1.658  & $4.864 \times 10^{11}$  \\
 $\rho X(\isotm{Cr}{48})$    & $7.020 \times 10^{18}$  & 1.793  & $2.026 \times 10^{18}$  & 1.614  & $6.621 \times 10^{17}$  \\
 $\rho X(\isotm{Fe}{52})$    & $3.631 \times 10^{19}$  & 1.579  & $1.215 \times 10^{19}$  & 1.597  & $4.018 \times 10^{18}$  \\
 $\rho X(\isotm{Fe}{54})$    & $4.670 \times 10^{20}$  & 1.807  & $1.335 \times 10^{20}$  & 1.869  & $3.654 \times 10^{19}$  \\
 $\rho X(\isotm{Ni}{56})$    & $4.858 \times 10^{20}$  & 1.808  & $1.387 \times 10^{20}$  & 1.860  & $3.820 \times 10^{19}$  \\
 $\rho Y_e$                  & $2.372 \times 10^{19}$  & 1.649  & $7.564 \times 10^{18}$  & 1.587  & $2.518 \times 10^{18}$  \\
 $\rho \bar{A}$              & $4.009 \times 10^{21}$  & 1.693  & $1.240 \times 10^{21}$  & 1.690  & $3.843 \times 10^{20}$  \\
 $\rho (B/A)$                & $4.222 \times 10^{20}$  & 1.666  & $1.330 \times 10^{20}$  & 1.605  & $4.374 \times 10^{19}$  \\
\enddata
\end{deluxetable}



\subsection{Detonation}

The purpose here is to look at what timestep is taken when we use the nuclear burning limiters.
Run with aprox19 only, Strang and SDC, then aprox19 + NSE, Strang and NSE.

Look at number of SDC iterations too.

How does the timestep limiter work in the middle of SDC iterating?

aprox21

look at number of RHS calls, wallclock time

output every step and then for a single zone plot enuc every timestep to see if it oscillates


\subsection{Reacting Buoyant Bubble}

We next consider a buoyant, reacting bubble in a stratified,
plane-parallel atmosphere.  This test exercises the treatment of
hydrodynamical sources in the SDC integration.  We use the set of
initial conditions that was first shown in \cite{ABNZ:III}.

\subsection{Massive Star}

Do this in 1-d
Show the amount of work needed for Strang vs. SDC

Play around with the threshold where NSE kicks in

\section{Summary}

We presented a simplified spectral deferred corrections scheme for coupling
hydrodynamics and reactions.

Future work is to extend this methodology to MHD.



\acknowledgements \castro\ is freely available at
\url{http://github.com/AMReX-Astro/Castro}.  All of the code and
problem setups used here are available in the git repo.  The work at
Stony Brook was supported by DOE/Office of Nuclear Physics grant
DE-FG02-87ER40317.  This material is based upon work supported by the
U.S. Department of Energy, Office of Science, Office of Advanced
Scientific Computing Research and Office of Nuclear Physics,
Scientific Discovery through Advanced Computing (SciDAC) program under
Award Number DE-SC0017955.  This research was supported by the
Exascale Computing Project (17-SC-20-SC), a collaborative effort of
the U.S. Department of Energy Office of Science and the National
Nuclear Security Administration.

\software{\amrex\ \citep{amrex_joss},
          \castro\ \citep{castro},
          GNU Compiler Collection (\url{https://gcc.gnu.org/}),
          Linux (\url{https://www.kernel.org}),
          matplotlib (\citealt{Hunter:2007},  \url{http://matplotlib.org/})
          NumPy \citep{numpy,numpy2},
          python (\url{https://www.python.org/}), 
          SymPy \citep{sympy}
         }



\appendix

\section{Jacobian}

\label{sec:app:jacobian}

To solve the reaction system implicitly, the ODE solver needs the
Jacobian, $\partial \Rb/\partial \Uc^\prime$, where $\Uc^\prime =
(\rho X_k, \rho e)^\intercal$ is the subset of the conserved variables we are
integrating.  We follow the method of \cite{castro_sdc} and factor
this into two pieces,
\begin{equation}
\label{eq:factored_jac}
{\bf J} = \frac{\partial \Rb}{\partial {\bf w }} \frac{\partial {\bf w}}{\partial \Uc^\prime}.
\end{equation}
where the state ${\bf w}$ is chosen to match the set of variables used to evaluate the reaction rates.
Writing this out for two species, $X_\alpha$ and $X_\beta$, we have
\begin{equation}
\Uc^\prime = \left ( \begin{array}{c} \rho X_\alpha \\ \rho X_\beta \\ \rho e \end{array} \right )
\end{equation}
For interfacing with the reaction network, we use
\begin{equation}
{\bf w} = \left ( \begin{array}{c} X_\alpha \\  X_\beta \\ T \end{array} \right )
\end{equation}
Note: even though we are using $T$ here instead of $e$, we still do the overall ODE integration
in terms of $(\rho e)$, consistent with the Strang method described in \cite{strang_rnaas}.
%%  in that
%% we use $e$ instead of $T$ as the independent variable.  If the network gives derivatives in terms
%% of temperature, we can convert as:
%% \begin{equation}
%% \left . \frac{\partial \phi}{\partial T} \right |_\rho =
%%   \left . \frac{\partial \phi}{\partial e} \right |_\rho 
%%   \left . \frac{\partial e}{\partial T} \right |_\rho =
%%   c_v \left . \frac{\partial \phi}{\partial e} \right |_\rho
%% \end{equation}
%% for some quantity $\phi(\rho, T)$.  Here we identified the specific heat
%% at constant volume, $c_v = \partial e/\partial T|_\rho$.

The Jacobian transformation $\partial \Uc^\prime/\partial {\bf w}$ is:
\begin{equation}
\frac{\partial \Uc^\prime}{\partial {\bf w}} = \left (
   \begin{array}{ccc}
       \rho & 0 & 0 \\
       0 & \rho & 0  \\
       \rho  e_{X_\alpha} & \rho e_{X_\beta} & \rho c_v \\
     \end{array}\right)
\end{equation}
where we use the following notation for compactness:
\begin{equation}
e_{X_k} = \dedXd
\end{equation}
and the specific heat at constant volume is
\begin{equation}
c_v = \left . \frac{\partial e}{\partial T} \right |_{\rho, X_k}
\end{equation}
We get the inverse
(computed via SymPy) as:
\begin{equation}
\renewcommand{\arraystretch}{1.5}
\frac{\partial {\bf w}}{\partial \Uc^\prime} = \left (
  \begin{array}{ccc}
   \frac{1}{\rho} & 0 & 0 \\
   0 & \frac{1}{\rho} & 0 \\
    -\frac{e_{X_\alpha}}{\rho c_v} & -\frac{e_{X_\beta}}{\rho c_v} & \frac{1}{\rho c_v} \\
   \end{array}\right)
\renewcommand{\arraystretch}{1}
\end{equation}

The reaction vector is
\begin{equation}
\Rb(\Uc^\prime) = \left (  \begin{array}{c} \rho \omegadot_\alpha \\ \rho \omegadot_\beta \\ \rho \Sdot \end{array} \right )
\end{equation}
We take all the quantities to be functions of $\rho$, $e$, and $X_k$,
but since $\rho$ doesn't change from the reactions (the reactive
source of the continuity equation is zero), it is held constant in
these derivatives.  The Jacobian is computed as $\partial \Rb/\partial
{\bf w}$:
\begin{equation}
\renewcommand{\arraystretch}{1.5}
\frac{\partial \Rb}{\partial {\bf w}} = \left (
  \begin{array}{ccc}
     \rho \frac{\partial \omegadot_\alpha}{\partial X_\alpha} &
     \rho \frac{\partial \omegadot_\alpha}{\partial X_\beta} & 
     \rho \frac{\partial \omegadot_\alpha}{\partial T} \\
      %
     \rho \frac{\partial \omegadot_\beta}{\partial X_\alpha} &
     \rho \frac{\partial \omegadot_\beta}{\partial X_\beta} &
     \rho \frac{\partial \omegadot_\beta}{\partial T} \\
     %
     \rho \frac{\partial \Sdot}{\partial X_\alpha} &
     \rho \frac{\partial \Sdot}{\partial X_\beta} &
     \rho \frac{\partial \Sdot}{\partial T} \\
  \end{array}
  \right )
\renewcommand{\arraystretch}{1}
\end{equation}

The final Jacobian is found by multiplying these two:
\begin{equation}
\renewcommand{\arraystretch}{1.5}
\frac{\partial \Rb}{\partial \Uc^\prime} = \left (
  \begin{array}{ccc}
    \frac{\partial \omegadot_\alpha}{\partial X_\alpha} - \frac{e_{X_\alpha}}{c_v} \frac{\partial \omegadot_\alpha}{\partial T} &
    \frac{\partial \omegadot_\alpha}{\partial X_\beta} - \frac{e_{X_\beta}}{c_v} \frac{\partial \omegadot_\alpha}{\partial T} &
    \frac{1}{c_v} \frac{\partial \omegadot_\alpha}{\partial T} \\
     %
    \frac{\partial \omegadot_\beta}{\partial X_\alpha} - \frac{e_{X_\alpha}}{c_v} \frac{\partial \omegadot_\beta}{\partial T} &
    \frac{\partial \omegadot_\beta}{\partial X_\beta} - \frac{e_{X_\beta}}{c_v} \frac{\partial \omegadot_\beta}{\partial T} &
    \frac{1}{c_v} \frac{\partial \omegadot_\beta}{\partial T} \\
     %
     \frac{\partial \Sdot}{\partial X_\alpha} -  \frac{e_{X_\alpha}}{c_v} \frac{\partial \Sdot}{\partial T} &
     \frac{\partial \Sdot}{\partial X_\beta} -  \frac{e_{X_\beta}}{c_v} \frac{\partial \Sdot}{\partial T} &
     \frac{1}{c_v} \frac{\partial \Sdot}{\partial T} \\
  \end{array}
  \right )
\renewcommand{\arraystretch}{1}
\end{equation}
We note that the form of these entries is the same as one would arrive at if you start with the rates expressed as
$\omega_k(\rho, T(\rho, X_j, e), X_j)$ and note that constant $e$ implies that
\begin{equation}
\left . \frac{\partial T}{\partial X_k} \right |_{\rho, e} = - \frac{e_{X_k}}{c_v}
\end{equation}

While VODE can compute the entire Jacobian, ${\bf J}$ numerically via
differencing, we found that does not give reliable results.  Instead,
we compute compute the derivatives with respect to $X_k$ and $T$
one-sided differencing, following the algorithm in \cite{lsode} to
minimize numerical noise.  We then use the equation of state to
compute $e_{X_k}$ and $c_v$ and construct the entries of the final
Jacobian.


%======================================================================
% References
%======================================================================

\bibliographystyle{aasjournal}
\bibliography{ws}

\end{document}
