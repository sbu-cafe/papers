\documentclass[times,modern]{aastex63}

% these lines seem necessary for pdflatex to get the paper size right
\pdfpagewidth 8.5in
\pdfpageheight 11.0in

\usepackage[T1]{fontenc}
\usepackage{epsf,color,amsmath}

\usepackage{cancel}
\usepackage{tcolorbox}

\newcommand{\sfrac}[2]{\mathchoice%
  {\kern0em\raise.5ex\hbox{\the\scriptfont0 #1}\kern-.15em/
    \kern-.15em\lower.25ex\hbox{\the\scriptfont0 #2}}
  {\kern0em\raise.5ex\hbox{\the\scriptfont0 #1}\kern-.15em/
    \kern-.15em\lower.25ex\hbox{\the\scriptfont0 #2}}
  {\kern0em\raise.5ex\hbox{\the\scriptscriptfont0 #1}\kern-.2em/
    \kern-.15em\lower.25ex\hbox{\the\scriptscriptfont0 #2}} {#1\!/#2}}


\newcommand{\castro}{{\sf Castro}}
\newcommand{\maestro}{{\sf Maestro}}
\newcommand{\flash}{{\sf Flash}}
\newcommand{\amrex}{{\sf AMReX}}

\newcommand{\isot}[2]{$^{#2}\mathrm{#1}$}
\newcommand{\isotm}[2]{{}^{#2}\mathrm{#1}}

\newcommand{\gcc}{\mathrm{g~cm^{-3} }}
\newcommand{\cms}{\mathrm{cm~s^{-1} }}

\newcommand{\nablab}{{\mathbf{\nabla}}}
\newcommand{\Ub}{\mathbf{U}}
\newcommand{\gb}{\mathbf{g}}
\newcommand{\omegadot}{\dot{\omega}}
\newcommand{\Sdot}{\dot{S}}
\newcommand{\ddx}[1]{{\frac{{\partial#1}}{\partial x}}}
\newcommand{\ddxs}[1]{{\frac{{\partial}}{\partial x}}#1}
\newcommand{\ddt}[1]{{\frac{{\partial#1}}{\partial t}}}
\newcommand{\odt}[1]{{\frac{{d#1}}{dt}}}
\newcommand{\divg}[1]{{\nablab \cdot \left (#1\right)}}
\newcommand{\dedr}{\left . {\partial{}e}/{\partial\rho}\right |_{T, X_k}}
\newcommand{\dedrd}{\left . \frac{\partial{}e}{\partial\rho}\right |_{T, X_k}}
\newcommand{\dedX}{\left . {\partial{}e}/{\partial{}X_k} \right |_{\rho, T}}
\newcommand{\dedXd}{\left . \frac{\partial{}e}{\partial{}X_k} \right |_{\rho, T, X_{j,j\ne k}}}
\newcommand{\dedT}{\left . {\partial{}e}/{\partial{}T} \right |_{\rho,X_k}}
\newcommand{\dedTd}{\left . \frac{\partial{}e}{\partial{}T} \right |_{\rho,X_k}}

\newcommand{\Ic}{{\boldsymbol{\mathcal{I}}}}
\newcommand{\Ics}{{\mathcal{I}}}
\newcommand{\smax}{{s_\mathrm{max}}}
\newcommand{\kth}{k_\mathrm{th}}
\usepackage{bm}

\newcommand{\Uc}{{\,\bm{\mathcal{U}}}}
\newcommand{\Fb}{\mathbf{F}}
\newcommand{\Sc}{\mathbf{S}}

\newcommand{\xv}{{(x)}}
\newcommand{\yv}{{(y)}}
\newcommand{\zv}{{(z)}}

\newcommand{\ex}{{\bf e}_x}
\newcommand{\ey}{{\bf e}_y}
\newcommand{\ez}{{\bf e}_z}

\newcommand{\Ab}{{\bf A}}
\newcommand{\Sq}{{\bf S}_\qb}
\newcommand{\Sqhydro}{{\Sq^{\mathrm{hydro}}}}
\newcommand{\qb}{{\bf q}}

\newcommand{\Shydro}{{{\bf H}}}
\newcommand{\Hb}{{\bf H}}
\newcommand{\Rb}{{\bf R}}
\newcommand{\Rq}{{\bf R}}
\newcommand{\Adv}[1]{{\left [\boldsymbol{\mathcal{A}} \left(#1\right)\right]}}
\newcommand{\Advt}[1]{{\left [\boldsymbol{\mathcal{\tilde{A}}} \left(#1\right)\right]}}
\newcommand{\Advss}[1]{{\left [{\mathcal{{A}}} \left(#1\right)\right]}}
\newcommand{\Advsst}[1]{{\left [{\mathcal{\tilde{A}}} \left(#1\right)\right]}}
\newcommand{\Advs}[1]{\boldsymbol{\mathcal{A}} \left(#1\right)}

\newcommand{\avg}[1]{{\left \langle #1 \right \rangle}}

\newcommand{\nse}[1]{{\mathtt{NSE}( #1 )}}

\newcommand{\rhonse}{{\rho_\mathrm{nse}}}
\newcommand{\tnse}{{T_\mathrm{nse}}}
\newcommand{\Anse}{{A_\mathrm{nse}}}
\newcommand{\Bnse}{{B_\mathrm{nse}}}
\newcommand{\Cnse}{{C_\mathrm{nse}}}

\newcommand{\out}{{\rm out}}
\newcommand{\inp}{{\rm in}}


\setlength{\marginparwidth}{0.75in}
\newcommand{\MarginPar}[1]{\marginpar{\vskip-\baselineskip\raggedright\tiny\sffamily\hrule\smallskip{\color{red}#1}\par\smallskip\hrule}}

\begin{document}
%======================================================================
% Title
%======================================================================
%\title{A Simplified Spectral Deferred Correction Method for Coupling Hydrodynamics with Reaction Networks and Nuclear Statistical Equilibrium}
\title{Notes on Extending Simplified SDC to Nuclear Statistical Equilibrium}


%======================================================================
% Abstract and Keywords
%======================================================================
\begin{abstract}
\end{abstract}

\keywords{hydrodynamics---methods: numerical}


\section{Reaction Networks and Nuclear Statistical Equilibrium}

The NSE table requires:
\begin{equation}
  \label{eq:aux:ye}
  Y_e = \sum_k \frac{Z_k X_k}{A_k}
\end{equation}
(where $A_k$ and $Z_k$ are the atomic weight and atomic number of nucleus $k$) and provides
\begin{align}
\label{eq:aux:abar}
\bar{A} &= \left [ \sum_k \frac{X_k}{A_k} \right ]^{-1} \\
\label{eq:aux:bea}
\left (\frac{B}{A} \right ) &= \sum_k \frac{B_k X_k}{A_k}
\end{align}
where $B_k$ is the binding energy of nucleus $k$.  In our simulations,
we store these 3 quantities as auxiliary data that is carried along
with the rest of the fluid state in each zone.  The table also returns
values of the mass fractions mapped onto the 19-isotopes we carry, $X_k$, and the time-derivative of $Y_e$, $dY_e/dt$.
We use the notation:
\begin{equation}
\nse{\rho,T,Y_e} \rightarrow \bar{A}, X_k, (B/A), dY_e/dt
\end{equation}
to represent the NSE table call and its inputs and outputs.

We can derive evolution equations for each of these composition quantities as:
\begin{align}
\frac{DY_e}{Dt} &= \sum_k \frac{Z_k}{A_k} \frac{DX_k}{Dt} = \sum_k \frac{Z_k}{A_k} \omegadot_k \\
\frac{D\bar{A}}{Dt} &= -\bar{A}^2 \sum_k \frac{1}{A_k} \frac{DX_k}{Dt} = -\bar{A}^2 \sum_k \frac{1}{A_k} \omegadot_k \\
\frac{D}{Dt} \left (\frac{B}{A} \right ) &= \sum_k \frac{B_k}{A_k} \frac{DX_k}{Dt} = \sum_k \frac{B_k}{A_k} \omegadot_k
\end{align}
For Strang split coupling of hydro and reactions, $DX_k/Dt = 0$,
therefore each of these auxillary equations obeys an advection
equation in the hydro part of the advancement.  In the SDC algorithm,
there will be a reactive source (an $\Ic_q$) for each of these that is
computed in the same manner as above.  We note that our NSE table
provides the evolution of $Y_e$ due to reactions ($DY_e/Dt$) directly.

The compositional quantities
it carries, $\bar{A}$ and $Y_e$ are not representable from the 19
isotopes we carry in the main network. For this reason, when we are
using the NSE network, we always provide these two composition quantities as
inputs to the EOS rather than using the $X_k$ directly.
the EOS directly from the auxiliary state in each zone instead of
using the $X_k$ directly.
Our equation of state needs the mean charge per nucleus, $\bar{Z}$, in addition
to the auxiliary quantities, which is computed as
\begin{equation}
\bar{Z} = \bar{A} \sum_k \frac{Z_k X_k}{A_k} = \bar{A} Y_e
\end{equation}
(see, e.g., \citealt{flash}).



\subsection{Strang-split algorithm for NSE}

For Strang splitting, we alternate the reaction and hydrodynamics,
with each operation working on the output of the previous operation.
If we define an advection operator over a timestep $\Delta t$ acting on $\Uc$ as
$\mathbb{A}_{\Delta t}(\Uc)$ and a reaction operator over a timestep of
$\Delta t/2$ acting on $\Uc$ as $\mathbb{R}_{\Delta t/2}(\Uc)$, then the advance appears
as:
\begin{align}
  \Uc^\star &= \mathbb{R}_{\Delta t/2}(\Uc^n) \\
  \Uc^{n+1,\star} &= \mathbb{A}_{\Delta t}(\Uc^\star) \\
  \Uc^{n+1} &= \mathbb{R}_{\Delta t/2}(\Uc^{n+1,\star})
\end{align}
Tthe hydrodynamic and reactive substeps over
the overall time-advancemenet scheme using the aprox19 + NSE network
are as follows:
\begin{itemize}

\item {\em Hydrodynamics update}
  
  At the beginning of each hydrodynamic update we have an input state,
  $\Uc_{\rm in}$ that we wish to integrate over $\Delta t$ to obtain $\Uc_{\rm out}$
  The hydrodynamics update proceeds as normal, but with an advection
  equation for each of the auxiliary composition variables:
  \begin{align}
    \ddt{(\rho Y_e)} + \ddx{(\rho Y_e u)} &= 0 \\
    \ddt{(\rho \bar{A})} + \ddx{(\rho \bar{A} u)} &= 0 \\
    \ddt{[\rho (B/A)]} + \ddx{[\rho (B/A) u]} &= 0 
  \end{align}


\item {\em Reactive update}

  At the beginning of each reactive update we have an input state,
  $\Uc_{\rm in}$ that we wish to integrate over $\Delta t/2$ to obtain $\Uc_{\rm out}$.
  
  \begin{itemize}

    \item For a zone that is in NSE:

      The goal is to update the composition and thermodynamics due to the 
      change in the nuclei abundances over the (half-)timestep.  \citet{ma:2013}
      uses a first-order in time difference to get the new composition state
      and evaluates the energy release from the change in binding energy in the NSE
      state.  They only apply a fraction (0.3) of the energy release to the internal
      energy in a zone, to avoid a potential instability that can arise if too much
      energy is added to a zone in a single timestep.  We prefer to deal with this
      issue through the retry mechanism already built into \castro.

      The approach we use begins with calling the NSE table with the input state:
      \begin{equation}
        \nse{\rho_\inp, T_\inp, (Y_e)_\inp} \rightarrow \bar{A}^\star, (X_k)^\star, (B/A)^\star, dY_e/dt
        \end{equation}
      We indicate with a $\star$ that most of those values are
      provisional, and we will seek a better value in the corrector step.
      We only update $Y_e$ from this call:
      \begin{equation}
        (Y_e)_\out = (Y_e)_\inp + \Delta t \frac{dY_e}{dt}
      \end{equation}
      and then we use the table again, but with the updated $Y_e$
      (note that $\rho_{\rm out}=\rho_{\rm in}$ in the Strang reaction formulation):
      \begin{equation}
        \nse{\rho_\out, T_\inp, (Y_e)_\out} \rightarrow \bar{A}_\out, (X_k)_\out, (B/A)_\out, dY_e/dt
      \end{equation}
      In this formulation, we have not updated the temperature. 
      This corrects the composition so that it is consistent with the updated
      $Y_e$.  We use these values $\bar{A}_\out$, $(X_k)_\out$, and $(B/A)_\out$ to then complete the update, computing the energy release, $\Sdot$ as:
      \begin{equation}
        \label{eq:nse_energy}
        \Sdot = \left [ \left ( \frac{B}{A} \right )_\out -
          \left ( \frac{B}{A} \right )_\inp \right ] N_A \frac{1}{\Delta t}
      \end{equation}

    \item For zones not in NSE:

    \begin{itemize}
    \item Integrate the full reaction network (Eqs.~\ref{eq:strang:e} and \ref{eq:strang:X}) as usual for Strang splitting
    \item Update the aux quantities at the end of the burn using Eqs.~\ref{eq:aux:ye}, \ref{eq:aux:abar}, and \ref{eq:aux:bea} with the new mass fractions, $X_k$.
    \end{itemize}
  \end{itemize}
\end{itemize}

\subsection{SDC-NSE Coupling Approach I}

This is the approach on the {\tt development} branch.

With SDC evolution, when we are in NSE, we need to do the advective
and reactive updates together.  We want to compute:
\begin{equation}
\Uc^{n+1,(k)} = \Uc^n + \Delta t \Adv{\Uc}^{n+1/2,(k)} + \Delta t \left [\Rb (\Uc) \right ]^{n+1/2,(k)}
\end{equation}
For density and momentum, we can do this update already, since there
are no reactive sources.  That gives us $\rho^{n+1,(k)}$ and $(\rho
\Ub)^{n+1,(k)}$.  For the other quantities, the NSE table gives only the
instantaneous values. aside from $Y_e$.  We therefore do an iterative
update.

Calling the NSE table on the starting state gives us:
\begin{equation}
\nse{\rho^n, T^n, (Y_e)^n} \rightarrow \bar{A}^n, (X_k)^n, (B/A)^n, (dY_e/dt)^n
\end{equation}
We now compute a first approximation to the reactive source:
\begin{align}
R^{(0)}(\rho e) &= 0 \\
R^{(0)}(\rho Y_e) &= \rho^n (dY_e/dt)^n \\
R^{(0)}(\rho \bar{A}) &= 0
\end{align}

Now we iterate $\xi = 0, \ldots N-1$, doing the following:

\begin{itemize}
\item Update the energy as:
   \begin{equation}
     (\rho e)^{n+1,(\xi)} = (\rho e)^n + \Delta t \Advss{\rho e}^{n+1/2,(k)} + \Delta t R^{(\xi)}(\rho e)
   \end{equation}

\item Update the auxiliary data as:
   \begin{equation}
     (\rho \alpha_l)^{n+1,(\xi)} = (\rho \alpha_l)^n + \Delta t \Advss{\rho \alpha_l}^{n+1/2,(k)} + \Delta t R^{(\xi)}(\rho \alpha_l)
   \end{equation}

\item Compute the updated temperature, $T^{n+1,(k)}$ as
  \begin{equation}
    T^{n+1,(\xi)} = T(\rho^{n+1,(k)},  \alpha^{n+1,(\xi)},  e^{n+1,(\xi)})
  \end{equation}
  where $\alpha^{n+1,(\xi)} = (\rho  \alpha)^{n+1,(\xi)} / \rho^{n+1,(k)}$ and $e^{n+1,(\xi)} = (\rho e)^{n+1,(\xi)} / \rho^{n+1,(k)}$.

\item Call the NSE table to get the new NSE state
  \begin{equation}
    \nse{\rho^{n+1,(k)}, T^{n+1,(\xi)}, (Y_e)^{n+1,(\xi)}} \rightarrow \bar{A}^{n+1,(\xi)}, (X_k)^{n+1,(\xi)}, (B/A)^{n+1,(\xi)}, (dY_e/dt)^{n+1,(\xi)}
  \end{equation}

\item Recompute the reactive source terms for the next iteration.

  The energy generation rate is found as:
  \begin{equation}
    (\rho \Sdot)^{n+1/2,(\xi)} = \frac{\rho^{n+1,(k)} + \rho^n}{2}
    \left [ \left ( \frac{B}{A} \right )^{n+1,(\xi)} -
      \left ( \frac{B}{A} \right )^n \right ] N_A \frac{1}{\Delta t}
  \end{equation}
  and our new auxiliary sources are:
\begin{align}
R^{(\xi+1)}(\rho e) &= (\rho \Sdot)^{n+1/2,(\xi)} \\
R^{(\xi+1)}(\rho Y_e) &= \frac{\rho^{n+1,(k)} + \rho^n}{2} \frac{1}{2} \left [ \left (\frac{dY_e}{dt} \right )^n + \left (\frac{dY_e}{dt} \right )^{n+1,(\xi)} \right ] \\
R^{(\xi+1)}(\rho \bar{A}) &= \frac{\rho^{n+1,(k)} + \rho^n}{2} \frac{1}{\Delta t} \left [ \bar{A}^{n+1,(\xi)} - \frac{(\rho \bar{A})^n}{\rho^n} \right ]
\end{align}

\end{itemize}

With the iteration complete, we can do the final update of the total energy using last iterations prediction of the source (for iteration $N$):
\begin{equation}
(\rho E)^{n+1,(k)} = (\rho E)^n + \Delta t \Advss{\rho E}^{n+1/2} + \Delta t (\rho \Sdot)^{n+1/2,(N)}
\end{equation}
and we take the values of the auxiliary state and mass fractions from the last call to the NSE table:
\begin{align}
(\rho Y_e)^{n+1,(k)} &= \rho^{n+1,(k)} (Y_e)^{n+1,(N-1)} \\
(\rho \bar{A})^{n+1,(k)} &= \rho^{n+1,(k)} (\bar{A})^{n+1,(N-1)} \\
(\rho (B/A))^{n+1,(k)} &= \rho^{n+1,(k)} (B/A)^{n+1,(N-1)} \\
(\rho Y_e)^{n+1,(k)} &= \rho^{n+1,(k)} (X_k)^{n+1,(N-1)}
\end{align}

We note that this update is not fully second order, however, the
treatment of $Y_e$ is, and we expect that to have the dominant effect
in the thermodynamic evolution through NSE.

\begin{tcolorbox}
{\bf Issues:} I think that when we call 
\begin{equation}
    \nse{\rho^{n+1,(k)}, T^{n+1,(\xi)}, (Y_e)^{n+1,(\xi)}} \rightarrow \bar{A}^{n+1,(\xi)}, (X_k)^{n+1,(\xi)}, (B/A)^{n+1,(\xi)}, (dY_e/dt)^{n+1,(\xi)}
\end{equation}

The inputs already have seen advection, so that means that the outputs know about advection.
So when we use $(B/A)^{n+1,(\xi)}$ to compute the energy release, are we also including the
change due to advection?
\end{tcolorbox}

\subsection{SDC-NSE Coupling Approach II}

This is on {\tt older\_sdc\_fix} branch, PR 1092.

It differs from above as in that it attempts to remove the advective
part from the energy generation rate, addressing the issue above.

Now we do compute a $(B/A)$ without advection:
\begin{equation}
   \widetilde{\rho \left ( \frac{B}{A} \right ) } = \left [ \rho \left ( \frac{B}{A} \right ) \right ]^{n+1,(\xi)} - \Advss{\rho (B/A)}^{n+1/2,(k)}
\end{equation}
and do the energy generation as:
\begin{equation}
   (\rho \Sdot)^{n+1/2,(\xi)} = 
     \left \{ \widetilde{\rho \left ( \frac{B}{A} \right )} -
       \left [ \rho \left ( \frac{B}{A} \right ) \right ]^n \right \} N_A \frac{1}{\Delta t}
\end{equation}
This will not include the advection.

\subsection{SDC-NSE Coupling Approach III}

This is the branch {\tt fix\_sdc\_nse} in PR 1090.

This follows the approach from the original SDC paper, for $\mathcal{I}_q$,  by first
predicting the update of $(B/A)$ with just advection, and using this to compute
the change in $(B/A)$ from just reactions.
\begin{equation}
\left [ \rho \left ( \frac{B}{A} \right ) \right ]^\star = 
   \left [ \rho \left ( \frac{B}{A} \right ) \right ]^n + \Delta t \Advss{\rho (B/A)}^{n+1/2}
\end{equation}
then we compute
\begin{equation}
\left (\frac{B}{A} \right )^\star = \left [ \rho \left ( \frac{B}{A} \right ) \right ]^\star  / \rho^{n+1}
\end{equation}
Then the energy generation is:
\begin{equation}
   (\rho \Sdot)^{n+1/2,(\xi)} =  \rho^{n+1/2}
     \left [ \left ( \frac{B}{A} \right )^{n+1} -
              \left ( \frac{B}{A} \right )^\star \right ] N_A \frac{1}{\Delta t}
\end{equation}

\subsection{Thoughts}

\begin{itemize}

\item Imagine that we evolve in a zone such that the NSE state does
  not change, but density does (this would mean that the temperature
  must have changed).  In that case the energy generation should be
  zero, so writing is like:
  \begin{equation}
    \rho \Sdot = \rho^{n+1/2} \left [ \left ( \frac{B}{A} \right )^{n+1} - \left ( \frac{B}{A} \right )^n \right ] \frac{N_A}{\Delta t}
  \end{equation}
  captures that.

\item Can we derive an energy equation that directly includes $(B/A)$ in the definition
  of $E$?  (This is what the combustion codes do).

\item What if we just try a first-order accurate update of density?

\item What if we just do first order accurate in $Y_e$?

\end{itemize}

%======================================================================
% References
%======================================================================

\bibliographystyle{aasjournal}
\bibliography{ws}

\end{document}
