%\documentclass[12pt, preprint]{aastex}
\documentclass[preprint,times,tighten]{aastex63}

\usepackage{epsf,color,amsmath}

\usepackage{cancel}

\newcommand{\sfrac}[2]{\mathchoice%
  {\kern0em\raise.5ex\hbox{\the\scriptfont0 #1}\kern-.15em/
    \kern-.15em\lower.25ex\hbox{\the\scriptfont0 #2}}
  {\kern0em\raise.5ex\hbox{\the\scriptfont0 #1}\kern-.15em/
    \kern-.15em\lower.25ex\hbox{\the\scriptfont0 #2}}
  {\kern0em\raise.5ex\hbox{\the\scriptscriptfont0 #1}\kern-.2em/
    \kern-.15em\lower.25ex\hbox{\the\scriptscriptfont0 #2}} {#1\!/#2}}

\newcommand{\myhalf}{\sfrac{1}{2}}
\newcommand{\nph}{{n+\myhalf}}
\newcommand{\nmh}{{n-\myhalf}}

\newcommand{\inp}{\mathrm{in}}
\newcommand{\outp}{\mathrm{out}}

% boldsymbol means bold italic
\newcommand{\eb}{{\bf{e}}}
\newcommand{\Ub}{{\bf{U}}}
\newcommand{\xb}{{\bf{x}}}
\newcommand{\kb}{{\bf{k}}}
\newcommand{\Vb}{{\bf{V}}_n}
\newcommand{\Vbhat}{{\bf{\widehat{V}}}_n}
\newcommand{\Omegab}{{\bf{\Omega}}}
\newcommand{\gb}{{\bf{g}}}
\newcommand{\rb}{{\bf{r}}}

\newcommand{\pb}{p_\mathrm{base}}
\newcommand{\epsdot}{\dot{\epsilon}}
\newcommand{\qburn}{q_\mathrm{burn}}
\newcommand{\rt}{\tilde{r}_0}


\newcommand{\nablab}{\mathbf{\nabla}}
\newcommand{\dt}{\Delta\ t}

\newcommand{\omegadot}{\dot{\omega}}

\newcommand{\Hext}{H_{\rm ext}}
\newcommand{\Hnuc}{H_{\rm nuc}}
\newcommand{\kth}{k_{\rm th}}

\newcommand{\Gammaonebar}{\overline{\Gamma}_1}
\newcommand{\Sbar}{\overline{S}}

\newcommand{\etarho}{\eta_\rho}
\newcommand{\etarhoh}{\eta_{\rho~h}}

\newcommand{\Ubt}{\widetilde{\Ub}}
\newcommand{\wt}{\widetilde{w}}

\newcommand{\He}{$^4$He}
\newcommand{\C}{$^{12}$C}
\newcommand{\Fe}{$^{56}$Fe}

\newcommand{\isot}[2]{$^{#2}\mathrm{#1}$}
\newcommand{\isotm}[2]{{}^{#2}\mathrm{#1}}

\newcommand{\maestro}{{\sf MAESTRO}}
\newcommand{\castro}{{\sf Castro}}
\newcommand{\amrex}{{\sf AMReX}}
\newcommand{\pynucastro}{{\sf pynucastro}}

\newcommand{\avg}[1]{\overline{#1}}
\newcommand{\avgtwod}[1]{\langle~#1 \rangle}
\newcommand{\rms}[2]{\left(\delta#1\right)_{r_{#2}}}
\newcommand{\mymax}[1]{\left(#1\right)_{\rm max}}

\newcommand{\Tb}{\ensuremath{T_\mathrm{base}}}
\newcommand{\gcc}{\mathrm{g~cm^{-3} }}
\newcommand{\cms}{\mathrm{cm~s^{-1} }}

\newcommand{\half}{\frac{1}{2}}

\setlength{\marginparwidth}{0.5in}

\newcommand{\MarginPar}[1]{
    \marginpar{\vskip-\baselineskip%
               \raggedright%
               \tiny\sffamily%
               {\color{red}\hrule%
               \smallskip%
               #1\par%
               \smallskip%
               \hrule}}%
}

\newcommand{\AssignTo}[1]{
    \marginpar{\vskip-\baselineskip%
               \raggedright%
               \tiny\sffamily%
               {\color{blue}\hrule%
               \smallskip%
               #1\par%
               \smallskip%
               \hrule}}%
}

\begin{document}
%======================================================================
% Title
%======================================================================
\title{Dynamics of Laterally Propagating Flames in X-ray Bursts}

\shorttitle{Lateral Flame Dynamics}
\shortauthors{Eiden et al.}

\author[0000-0001-6191-4285]{Kiran Eiden}
\affiliation{Dept.\ of Physics and Astronomy, Stony Brook University,
             Stony Brook, NY 11794-3800}

\author[0000-0001-8401-030X]{Michael Zingale}
\affiliation{Dept.\ of Physics and Astronomy, Stony Brook University,
             Stony Brook, NY 11794-3800}

\author[0000-0002-1530-781X]{Alice Harpole}
\affiliation{Dept.\ of Physics and Astronomy, Stony Brook University,
             Stony Brook, NY 11794-3800}

\author[0000-0003-2300-5165]{Donald Willcox}
\affiliation{Lawrence Berkeley National Laboratory, Berkeley, CA}

\author{Yuri Cavecchi}
\affiliation{Mathematical Sciences and STAG Research Centre, University of Southampton, SO17 1BJ}

\author[0000-0003-0439-4556]{Max P.\ Katz}
\affiliation{NVIDIA Corp}

\author[0000-0001-8092-1974]{Weiqun Zhang}
\affiliation{Lawrence Berkeley National Laboratory, Berkeley, CA}


\correspondingauthor{Michael Zingale}
\email{Michael.Zingale@stonybrook.edu}


%======================================================================
% Abstract and Keywords
%======================================================================
\begin{abstract}
We investigate the structure of laterally-propagating flames through
the highly-stratified burning layer in an X-ray burst.  Simple
two-dimensional simulations of flame propagation are performed through
a rotating plane-parallel atmosphere, exploring the structure of the
flame.  We discuss the approximations needed to capture the length and
time scales at play in an X-ray burst.  Our studies complement the
other multidimensional studies of burning in X-ray bursts.
\end{abstract}

\keywords{convection---hydrodynamics---methods: numerical---stars: neutron---X-rays: bursts}

%======================================================================
% Introduction
%======================================================================
\section{Introduction}\label{Sec:Introduction}

X-ray bursts (XRBs) are thermonuclear explosions of an accreted H or
He layer on the surface of a neutron star (see~\citealt{galloway:2017}
for a review).  Observations of bursts can help constrain the
properties of the underlying neutron star, helping to illuminate the
nuclear equation of state~\citep{steiner:2010,ozel2016masses}.  Extensive
observations of brightness oscillations during the rise have provided
evidence that the burning begins in a localized region and spreads
over the surface of the neutron
star~\citep{bhattacharyya:2006,bhattacharyya:2007,chakraborty:2014}.

One dimensional studies of XRBs are very successful in predicting the
lightcurves and recurrance times (see, e.g., \citealt{woosley-xrb}).
These assume spherical symmetry and thus cannot capture the effects of
localized burning spreading across the neutron star.  These studies
have also been used to explore the sensitivity of the burst
observables to accretion and reaction
rates~\citep{cyburt:2010,Jose2010a,Lampe2016}, and model individual
bursts~\citep{johnston:2019}.

Multidimensional simulations of burning on a neutron star are more
difficult, with both the temporal and spatial scales presenting
challenges (see \citealt{astronum_2018} for an overview).  For the spatial scales, we need to
resolve the reaction zone, $\mathcal{O}(10~\mbox{cm})$ or smaller, the
scale height of the atmosphere, $\mathcal{O}(500~\mbox{cm})$, and the
Rossby scale where the Coriolis force balances the lateral pressure
gradient, $\mathcal{O}(10^5~\mbox{cm})$~\citep{spitkovsky2002}.  For
the temporal scales, capturing the rise, $\mathcal{O}(1~\mbox{s})$,
and the decay of the burst, $\mathcal{O}(10~\mbox{s})$, and the
accretion period between bursts, $\mathcal{O}(10^4~\mbox{s})$ is
beyond the ability of multidimensional hydro codes currently.
Nevertheless, significant progress has been made in understanding the
multidimensional nature of XRBs, through various approximations.x

Laterally propagating detonations were modeled by
\citet{fryxellwoosley82,hedet}, but these would only be applicable at
very high densities, since it is difficult to detonate helium at the
densities found in normal XRBs, and even harder to detonate hydrogen
because of the waiting times for weak reactions.  This means that
detonations may only really be applicable to superbursts (where carbon
is the reactant)~\citep{Weinberg2006b,Weinberg2007}.

Global multidimensional studies were performed by
\citet{spitkovsky2002}, where it was demonstrated that the Coriolis
force plays an important role in confining the burning as it spreads
across the neutron star surface.  These calculations used the
shallow-water approximation, so the vertical details of the
atmosphere's structure we not modeled.  The model showed that the
horizontal pressure gradient between the ash and fuel can be important
in accelerating the burning front.

Small-domain studies of convective burning in XRBs preceding flame
development have been done in two-dimensions \citep{lin:2006,xrb,xrb2}
and three-dimensions \citep{xrb3d}.  These calculations used low Mach
number methods that approximate the hydrodynamics equations to filter
soundwaves, enabling large timesteps and efficient modeling of
subsonic convection.  While these calculations could not support
the lateral differences needed for flame spreading, they can help
understand the role that convection plays in distributing the initial
burning products vertically throughout the neutron star atmosphere as well
as the nature of any turbulence the burning front might encounter.



\cite{cavecchi:2013,art-2015-cavecchi-etal,art-2016-cavecchi-etal} \MarginPar{need a summary}

For deflagrations, we either need to resolve the structure of the
reaction zone or use a flame model.  Flame models usually assume that
the flame structure is thin compared to the size of the system (see,
e.g., \citet{Ropke2007} for applications to Type Ia supernovae).  For
XRBs however, the flame thickness is comparable to the scale height of
the atmosphere, so we cannot use these approximations.  Accurate
models of flames therefore require that we resolve the thermal width,
which is $\mathcal{O}(10~\mbox{cm})$ for helium flames~\citep{Timmes00}.

The goal of this study is to understand what numerical and physical
approximations are needed to perform a full hydrodynamical,
multidimensional simulation of flame propagation through the
atmosphere on a neutron star.  For simplicity in this first set of
calculations, we will use pure helium composition.  These studies
complement the other multidimensional work described above in helping
us to build a picture of the dynamics of X-ray bursts.


%======================================================================
% Numerical Approach
%======================================================================
\section{Numerical Approach}\label{Sec:numerics}

All simulations are performed with the \castro\ hydrodynamics
code~(\citealt{castro}; see also \citealt{astronum:2017} for a recent
description).  We evolve the system of fully compressible Euler
equations for reacting flow:
\begin{eqnarray}
\frac{\partial( \rho X_k)}{\partial t} &=& - \nabla\cdot (\rho \Ub X_k) + \rho \omegadot_k \\
%
\frac{\partial (\rho \Ub)}{\partial t} &=& - \nabla\cdot (\rho \Ub \Ub) - \nabla p +
    \rho \gb \nonumber \\
  &&-2 \rho \Omegab\times \Ub - \Omegab \times (\Omegab \times \rb) \\
%
\frac{\partial (\rho E)}{\partial t} &=& - \nabla \cdot (\rho \Ub E + p\Ub) +
    \nabla \cdot \kth \nabla T + \rho \epsdot +\nonumber\\
  && \rho \Ub \cdot \gb - \rho (\Omegab \cdot \rb)(\Omegab \cdot \Ub) + \rho |\Omegab|^2 (\Ub \cdot \rb)
\end{eqnarray}
Here, $\rho$ is the mass density, $\Ub$ is the velocity, and $E$ is
the specific total energy, which is related to the specific internal
energy as $e = E - |\Ub|^2/2$.  Species are described by mass
fractions, $X_k$ (such that $\sum_k X_k = 1$), and creation rates,
$\omegadot$, and are related to the total specific energy generation
rate, $\epsdot$.  The total mass conservation,
\begin{equation}
\frac{\partial \rho}{\partial t} = - \nabla\cdot (\rho \Ub)
\end{equation}
implies $\sum_k \omegadot_k = 0$.  \castro\ uses an unsplit piecewise
parabolic method (PPM) with characteristic tracing for solving the
hydrodynamics~\citep{ppm,millercolella:2002}, generalized to an
arbitrary equation of state~\citep{zingalekatz}.  Reactions are
incorporated via Strang splitting~\citep{strang:1968}, giving a method
that is overall second-order accurate in space and time.
\castro\ uses the \amrex\ adaptive mesh refinement
library~\citep{amrex_joss} to manage a hierarchy of grids at different
resolutions.

An equation of state of the form $p = p(\rho, e, X_k)$ completes the
thermodynamic description of the system.  Finally, $\kth$ is the
thermal conductivity and $T$ is the temperature.  Diffusion is treated
explicitly in time.

Since the neutron star rotates, we work in a corotating frame, with
angular velocity $\Omegab$.  Further, for the two-dimensional
simulations presented here, we work in axisymmetric coordinates,
but we advect a third component of velocity, coming out of the
simulation plane, that participates in the Coriolis force (sometimes
described as a 2.5D simulation).  We will take the
\castro\ $x$-coordinate to be the cylindrical radial coordinate with
corresponding velocity $u$, the \castro\ $y$-coordinate to be the
cylindrical vertical coordinate with corresponding velocity $v$, and the
\castro\ $z$-coordinate to be the cylindrical azimuthal coordinate,
with corresponding velocity $w$.  A righthanded coordinate system has
positive $w$ pointing out of the page.  We will take $\Omegab =
\Omega_0 \hat{\bf y}$ for the angular rotation rate, and $\gb = -g
\hat{\bf y}$ for the gravitational acceleration, with $g$ constant.
With these choices, the Coriolis force is:
\begin{equation}
-2\rho \Omegab \times \Ub =
   -2\rho \left ( \Omega_0 w \hat{x} - \Omega_0 u \hat{z} \right )
\end{equation}
We will neglect the centrifugal force---with our plane-parallel
geometry, this will act only in the lateral direction, and is not
expected to greatly affect the dynamics.  Carrying the Coriolis force
allows us to capture the geostrophic balance\MarginPar{ref, Spitkovsky 2002?} that sets
up via lateral hydrostatic equilibrium.  In the discussions below, we will use the Castro coordinate
names, $(x, y)$ in our notation.


Writing the momentum equation in terms of the $u$, $v$, and $w$
components, and neglecting the centrifugal force, we have:
\begin{align}
\frac{\partial (\rho u)}{\partial t} + \nabla \cdot (\rho u \Ub) +
     \frac{\partial p}{\partial x} &= -2\rho \Omega_0 w  \\
%
\frac{\partial (\rho v)}{\partial t} + \nabla \cdot (\rho v \Ub) +
     \frac{\partial p}{\partial y} &= -\rho g \\
%
\frac{\partial (\rho w)}{\partial t} + \nabla \cdot (\rho w \Ub) +
     \cancelto{0}{\frac{\partial p}{\partial z}} &=
    2\rho \Omega_0 u
\end{align}
where we cancel $\partial p/\partial z$ because there are no
variations in the azimuthal direction.  This allows us to recast the
$w$-velocity equation as a simple advection equation:
\begin{equation}
\frac{\partial w}{\partial t} + \Ub \cdot \nabla w = 2\Omega_0 u
\end{equation}
In our geometry, the flame will propagate from left to right, so $u$
will be positive and the Coriolis force results in $w > 0$ (out of the
simulation plane).  The algorithmic implementation of rotation in
\castro\ is described in \cite{wdmergerI}.

We use a general stellar equation of state with nuclei (treated as an
ideal gas), photons, and degenerate/relativistic electrons, as
described in \cite{timmes_swesty:2000}.  For reactions, we use a 13
isotope alpha chain, derived from the {\tt aprox13}
network~\citep{timmes_aprox13}.  For one of our runs, we use the
smaller 7-isotope network described in \citet{iso7}.  We integrate the
network using the VODE integration package~\citep{vode}, and our
implementation is provided in the StarKiller Microphysics
source~\citep{starkiller}.

We note that we do not explicitly model viscosity.  The reactions will
provide the smallscale cutoff to the instabilities and turbulence at
the flame front.  We also do not include species
diffusion---astrophysical flames tend to have large Lewis numbers, so
this is not expected to be important~\citep{timmeswoosley:1992}.
Finally, we use the thermal conductivities described in
\citet{Timmes00}.


All simulations use adaptive mesh refinement to refine on the
atmosphere (leaving the space between the top of the atmosphere and
upper boundary at low resolution).  In addition to the base grid, we
use up to 3 refinement levels, the first one a factor of 4 finer than
the previous and the remaining each a factor of 2 finer than the
previous.  \castro\ subcycles in time, so the finer grids are evolved
with a finer timestep than the coarse grids.  Occasionally the
timestep chosen at the start of a cycle will violate the CFL condition
during the advance of the finer grids.  In this case, we restart the
finer grid evolution with a smaller timestep, subcycling within the
larger timestep hierarchy.  We use a CFL number of 0.8 for our
simulations.  
% \MarginPar{true for all?} 
The base grid for our standard
simulations is $768\times 192$ zones and the corresponding finest grid
when 3 refinement levels are added would be $12288\times 3072$ zones.
Our standard domain is $1.2288\times 10^5~\mathrm{cm} \times
3.072\times 10^4~\mathrm{cm}$, so we have $10~\mathrm{cm}$ resolution
on the finest grid.  We only refine the fuel layer in the left half of
the domain at the highest resolution (and only down to densities of
$2.5\times 10^4~\gcc$), since this is where we expect the flame to
propagate.  At the start of the simulation, $3.4\%$ of the domain is
at the finest resolution.  This increases to $7.4\%$ by the end of the
simulation, because of the increase in the scale height of the atmosphere
behind the flame.

Thermal diffusion is modeled explicitly, using a predictor-corrector
scheme to achieve second-order accuracy.  A verification test of the
diffusion scheme is shown in Appendix~\ref{app:diffusion}.  The
explicit thermal diffusion requires a timestep limiter of the form:
\begin{equation}
\Delta t_\mathrm{diff} \le \frac{1}{2} \frac{\Delta x^2}{\mathcal{D}}
\end{equation}
where $\mathcal{D} = \kth/(\rho c_v)$ is the thermal diffusivity.  The
diffusivity increases rapidly at the top of the atmosphere, making
these low density regions determine the overall timestep for the
simulations.  Therefore, we disable thermal conduction at low
densities where it is not expected to be important.

We use hydrostatic boundary conditions on the lower boundary, using
a discretized hydrostatic equilibrium equation of the form:
\begin{equation}
\label{eq:hse}
p_i = p_{i-1} + \frac{1}{2} \Delta y (\rho_i + \rho_{i-1}) g
\end{equation}
holding the temperature constant in the ghost cells.  This is solved
together with the equation of state.  The velocity is reflected here.
This procedure follows the form described in \citet{ppm-hse}.  The
left boundary is reflecting and the right boundary is a zero-gradient
outflow.  The top boundary sets the state to simply the conditions
in our outer buffer region of the initial model (see below) with the
normal velocity set to the larger of zero or the velocity at the top
of the domain (this prevents incoming velocities at the top) and the
transverse velocities set to zero.

When we begin the simulation, there is a transient phase as the flame
gets established.  Material that is forced upward will encounter the
steep density gradient at the top of the atmosphere and accelerate as
it is blown out of the atmosphere.  Eventually this material will fall
back to the top of the atmosphere.  In this paper, we are mostly
concerned with the behavior of the flame and not any material that is
violently blown out of the top of the atmosphere, so we apply a sponge
to this region.  This is similar to what we've done in \cite{xrb3d}.
The sponge drives the velocity of the material in the low
density \MarginPar{add mathematical form of sponge} regions at the top
of the atmosphere to zero.  This sponging helps increase our timestep
as well.




%======================================================================
% Flame
%======================================================================
\section{Flame Properties}\label{Sec:Flame}

The speed and thickness of a laminar helium flame are determined by the
energy generation rate and conductivity, and scale roughly as 
\begin{equation}
v_f \approx \sqrt{k_\mathrm{th} \dot{\epsilon}} \qquad
\lambda_f \approx \sqrt{\frac{k_\mathrm{th}}{\dot{\epsilon}}}
\end{equation}
\citep{orourke:1979,khokhlov:1993}.
At the densities we consider in this simulation, the pure He flame
speed is quite slow and would require a prohibitively long integration
time to see significant evolution of the burning.  We therefore
consider boosted flames in this first paper, to accelerate the burning
and allow us to understand the qualitative effects of laterally
propagating flames.  To boost the flame, while keeping the thickness
the same, we can multiply both the burning rate and the conductivity
by the same factor.  For our standard calculations, we choose 10 for
each, to give a flame 10$\times$ faster.  We call this the ``10/10''
flame.  We will also do a simulation with the reactions and
conductivity both boosted by 5, the ``5/5'' flame.

To understand the time and length scales involved in flame
propagation, we do a 1D simulation of a laminar flame using our
microphysics.  Figure~\ref{fig:flame} shows the flame properties for a 
% \AssignTo{Kiran and Mike will remake with the 10/10 boosted flame}
He flame using the standard conductivities and our simple reaction
network.  This flame had a density of $2\times 10^6~\gcc$ and 
% \MarginPar{double check}
temperature of $5\times 10^7$~K.  We note that this flame speed is
about $2\times 10^4~\cms$, the flame width is about 20~cm, and it
takes about 0.1~s to settle into a sustained flame.  Note this speed
is quite small compared to the speed estimated in
\citet{spitkovsky2002}.  We expected a multidimensional flame to
accelerate due to hydrodynamics interactions (wrinkling, turbulence
interactions, directed flows feeding fuel into the flame, ...).


\begin{figure*}[t]
\plottwo{flame}{speed}
\caption{\label{fig:flame} Our ``standard'' flame.}
\end{figure*}

We measure the flame width as:
\begin{equation}
\lambda_f \equiv \frac{\Delta T}{\max\{|\nabla T|\}}
\end{equation}
Experience with modeling resolved flames suggests that we need a
spatial resolution, $\Delta x$ of $\lambda_f/\Delta x \sim 5$~\citep{SNld}.  These
conditions represent the bottom of the He layer.  As the density
decreases with altitude, the flame thickness increases and the flame
speed decreases, so we will easily resolve the flame structure throughout
the rest of the atmosphere.

Flame properties are show in Table XX \MarginPar{we'll redo the table with 10/10 and 5/5/ and unboosted flame}

\iffalse

\begin{table}

  \caption{Propagation Rate ($10^5$ cm/s)}
  
  \begin{center}
    \begin{tabular}{c||c|c|c|c}
      
      & 10 & 15 & 20 & 40 \\
      \hline\hline
      10/10 & 1.1 & 1.1 & 1.1 & 1.3 \\
      \hline
      20/5 & 1.0 & 1.0 & 1.0 & 1.1 \\
      \hline
      25/4 & 0.97 & 0.97 & 0.98 & \\
      \hline
      $\frac{1}{2}\rho$ & 0.45 & 0.43 & & \\
      
    \end{tabular}
  \end{center}
  
\end{table}

\begin{table}
  
  \caption{Flame Width (cm)}
  
  \begin{center}
    \begin{tabular}{c||c|c|c|c}
      
      & 10 & 15 & 20 & 40 \\
      \hline\hline
      10/10 & 44 & 57 & 70 & 119 \\
      \hline
      20/5 & 56 & 72 & 88 & 140 \\
      \hline
      25/4 & 62 & 78 & 94 & \\
      \hline
      $\frac{1}{2}\rho$ & 89 & 109 & & \\
      
    \end{tabular}
  \end{center}
  
\end{table}

\fi

%======================================================================
% Initial Model
%======================================================================
\section{Initial Model}\label{Sec:inital_model}

% \begin{figure}[t]
% \centering
% \plotone{model}
% \caption{\label{fig:sketch} A schematic of our model setup.}
% \end{figure}

We wish to create an initial atmosphere consisting of a hot
``post-flame'' region and a cooler atmosphere that the flame will
laterally propagate into.  We put the hot region at the very left of
the domain (the origin of the axisymmetric coordinates).  To create
these initial conditions, we produce two different hydrostatic models,
a ``hot'' model that will represent the perturbation that drives the
flame and a ``cool'' model that will represent the state ahead of the
flame.  These will have different scale heights.  To create these
models, we break the vertical structure of the atmosphere into 4
layers: the underlying star, a ramp-up to the base of the
accreted atmosphere, a fuel layer representing the bulk of the
atmosphere where the flame will propagate, and an outer, low density,
isothermal buffer above the atmosphere that allows us
to capture expansion and explosive dynamics.  
% This is sketched in
% Figure~\ref{fig:sketch}.

The temperature profile in the star and ramp region is given as:
\begin{equation}
T = T_\star + \frac{1}{2} (T_\mathrm{hi} - T_\star) \left [ 1 + \tanh\left( \frac{\tilde{y}}{2 \delta_\mathrm{atm}} \right ) \right ]
\end{equation}
with
\begin{equation}
\tilde{y} = y - H_\star - \frac{3}{2} \delta_\mathrm{atm}
\end{equation}
Here, $\delta_\mathrm{atm}$ is a characteristic width of the transition
ramp, $T_\mathrm{hi}$ is the highest temperature in the HSE model---it
will represent the base of the fuel layer.

The mass fractions use this same profile, switching from a set
describing the underlying star, ${X_k}_\star$, and the set for the
accreted material, ${X_k}_\mathrm{atm}$, which is used in the
isentropic and outer regions.  Note, since the profile above is
linear in $X$, if the initial mass fractions sum to one, then the blended
mass fractions in the ramp region also sum to one.

We specify the density, $\rho_\mathrm{int} = \rho(y = H_\star)$ as the
starting point for the integration of hydrostatic equilibrium.  This
is just below the ramp-up region---this ensures that regardless of
what the peak temperature ($T_\mathrm{hi}$) is, the state beneath the
ramp-up region remains unchanged.  Therefore, we will still be in
lateral equilibrium in the star region.  We will denote the density
where $T = T_\mathrm{hi}$ as $\rho_\mathrm{fuel}$.


Creating the model involves specifying $T_\star$, $T_\mathrm{hi}$,
$T_\mathrm{lo}$, $\rho_\mathrm{int}$, $H_\star$,
$\delta_\mathrm{atm}$, ${X_k}_\star$, and ${X_k}_\mathrm{atm}$.  We
then integrate outwards from the base of the ramp region ($y =
H_\star$), enforcing the discrete form of hydrostatic equilibrium,
Eq.~\ref{eq:hse}.  Integrating upwards, we would find $p_i$ and
$\rho_i$ using a Newton solve together either with the temperature
specified, $T_i = T(p_i, \rho_i, \{X_k\}_i)$ (in the isothermal, ramp,
and buffer layers) or constant entropy, $s_i = s(p_i, \rho_i,
\{X_k\}_i)$, in the fuel layer.  In each case, using the equation of
state.  This follows the procedures described in \citet{ppm-hse}.  We
use the constant temperature for all $y < H_\star +
3\delta_\mathrm{atm}$.  Above this, we switch to isentropic until the
temperature drops to a floor value, $T_\mathrm{lo}$, at which point we
again keep the temperature constant.  The integration of the
atmosphere continues until the density falls to a low density cutoff,
$\rho_\mathrm{cutoff}$.  The material above this height is taken to
have constant density and temperature.

\begin{deluxetable}{lcc}
\tablecaption{\label{table:params} Initial model parameters.}
\tablehead{\colhead{parameter} & \colhead{cool} & \colhead{hot}}
\startdata
$T_\star$       & \multicolumn{2}{c}{$10^8$~K} \\
$T_\mathrm{hi}$ & $2\times 10^8$~K & $1.4\times 10^9$~K \\
$T_\mathrm{lo}$ & \multicolumn{2}{c}{$8\times 10^6$~K} \\
$\rho_\mathrm{int}$ & \multicolumn{2}{c}{$3.43\times 10^6~\gcc$} \\
$\rho_\mathrm{fuel}$\tablenotemark{a} & $2.36\times 10^6~\gcc$ & $1.20\times 10^6~\gcc$ \\
$\rho_\mathrm{cutoff}$ & \multicolumn{2}{c}{$10^{-4}~\gcc$} \\
$g$             & \multicolumn{2}{c}{$-1.5\times 10^{14}~\mathrm{cm~s^{-2}}$} \\
$H_\star$       & \multicolumn{2}{c}{2000~cm} \\
$\delta_\mathrm{atm}$ & \multicolumn{2}{c}{50~cm} \\
$X_\star(\isotm{Ni}{56})\tablenotemark{b}$ & \multicolumn{2}{c}{1.0} \\
$X_\mathrm{atm}(\isotm{He}{4})\tablenotemark{b}$ & \multicolumn{2}{c}{1.0} \\
\enddata
%
\tablenotetext{a}{This is not an input parameter, but instead is
  computed during integration.  We list it here for reference.}
%
\tablenotetext{b}{All other species are taken as 0.}
\end{deluxetable}

The choice of factors in front of $\delta_\mathrm{atm}$ were designed
to make sure the peak $T$ is attained at the desired density
of the burning layer.

We create two models, a ``cool'' model representing the atmosphere
ahead of the flame and a ``hot'' (or perturbed) model that will
initiate the flame by driving the burning.  The parameters we use for
the model generation are listed in Table~\ref{table:params}.

We blend the hot and cold models laterally to produce the perturbation
needed to initiate a localized flame, with the hot model at the
origin of the axisymmetric geometry.  The blending is done as:
\begin{align}
p(x,y) &= f(x) p_\mathrm{hot}(y) + [1-f(x)] p_\mathrm{cool}(y) \\
\rho(x,y) &= f(x) \rho_\mathrm{hot}(y) + [1-f(x)] \rho_\mathrm{cool}(y) \\
X_k(x,y) &= f(x) {X_k}_\mathrm{hot}(y) + [1-f(x)] {X_k}_\mathrm{cool}(y)
\end{align}
with
\begin{equation}
f(x) = \begin{cases}
     1 & x < x_\mathrm{pert} \\
   1 - \frac{x - x_\mathrm{pert}}{\delta_\mathrm{blend}} & x_\mathrm{pert} \le x \le x_\mathrm{pert} + \delta_\mathrm{blend} \\
     0 & x > x_\mathrm{pert} + \delta_\mathrm{blend}
\end{cases}
\end{equation}
Since the equation of hydrostatic equilibrium is linear and our
blending is a linear combination of two models in hydrostatic
equilibrum, the blended model is also in vertical equilibrium initially.  We choose
$x_\mathrm{pert} = 1.024\times 10^4$~cm and $\delta_\mathrm{blend} = 2048$~cm.  Once
the blended model is constructed, we compute $T(x,y)$ and $(\rho e)(x,y)$
from the equation of state,
\begin{align}
  T(x,y) &= T(\rho(x,y), p(x,y), X_k(x,y)) \\
  (\rho e)(x,y) &= \rho(x, y) \cdot e(\rho(x,y), p(x,y), X_k(x,y)) 
\end{align}

\begin{figure}[t]
\centering
\epsscale{0.75}
\plotone{initial_model_paper}
\caption{\label{fig:initial_models} Our ``cool'' (solid) and ``hot''
  (dashed) initial models, showing both the density and temperature.}
\epsscale{1.0}
\end{figure}



%======================================================================
% Results
%======================================================================
\section{Simulations and Results}\label{Sec:results}

Table~\ref{table:sim_names} summarizes the simulations that we ran.
The majority of them used a reaction rate boosting of $10$ and a
conductivity boosting of $10$, which should increase the flame speed
by a factor of $10$.  Most simulations used a resolution of
$10~\mbox{cm}$ and a domain width of a little more than one kilometer.  
We use an artifically high rotation rate of
$2000~\mbox{Hz}$, which gives a Rossby length of
\begin{equation}
L_R = \frac{\sqrt{g H_0}}{\Omega} \sim 3\times 10^4~\mathrm{cm}
\end{equation}
using a scale height $H_0 = 10^3~\mathrm{cm}$.  This is about one
quarter of domain width.  The run at $20~\mbox{cm}$ resolution used
one fewer level of refinement.  The slower rotating case using a
slightly wider domain to accommodate the expected larger Rossby
length.  We note that the entire simulation framework for these
calculations is freely available in the \castro\ github
repository\footnote{\url{https://github.com/AMReX-Astro/Castro}, using
  the {\tt flame\_wave} setup.}.  In the discussions below, we'll use
the simulation names defined in the table to refer to specific runs
and we'll use the 10/10 run as the reference calculation.

\begin{deluxetable}{lcccccc}
\tablecaption{\label{table:sim_names} Simulation parameters.}
\tablehead{\colhead{name} &
           \colhead{reaction} &
           \colhead{conductivity} &
           \colhead{fine grid} &
           \colhead{rotation} &
           \colhead{domain size} &
           \colhead{network} \\
%
           \colhead{} &
           \colhead{boost} &
           \colhead{boost} &
           \colhead{resolution} &
           \colhead{rate} &
           \colhead{} &
           \colhead{}
}
\startdata
10/10 & 10 & 10 & 10~cm & 2000~Hz & $1.2288\times 10^5~\mbox{cm} \times 3.072\times 10^4~\mbox{cm}$ & {\tt aprox13} \\
10/10-slow & 10 & 10 & 10~cm & 1000~Hz & $1.8432\times 10^5~\mbox{cm} \times 3.072\times 10^4~\mbox{cm}$ &{\tt aprox13} \\
10/10-lores & 10 & 10 & 20~cm & 2000~Hz & $1.2288\times 10^5~\mbox{cm} \times 3.072\times 10^4~\mbox{cm}$ &{\tt aprox13} \\
5/5 & 5 & 5 & 10~cm & 2000~Hz & $1.2288\times 10^5~\mbox{cm} \times 3.072\times 10^4~\mbox{cm}$  &{\tt aprox13} \\
10/10-iso7 & 10 & 10 & 10~cm & 2000~Hz & $1.2288\times 10^5~\mbox{cm} \times 3.072\times 10^4~\mbox{cm}$ &{\tt iso7} \\
\enddata
\end{deluxetable}





\subsection{General Features}

Figure~\ref{fig:time_series} shows the time evolution of the 10/10
simulation, focusing on the mean-molecular weight,
\begin{equation}
\bar{A} = \left ( \sum_k \frac{X_k}{A_k} \right )^{-1}
\end{equation}
In each frame the buffer of \isot{Ni}{56} that serves as the
underlying neutron star is seen spanning the bottom of the domain.
Above that, the composition begins as pure \isot{He}{4}, but as the
simulation progresses, the flame processes this to heavier nuclei,
increasing $\bar{A}$.  By about $10~\mbox{ms}$, we see the flame front
is reasonable well-defined.  We see that the bottom of the burning
front is listed off of the base of the atmosphere, greatly increasing
the surface area of the burning compared with a perfectly vertical
flame front.  By $20~\mbox{ms}$ the flame has moved out substantially
and we are beginning to see a gradient in the composition of the ash,
with the heavier nuclei furthest behind the flame.  The boosting of
the burning likely artificially increases this effect.


% made with time_series.py in flame_wave_boost_10_10_paper with --skip 100
\begin{figure}[h]
\centering
\plotone{time_series}
\caption{\label{fig:time_series} Time series of the mean molecular weight of the 
flame for our standard 10/10 simulation.}
\end{figure}

Figure~\ref{fig:10_10_overview} shows the temperature, energy
generation rate, and $w$ component of the velocity (the out-of-plane
velocity induced by the Coriolis force).  This latter field
illustrates the hurricane effect setup by the laterally spreading
burning front.  In the energy generation rate plot, we see that the
burning is mostly concentrated toward the bottom of the layer, as
expected since the density is greatest there.  We see that the peak of
the burning has moved off of the symmetry axis, demonstrating that the
burning front is propagating to the right.  In the temperature plot,
we see the effect of our refinement criteria focusing only on the
atmosphere where we are most dense, with an artificial drop in the
temperature at the refinement boundary likely due to the thermal
diffusion.  As we'll see when the lower resolution case, we do not
believe that this affects the results.

A final feature worth noting is the ash that seems to move along the
surface at a higher velocity than the flame, via surface gravity
waves.  The sponging that we perform is likely damping this some, and
the method by which we initialize the flame may induce a larger
transient than in nature, nevertheless, this surface ash is intriguing
because it affects the composition of the photosphere ahead of the
burning front, potentially changing our interpretation of
observations.  This is something that will be explored more fully in
the future.

% made with overview.py
\begin{figure}[h]
\centering
\plotone{flame_wave_boost_10_10_overview}
\caption{\label{fig:10_10_overview} Temperature, energy generation rate, and out-of-plane velocity for the 10/10 simulation at 20~ms.}
\end{figure}


Our default resolution puts $\sim 80$ zones vertically in the
``cool'' \MarginPar{double check} model atmosphere height.  To
understand the effects of resolution, we also perform a run with one
fewer level, giving at $20~\mbox{cm}$ resolution overall (this is our
10/10-lores run).  Figure~\ref{fig:10_10_lowres} shows the fields at
20~ms.  The structure is largely the same as the 10/10 run, with
largely the same flame shape and position, and the same structure in
the energy generation rate.  Since we do not have a jump in refinement
right below the atmosphere, we don't see the strong diffusive effect
there, but we do see some cooling in the \isot{Ni}{56} region.  This
is likely a resolution effect, again due to the strong gradient in
temperature at the base of the atmosphere.


% made with slice_multi_crop.py
\begin{figure}[h]
\centering
\plotone{flame_wave_boost_10_10_lowres_slice}
\caption{\label{fig:10_10_lowres} Temperature, mean molecular weight,
  energy generation rate, and out-of-plane velocity for the 10/10 low
  resolution simulation at 20~ms.}
\end{figure}




\subsection{Effects of our Approximations}

The results above all used a boosting of 10/10.  To see how the
results are sensitivity to our boosting we ran a simulation with a
boosting of 5/5.  This is shown in Figure~\ref{fig:5_5_overview}.  The
results look qualitatively the same---a laterally propagating flame
develops that is lifted off of the bottom of the fuel layer.  The
flame has not moved out as far as in the 10/10 simulation, simply
because there is less energy release, but we expect that if we were to
run this out twice as long, the flame would have advanced to the
position seen in our 10/10 runs.  The good agreement in the structure
of the flame seen with the lower boosting gives us confidence that the
overall aspects of the flame structure and acceleration we are seeing
are robust to the approximations we make.

\begin{figure}[h]
\centering
\plotone{flame_wave_boost_5_5_slice}
\caption{\label{fig:5_5_overview} Temperature, mean molecular weight, energy generation rate, and out-of-plane velocity for the 5/5 simulation at 20~ms.}
\end{figure}

We can also look at the 

\subsection{Flame Acceleration}

To measure the propagation rate of the burning front, we track the
location beyond the hottest part of the flame where
$\dot{e}_\mathrm{nuc}$ first drops to less than $0.1 \%$ of
maximum. Using $0.1 \%$ of the peak value rather than the peak value
itself helps to mitigate fluctuations caused by localized fluid
motions, as the reduced value lies in a less turbulent part of the
burning region. We also average $\dot{e}_\mathrm{nuc}$ over the
vertical coordinate to further dampen these fluctuations. After an
initial transient period typically spanning $\sim 3$ ms, the flame
settles into a state of steady propagation. The position data here are
well fitted by a linear function of time (as seen in Figure
\ref{fig:flame_speed}), and the resulting slope gives the velocity of
the flame front.

Table \ref{table:flame_speeds} gives the flame speed measured in each
multidimensional simulation, as well as laminar speeds obtained from
any corresponding 1D runs. The 2D flames propagate at speeds about an
order of magnitude faster than their 1D counterparts. The increase in
flame speed is likely a product of hydrodynamical effects such as
turbulence, wrinkling, and convective cycles, which bring cooler fuel
from ahead of the front into the hottest part of the burning region.

\begin{deluxetable}{lcc}
	\tablecaption{\label{table:flame_speeds}Turbulent and laminar flame speeds.}
	\tablehead{\colhead{run} & \colhead{$s_T$ (km s$^{-1}$)} & \colhead{$s_L$ (km s$^{-1}$)}}
	\startdata
	10/10 boost & $9.18 \pm 0.03$  & $1.06 \pm 0.01$ \\
	5/5 boost & $4.00 \pm 0.01$ & $0.56 \pm 0.01$ \\
	iso7 network & $7.56 \pm 0.02$ & $-$ \\
	lowres & $9.33 \pm 0.04$ & $-$ \\
	1000 Hz & $18.6 \pm 0.16$ & $-$ \\
	\enddata
	%
\end{deluxetable}

\begin{figure}[t]
\centering
\plotone{plots/results/speedplot_all.pdf}
\caption{\label{fig:flame_speed}The position of the burning front for each simulation run as a function of time. The dashed lines show linear least squares fits for $t \gtrsim 6$ ms.}
\end{figure}

{\color{red} Should say qualitatively what the plot shows}

\subsection{Entrainment, Flow Features}

\AssignTo{Alice, Don}

Show the phase plots and whatever it was that Don cooked up.

Baroclinicity is calculated as
\begin{equation}
    \boldsymbol{\psi} = \frac{1}{\rho^2} \nablab p \times \nablab \rho,
\end{equation}
and shows the misalignment of the local density and pressure gradients
\citep{Malone2014a}. As we are considering a 2d system, the component of the
baroclinicity we consider here (out of the plane) reflects the misalignment
of the fields in the plane of the simulation. Figure~\ref{fig:baroclinicity}
shows that the baroclinicity peaks along the flame front. This baroclinicity
generates vorticity, which in turn entrains material along the surface of the
flame.

\begin{figure}[t]
\centering
\plotone{plots/results/baroclinicity}
\caption{\label{fig:baroclinicity} Baroclinicity. This plot shows
$\ln \left(\mathbf{\psi}\right)$ at time $t = 0.02~\mathrm{s}$ for the 10/10 simulation.}
\end{figure}

We show some phase plots in Figure~\ref{fig:phase_plots}. The phase plot of the
energy generation rate against the as a function of
the $x$- and $y$-velocities shows that energy is preferentially generated in
regions with negative $x$-velocity. This most likely corresponds to material
being entrained along the underside of the flame front. There are loops of
points of similar $\dot{e}_\mathrm{nuc}$ in the outer edges of the plot.
These are likely to correspond to the vortices that appear within the flame,
and show the fluid moving in a circular motion in $u-v$ phase space.

The phase plot of the energy generation rate as a function of the $x$-velocity
and the density demonstrates that the energy generation rate peaks at
$\rho \sim 3 \times 10^5~\mathrm{g/cm}^3$, at the base of the flame.
\MarginPar{I'm guessing? It's at least towards the base}

\begin{figure}[t]
\centering
\plottwo{plots/results/vx_vy_129139}{plots/results/u_rho_129139}
\caption{\label{fig:phase_plots} Phase plots at time $t = 0.02~\mathrm{s}$.
\emph{Left}: Phase plot showing the energy generation rate as a function of
the $x$- and $y$-velocities. \emph{Right}: phase plot showing the energy
generation rate as a function of the $x$-velocity and the density.}
\end{figure}

Figure~\ref{fig:abar_temp} shows the energy generation rate as a function of the
temperature and $\bar{A}$. \MarginPar{I'm assuming will be defined somewhere above}
The energy generation rate peaks at low $\bar{A}$, where there is a high fraction
of unburnt material. The burning raises the temperature of this material, such
that the peak temperature coincides with the peak energy generation rate. The
material then cools again as the burning converts the fuel into ashes,
increasing $\bar{A}$ and reducing the energy generation rate as there is
less available fuel. Along the base of the plot we see cool unburnt fluid and
ashes. In the center of the plot there are thin `trails' in phase space, which
could correspond to less common reaction pathways in the reaction network.

\begin{figure}[t]
\centering
\plotone{plots/results/abar_temp_129139}
\caption{\label{fig:abar_temp} Phase plot of energy generation rate as a function of $\bar{A}$ and temperature at time $t = 0.02~\mathrm{s}$.}
\end{figure}





\section{Discussion}

\AssignTo{everyone}

Full hydrodynamics models of flame spreading in XRB are possible with certain approximations.

CPU cost

Accurate simulations require
\begin{enumerate}
\item Resolving the scale height of the atmosphere

\item Thin transition between underlying neutron star and atmosphere, to ensure that the peak temperature
hits right at the proper base density
\end{enumerate}

The main approximations we made were
\begin{enumerate}
\item A higher than normal rotation rate
\item A boosted flame
\item Simplified reaction network
\item A 2.5D model for the flow
\end{enumerate}
The first two of these were needed to reduce the spatial and temporal
scales needed to model, making the simulations feasible.  Using the
simulation framework developed here, these both can be relaxed in the
future, at the cost of more computer time.  The same goes for 2.5D
vs.\ full 3D---the only difference is computer time, and our future
calculations will explore the 3D evolution and compare to these.  In particular,
in 3D we will be able to explore shear instabilities at the flame front.
Larger networks are a straightforward change, and already supported in
\castro\ using the \pynucastro\ framework~\citep{pynucastro} and JINA
ReacLib rate database~\citep{reaclib}.

These studies complement the\MarginPar{???? Finish this sentence}

Other future work includes mixed H/He bursts.  This will require a
different reaction network, and perhaps different (more relaxed)
resolution requirements.

There are several changes we will pursue to the algorithm used to
model these XRBs.  First, as shown in \citet{castro-sdc}, we have
developed a fourth-order (in space and time) method for coupling
hydrodynamics and reactions that should greatly improve the accuracy
of the simulations.  We expect that by using this new high-order
algorithm we can drop a level of refinement from the simulations while
still accurately modeling the evolution.  We have also ported
\castro\ to GPUs, giving an order of magnitude performance boost on
nodes with both CPUs and GPUs.  The GPU-enabled algorithm will
increase the performance, allowing us to run bigger domains and/or
eliminate the boosting used here.

MHD, radiation transport


\acknowledgements \castro\ is open-source and freely available at
\url{http://github.com/AMReX-Astro/Castro}.  The problem setups used
here are available in the git repo as {\tt flame} and {\tt
  flame\_wave}.  The work at Stony Brook was supported by DOE/Office
of Nuclear Physics grant DE-FG02-87ER40317.  This research used
resources of the National Energy Research Scientific Computing Center,
a DOE Office of Science User Facility supported by the Office of
Science of the U.~S.\ Department of Energy under Contract
No.\ DE-AC02-05CH11231.  This research used resources of the Oak Ridge
Leadership Computing Facility at the Oak Ridge National Laboratory,
which is supported by the Office of Science of the U.S. Department of
Energy under Contract No. DE-AC05-00OR22725, awarded through the DOE
INCITE program.  This research has made use of NASA's Astrophysics
Data System Bibliographic Services.

\facilities{NERSC, OLCF}

\software{AMReX \citep{amrex_joss},
          Castro \citep{castro},
          GCC (\url{https://gcc.gnu.org/}),
          linux (\url{https://www.kernel.org/}),
          matplotlib (\citealt{Hunter:2007}, \url{http://matplotlib.org/}),
          NumPy \citep{numpy,numpy2},
          python (\url{https://www.python.org/}),
          valgrind \citep{valgrind},
          VODE \citep{vode},
          yt \citep{yt}}




\appendix

\section{Diffusion Tests}
\label{app:diffusion}

Show diffusion test problem and convergence.




%======================================================================
% References
%======================================================================

\bibliographystyle{aasjournal}
\bibliography{ws}


\end{document}
