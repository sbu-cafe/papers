\documentclass[times,modern]{elsarticle}

% these lines seem necessary for pdflatex to get the paper size right
\pdfpagewidth 8.5in
\pdfpageheight 11.0in

\usepackage[T1]{fontenc}
\usepackage{epsf,color,amsmath}

\usepackage{cancel}

\newcommand{\Sb}{{\bf S}}

\newcommand{\sfrac}[2]{\mathchoice%
  {\kern0em\raise.5ex\hbox{\the\scriptfont0 #1}\kern-.15em/
    \kern-.15em\lower.25ex\hbox{\the\scriptfont0 #2}}
  {\kern0em\raise.5ex\hbox{\the\scriptfont0 #1}\kern-.15em/
    \kern-.15em\lower.25ex\hbox{\the\scriptfont0 #2}}
  {\kern0em\raise.5ex\hbox{\the\scriptscriptfont0 #1}\kern-.2em/
    \kern-.15em\lower.25ex\hbox{\the\scriptscriptfont0 #2}} {#1\!/#2}}

\newcommand{\castro}{{\sf Castro}}
\newcommand{\amrex}{{\sf AMReX}}

\usepackage{bm}

\setlength{\marginparwidth}{0.75in}
\newcommand{\MarginPar}[1]{\marginpar{\vskip-\baselineskip\raggedright\tiny\sffamily\hrule\smallskip{\color{red}#1}\par\smallskip\hrule}}

\begin{document}
%======================================================================
% Title
%======================================================================
\title{Timestep Retries for Robust Simulations}

\author{Max P. Katz}
\author{Michael Zingale}

\maketitle

%======================================================================
% Abstract and Keywords
%======================================================================
\begin{abstract}
\end{abstract}

%======================================================================
% Introduction
%======================================================================
\section{Introduction}\label{Sec:Introduction}

Multiphysics simulation software is hard to implement robustly. One reason is that
physics modules usually have one or more failure modes, and the code may have
additional failure modes related to the coupling between modules. Some examples of
simulation failure modes in our field of astrophysics include:
\begin{itemize}
  \item Numerical integration failures in the system of ordinary differential equations (ODEs) representing nuclear fusion
  \item Numerical integration successes in coupled ODE systems that nevertheless result in nonphysical outcomes, like mass fraction > 1
  \item Convergence failures in iterative linear solvers, such as those used to solve the Poisson equation
  \item Convergence failures in iterative non-linear solvers, such as those used for radiation diffusion
  \item Failures in non-linear root finding, such as in non-monotonic equations of state
  \item Zones whose local velocity exceeds the CFL hydrodynamic stability criterion given the selected timestep
  \item Negative density in a zone after a hydrodynamic update
\end{itemize}
All of these failure modes (and more) are represented in the simulation code \castro\ \citep{castro_joss}.
In our experience, any useful science simulation will almost inevitably hit one of these failure modes.

There are (at least) three possible responses to such failures. First, the application can abort and tell the user
to restart from the latest checkpoint with some modified parameters in the hope of making the failure less likely.
This is undesirable because it can be very disruptive to a long-running simulation, and also because it may not be
clear what step will actually resolve the problem. Second, the application can attempt to patch over the issue.
For example, a negative density may be replaced by a small positive density, or a flux limiter could be applied
to prevent it from occurring in the first place \citep{positivity_preserving}. While this can often keep the
simulation going, it creates unphysical distortions in the simulation. Our anecdotal experience is that most
simulation codes use these first two approaches to deal with such failures.

A third approach, which we believe is generally more robust and less distortionary, is to retry a timestep
that encounters one of these failure modes. Typically one will use a smaller $\Delta t$ on the retry, although
other options exist, such as modifying the spatial resolution when using an adaptive mesh. One astrophysics
application that successfully does this, which we drew inspiration from for this work, is the stellar evolution
code MESA \citep{MESA}. MESA also has the notion of a ``backup'': a retry can be defined as a second attempt at
the current timestep, while a backup involves returning to an earlier timestep.

\section{Implementation Details}\label{Sec:Implementation}

A functional retry mechanism requires a decision scheme and associated code infrastructure for when to perform
a second attempt at the timestep. A retry scheme effectively turns a timestep into a \texttt{while} loop: attempt
an advance by $\Delta t$, and if it fails then attempt an advance by $\Delta t / 2$, and keep going until you have
reached $t_{i} + \Delta t$ where $t_{i}$ is the starting simulation time. So in both principle and practice,
implementing a retry mechanism is actually quite straightforward, with two complications as noted next.

First, depending on context, this scheme may require saving application state and then restoring that application
state at a later point. For example, \castro\ is a multi-level mesh refinement code where the fine levels are
subcycled with respect to the coarse levels \citep{castro,berger_colella}, and so fine levels typically need to
interpolate coarse data between time levels $t_{i}$ and $t_{i} + \Delta t$, where $\Delta t$ is the coarse grid
timestep. \castro\ does this by maintaining both an ``old'' state and a ``new'' state corresponding to those two
time levels. But subcycling within a given level due to retries will mean that at the end of the coarse step, the
``old'' state data will be at some later time, such as $t_{i} + 3\Delta t / 4$ in the case of two failures. This
needs to be reset back to the initial ``old'' state at $t_{i}$ for valid interpolation onto the fine grid. In \castro\
we handle this by making a backup copy of the data\footnote{For the GPU build of \castro, we save this backup copy
in host pinned memory since we are typically GPU memory constrained} at $t_{i}$ if (and only if) we encounter an
advance failure on the first try.

Second, this scheme may involve refactoring physics modules that perform irrevocable changes to application
state. One way to guarantee that the scheme works is for the sequence of physics modules corresponding to an
advance to be written as an idempotent operator: given an input ``old'' state at $t_{i}$, the advance updates
only the ``new'' state at $t_{i} + \Delta t$, and only in a way that if you re-run the timestep you get the same
result in the ``new'' state. This makes a retry scheme trivial: if you detect a failure, exit from the advance
immediately, and then retry the advance with a smaller timestep.

%======================================================================
% References
%======================================================================

\bibliographystyle{aasjournal}
\bibliography{ws}

\end{document}
